% rubber: module xelatex
\documentclass[12pt]{article}

\usepackage[spanish]{babel}
\usepackage[utf8]{inputenc}
% \usepackage{palatino}
\usepackage{verse}

\usepackage{fontspec}
\setmainfont{EB Garamond}

\date{}
\title{Poesía selecta de Pablo Neruda VI}
\begin{document}
\maketitle
\tableofcontents
\clearpage
\poemtitle {Las manos negativas (1968)}
\begin{verse}

Cuándo me vio ninguno\\
cortando tallos, aventando el trigo?\\
Quién soy, si no hice nada?\\
Cualquiera, hijo de Juan,\\
tocó el terreno\\
y dejó caer algo\\
que entró como la llave\\
entra en la cerradura\\
y la tierra se abrió de par en par.  

Yo no, no tuve tiempo,\\
ni enseñanza:\\
guardé las manos limpias\\
del cadáver urbano,\\
me despreció la grasa de las ruedas,\\
el barro inseparable de las costumbres claras\\
se fue a habitar sin mí las provincias silvestres:\\
la agricultura nunca se ocupó de mis libros\\
y sin tener qué hacer, perdido en las bodegas,\\
reconcentré mis pobres preocupaciones\\
hasta que no viví sino en las despedidas.  

Adiós dije al aceite, sin conocer la oliva,\\
y al tonel, un milagro de la naturaleza,\\
dije también adiós porque no comprendía\\
cómo se hicieron tantas cosas sobre la tierra\\
sin el consentimiento de mis manos inútiles.  

\end{verse}

\clearpage
\poemtitle {El culto (1969)}
\begin{verse}

Ay qué pasión la que cantaba\\
entre la sangre y la esperanza:\\
el mundo quería nacer\\
después de morir tantas veces:\\
los ojos no tenían lágrimas\\
después de haber llorado tanto.  

No había nada en las arterias,\\
todo se había desangrado\\
y sin embargo se arregló\\
otra vez el pecho del hombre.\\
Se levantaron las ciudades,\\
fueron al mar los marineros,\\
tuvieron niños las escuelas,\\
y los pájaros, en el bosque,\\
pusieron sus huevos fragantes\\
sobre los árboles quemados.  

Pero fue duro renovar\\
la sonrisa de la esperanza:\\
se plantaba en algunos rostros\\
y se les caía a la calle\\
y en verdad pareció imposible\\
rellenar de nuevo la tierra\\
con tantos huecos que dejó\\
la dentellada del desastre.  

Y cuando ya crecieron las flores,\\
las cinerarias del olvido,\\
un hombre volvió de Siberia\\
y recomenzó la desdicha.  

Y si las manos de la guerra,\\
las terribles manos del odio\\
nos hundieron de no creer,\\
de no comprender la razón,\\
de no conocer la locura,\\
siempre fue ajena aquella culpa\\
y ahora sin comprender nada\\
y sin conocer la verdad\\
nos pegamos en las paredes\\
de los errores y dolores\\
que partían desde nosotros\\
y estos tormentos otra vez\\
se acumularon en mi alma.  

\end{verse}

\clearpage
\poemtitle {Las guerras (1969)}
\begin{verse}

Ven acá, sombrero caído,\\
zapato quemado, juguete,\\
o montón póstumo de anteojos,\\
o bien, hombre, mujer, ciudad,\\
levántense de la ceniza\\
hasta esta página cansada,\\
destituida por el llanto.\\
Ven, nieve negra, soledad\\
de la injusticia siberiana,\\
restos raídos del dolor,\\
cuando se perdieron los vínculos\\
y se abrumó sobre los justos\\
la noche sin explicaciones.  

Muñeca del Asia quemada\\
por los aéreos asesinos,\\
presenta tus ojos vacíos\\
sin la cintura de la niña\\
que te abandonó cuando ardía\\
bajo los muros incendiados\\
o en la muerte del arrozal.  

Objetos que quedaron solos\\
cerca de los asesinados\\
de aquel tiempo en que yo viví\\
avergonzado por la muerte\\
de los otros que no vivieron.  

De ver la ropa tendida\\
a secar en el sol brillante\\
recuerdo las piernas que faltan,\\
los brazos que no las llenaron,\\
partes sexuales humilladas\\
y corazones demolidos.  

Un siglo de zapaterías\\
llenó de zapatos el mundo\\
mientras cercenaban los pies\\
o por la nieve o por el fuego\\
o por el gas o por el hacha!  

A veces me quedo agachado\\
de tanto que pesa en mi espalda\\
la repetición del castigo:\\
me costó aprender a morir\\
con cada muerte incomprensible\\
y llevar los remordimientos\\
del criminal innecesario:\\
porque después de la crueldad\\
y aun después de la venganza\\
no fuimos tal vez inocentes\\
puesto que seguimos viviendo\\
cuando mataban a los otros.  

Tal vez les robamos la vida\\
a nuestros hermanos mejores.  

\end{verse}

\clearpage
\poemtitle {Metamorfosis (1969)}
\begin{verse}

He recibido un puntapié\\
del tiempo y se ha desordenado\\
el triste cajón de la vida.\\
El horario se atravesó\\
como doce perdices pardas\\
en un camino polvoriento\\
y lo que antes fue la una\\
pasó a ser las ocho cuarenta\\
y el mes de abril retrocedió\\
hasta transformarse en noviembre.  

Los papeles se me perdieron,\\
no se encontraban los recibos,\\
se llenaron los basureros\\
con nombres de contribuyentes,\\
con direcciones de abogados\\
y números de deliciosas.  

Fue una catástrofe callada.  

Comenzó todo en un domingo\\
que en vez de sentirse dorado\\
se arrepintió de la alegría\\
y se portó tan lentamente\\
como una tortuga en la playa:\\
no llegó nunca al día lunes.  

Al despertarme me encontré\\
más descabellado que nunca,\\
sin precedentes, olvidado\\
en una semana cualquiera,\\
como una valija en un tren\\
que rodara a ninguna parte\\
sin conductor ni pasajeros.  

No era un sueño porque se oyó\\
un mugido espeso de vaca\\
y luego trajeron la leche\\
con calor aún de las ubres,\\
además de que me rodeaba\\
un espectáculo celeste:\\
la travesura de los pájaros\\
entre las hojas y la niebla.  

Pero lo grave de este asunto\\
es que no continuaba el tiempo.\\
Todo seguía siendo sábado\\
hasta que el viernes se asomaba.  

Adónde voy? Adónde vamos?\\
A quién podía consultar?  

Los monumentos caminaban\\
hacia atrás, empujando el día\\
como guardias inexorables.\\
Y se desplomaba hacia ayer\\
todo el horario del reloj.  

No puedo mostrar a la gente\\
mi colección de escalofríos:\\
me sentí solo en una casa\\
perforada por las goteras\\
de un aguacero inapelable\\
y para no perder el tiempo,\\
que era lo único perdido,\\
rompí los últimos recuerdos,\\
me despedí de mi botica,\\
eché al fuego los talonarios,\\
las cartas de amor, los sombreros,\\
y como quien se tira al mar\\
yo me tiré contra el espejo.  

Pero ya no me pude ver.\\
Sentía que se me perdía\\
el corazón precipitado\\
y mis brazos disminuyeron,\\
se desmoronó mi estatura,\\
a toda velocidad\\
se me borraban los años,\\
regresó mi cabellera,\\
mis dientes aparecieron.  

En un fulgor pasé mi infancia,\\
seguí contra el tiempo en el cauce\\
hasta que no vi de mí mismo,\\
de mi retrato en el espejo\\
sino una cabeza de mosca,\\
un microscópico huevillo\\
volviendo otra vez al ovario.  

\end{verse}

\clearpage
\poemtitle {Siempre yo (1969)}
\begin{verse}

Yo que quería hablar del siglo\\
adentro de esta enredadera\\
que es mi siempre libro naciente,\\
por todas partes me encontré\\
y se me escapaban los hechos.\\
Con buena fe que reconozco\\
abrí los cajones al viento,\\
los armarios, los cementerios,\\
los calendarios con sus meses\\
y por las grietas que se abrían\\
se me aparecía mi rostro.  

Por más cansado que estuviera\\
de mi persona inaceptable\\
volvía a hablar de mi persona\\
y lo que me parece peor\\
es que me pintaba a mí mismo\\
pintando un acontecimiento.  

Qué idiota soy dije mil veces\\
al practicar con maestría\\
las descripciones de mí mismo\\
como si no hubiera habido\\
nada mejor que mi cabeza,\\
nadie mejor que mis errores.  

Quiero saber, hermanos míos,\\
dije en la Unión de Pescadores,\\
si todos se aman como yo.\\
La verdad es —me contestaron—\\
que nosotros pescamos peces\\
y tú te pescas a ti mismo\\
y luego vuelves a pescarte\\
y a tirarte al mar otra vez.  

\end{verse}

\clearpage
\poemtitle {Condiciones (1969)}
\begin{verse}

Con tantas tristes negativas\\
me despedí de los espejos\\
y abandoné mi profesión:\\
quise ser ciego en una esquina\\
y cantar para todo el mundo\\
sin ver a nadie porque todos\\
se me parecían un poco.  

Pero buscaba mientras tanto\\
cómo mirarme hacia detrás,\\
hacia donde estaba sin ojos\\
y era oscura mi condición.\\
No saqué nada con cantar\\
como un ciego del populacho:\\
mientras más amarga la calle\\
me parecía yo más dulce.  

Condenado a quererme tanto\\
me hice un hipócrita exterior\\
ocultando el amor profundo\\
que me causaban mis defectos.\\
Y así sigo siendo feliz\\
sin que jamás se entere nadie\\
de mi enfermedad insondable:\\
de lo que sufrí por amarme\\
sin ser, tal vez, correspondido.  

\end{verse}

\clearpage
\poemtitle {Hoy es el día más (1969)}
\begin{verse}

Hoy es el día más, el que traía\\
una desesperada claridad que murió.\\
Que no lo sepan los agazapados:\\
todo debe quedar entre nosotros,\\
día, entre tu campana\\
y mi secreto.  

Hoy es el ancho invierno de la comarca olvidada\\
que con una cruz en el mapa y un volcán en la nieve\\
viene a verme, a volverme, a devolverme el agua\\
desplomada en el techo de mi infancia.\\
Hoy cuando el sol comenzó con sus espigas\\
a contar el relato más claro y más antiguo\\
como una cimitarra cayó la oblicua lluvia,\\
la lluvia que agradece mi corazón amargo.  

Tú, mi bella, dormida aún en agosto,\\
mi reina, mi mujer, mi extensión, geografía,\\
beso de barro, cítara que cubren los carbones,\\
tú, vestidura de mi porfiado canto,\\
hoy otra vez renaces y con el agua negra\\
del cielo me confundes y me obligas:\\
debo reanudar mis huesos en tu reino,\\
debo aclarar aún mis deberes terrestres.  

\end{verse}

\clearpage
\poemtitle {Rhodo y Rosía (1969)}
\begin{verse}

Rhodo, pétreo patriarca, la vio sin verla, era\\
Rosía, hija cesárea, labradora.  

Ancha de pechos, breve de boca y ojos,\\
salía a buscar agua y era un cántaro,\\
salía a lavar ropa y era pura,\\
cruzaba por la nieve y era nieve,\\
era estática como el ventisquero,\\
invisible y fragante era Rosía Raíz.  

Rhodo la destinó, sin saberlo, al silencio.  

Era el cerco glacial de la naturaleza:\\
de Aysén al Sur la Patagonia infligió\\
las desoladas cláusulas del invierno terrestre.  

La cabeza de Rhodo vivía en la bruma,\\
de cicatriz en cicatriz volcánica,\\
sin cesar a caballo, persiguiendo\\
el olor, la distancia, la paz de las praderas.  

\end{verse}

\clearpage
\poemtitle {Aquí termina y comienza éste libro (1970)}
\begin{verse}

Dice Rhodo: Yo me consumí\\
en aquel reino que quise fundar\\
y no sabía ya que estaba solo.\\
Fue mi noción quebrantar esa herencia\\
de sangre y sociedad: deshabitarme.\\
Y cuando dominé la paz terrible\\
de las praderas, de los ventisqueros,\\
me hallé más solitario que la nieve.  

Fue entonces: tú llegaste del incendio\\
y con la autoridad de tu ternura\\
comencé a continuarme y a extenderme.  

Tú eres el infinito que comienza.  

Tan simple tú, hierba desamparada\\
de matorral, me hiciste despertar\\
y yo te desperté, cuando los truenos\\
del volcán decidieron avisarnos\\
que el plazo se cumplía\\
yo no quise extinguirte ni extinguirme.  

\end{verse}

\clearpage
\poemtitle {A plena ola (1970)}
\begin{verse}

\emph{Es muy serio el viento del mes de marzo en el océano:\\
sin miedo: es día claro, sol ilustre,\\
yo con mil otros encima del mar\\
en la nave italiana que retorna a Nápoli.}  

\emph{Tal vez trajeron todos sus infidelidades,\\
enfermedades, tristes papeles, deudas, lágrimas,\\
dineros y derrotas en los números:\\
pero aquí arriba es difícil jugar con la razón\\
o complacerse con las desdichas ajenas\\
o mantenerse heridos por angas o por mangas:\\
hay tal ventolera que no se puede sufrir:\\
y como no veníamos preparados\\
aún para ser felices, aún y sin embargo\\
y subimos puentes y escalas para reflexionar,\\
el viento nos borró la cabeza, es extraño:\\
de inmediato sentimos que estábamos mejor:\\
sin cabeza se puede discutir con el viento.}  

\emph{A todos, melancólicos de mi especialidad,\\
los que inútilmente cargamos con pesadumbre propia\\
y ajena, los que pensamos tanto en las pequeñas cosas\\
hasta que crecen y son más grandes que nosotros,\\
a todos recomiendo mi claro tratamiento:\\
la higiene azul del viento en un día de sol,\\
un golpe de aire furioso y repetido\\
en el espacio atlántico sobre un barco en el mar,\\
dejando sí constancia de que la salud física\\
no es mi tema: es el alma mi cuidado:\\
quiero que las pequeñas cosas que nos desgarran\\
sigan siendo pequeñas, impares y solubles\\
para que cuando nos abandone el viento\\
veamos frente a frente lo invisible.}  

\end{verse}

\clearpage
\poemtitle {La tierra (1971)}
\begin{verse}

Amarillo, amarillo sigue siendo\\
el perro que detrás del otoño circula\\
haciendo entre las hojas circunferencias de oro,\\
ladrando hacia los días desconocidos.  

Así veréis lo imprevisto de ciertas situaciones:\\
junto al explorador de las terribles fronteras\\
que abren el infinito, he aquí el predilecto,\\
el animal perdido del otoño.\\
Qué puede cambiar de tierra a tiempo, de sabor a estribor,\\
de luz velocidad a circunstancia terrestre?\\
Quién adivinará la semilla en la sombra\\
si como cabelleras las mismas arboledas\\
dejan caer rocío sobre las mismas herraduras,\\
sobre las cabezas que reúne el amor,\\
sobre las cenizas de corazones muertos?  

Este mismo planeta, la alfombra de mil años,\\
puede florecer pero no acepta la muerte ni el reposo:\\
las cíclicas cerraduras de la fertilidad\\
se abren en cada primavera para las llaves del sol\\
y resuenan los frutos haciéndose cascada,\\
sube y baja el fulgor de la tierra a la boca\\
y el humano agradece la bondad de su reino.  

Alabada sea la vieja tierra color de excremento,\\
sus cavidades, sus ovarios sacrosantos,\\
las bodegas de la sabiduría que encerraron\\
cobre, petróleo, imanes, ferreterías, pureza,\\
el relámpago que parecía bajar desde el infierno\\
fue atesorado por la antigua madre de las raíces\\
y cada día salió el pan a saludarnos\\
sin importarle la sangre y la muerte que vestimos los hombres,\\
la maldita progenie que hace la luz del mundo.  

\end{verse}

\clearpage
\poemtitle {La rosa separada (1971)}
\begin{verse}

{\bfseries\scshape {III. La isla}}

Antigua Rapa Nui, patria sin voz,\\
perdónanos a nosotros los parlanchines del mundo:\\
hemos venido de todas partes a escupir en tu lava,\\
llegamos llenos de conflictos, de divergencias, de sangre,\\
de llanto y digestiones, de guerras y duraznos,\\
en pequeñas hileras de inamistad, de sonrisas\\
hipócritas, reunidos por los dados del cielo\\
sobre la mesa de tu silencio.  

Una vez más llegamos a mancillarte.  

Saludo primero al cráter, a Ranu Raraku, a sus párpados\\
de légamo, a sus viejos labios verdes:\\
es ancho, y altos muros lo circulan, lo encierran,\\
pero el agua allá abajo, mezquina, sucia, negra,\\
vive, se comunica con la muerte\\
como una iguana inmóvil, soñolienta, escondida.  

Yo, aprendiz de volcanes, conocí,\\
infante aún, las lenguas de Aconcagua,\\
el vómito encendido del volcán Tronador,\\
en la noche espantosa vi caer\\
la luz del Villarrica fulminando las vacas,\\
torrencial, abrasando plantas y campamentos,\\
crepitar derribando peñascos en la hoguera.  

Pero si aquí me hubiera dejado mi infancia,\\
en este volcán muerto hace mil años,\\
en este Ranu Raraku, ombligo de la muerte,\\
habría aullado de terror y habría obedecido:\\
habría deslizado mi vida en silencio,\\
hubiera caído al miedo verde, a la boca del cráter desdentado,\\
transformándome en légamo, en lenguas de la iguana.  

Silencio depositado en la cuenca, terror\\
de la boca lunaria, hay un minuto, una hora\\
pesada como si el tiempo detenido\\
se fuera a convertir en piedra inmensa:\\
es un momento, pronto\\
también disuelve el tiempo su nueva estatua imposible\\
y queda el día inmóvil, como un encarcelado\\
dentro del cráter, dentro de la cárcel del cráter,\\
adentro de los ojos de la iguana del cráter.  

{\bfseries\scshape {VI. La isla}}

Oh Melanesia, espiga poderosa,\\
islas del viento genital, creadas,\\
luego multiplicadas por el viento.  

De arcilla, bosques, barro, de semen que volaba\\
nació el collar salvaje de los mitos:\\
Polinesia: pimienta verde, esparcida\\
en el área del mar por los dedos errantes\\
del dueño de Rapa Nui, el Señor Viento.\\
La primera estatua fue de arena mojada,\\
él la formó y la deshizo alegremente.\\
La segunda estatua la construyó de sal\\
y el mar hostil la derribó cantando.\\
Pero la tercera estatua que hizo el Señor Viento\\
fue un moai de granito, y éste sobrevivió.  

Esta obra que labraron las manos del aire,\\
los guantes del cielo, la turbulencia azul,\\
este trabajo hicieron los dedos transparentes:\\
un torso, la erección del Silencio desnudo,\\
la mirada secreta de la piedra,\\
la nariz triangular del ave o de la proa\\
y en la estatua el prodigio de un retrato:\\
porque la soledad tiene este rostro,\\
porque el espacio es esta rectitud sin rincones,\\
y la distancia es esta claridad del rectángulo.  

{\bfseries\scshape {VIII. La isla}}

Los rostros derrotados en el centro,\\
quebrados y caídos, con sus grandes narices\\
hundidas en la costra calcárea de la isla,\\
los gigantes indican a quién? a nadie?\\
un camino, un extraño camino de gigantes:\\
allí quedaron rotos cuando avanzaron, cayeron\\
y allí quedó su peso prodigioso caído,\\
besando la ceniza sagrada, regresando\\
al magma natalicio, malheridos, cubiertos\\
por la luz oceánica, la corta lluvia, el polvo\\
volcánico, y más tarde\\
por esta soledad del ombligo del mundo:\\
la soledad redonda de todo el mar reunido.  

Parece extraño ver vivir aquí, dentro\\
del círculo, contemplar las langostas\\
róseas, hostiles caer a los cajones\\
desde las manos de los pescadores,\\
y éstos, hundir los cuerpos otra vez en el agua\\
agrediendo las cuevas de su mercadería,\\
ver las viejas zurcir pantalones gastados\\
por la pobreza, ver entre follajes  

la flor de una doncella sonriendo a sí misma,\\
al sol, al mediodía tintineante,\\
a la iglesia del padre Englert, allí enterrado,\\
sí, sonriendo, llena de esta dicha remota\\
como un pequeño cántaro que canta.  

{\bfseries\scshape {X. Los hombres}}

Sí, próximos desengañados, antes de regresar\\
al redil, a la colmena de las tristes abejas,\\
turistas convencidos de volver, compañeros\\
de calle negra con casas de antigüedades\\
y latas de basura, hermanastros\\
del número treinta y tres mil cuatrocientos veintisiete,\\
piso sexto, departamento a, be o jota\\
frente al almacén «Astorquiza, Williams y Compañía»\\
sí, pobre hermano mío que eres yo,\\
ahora que sabemos que no nos quedaremos\\
aquí, ni condenados, que sabemos\\
desde hoy, que este esplendor nos queda grande,\\
la soledad nos aprieta como el traje de un niño\\
que crece demasiado o como cuando\\
la oscuridad se apodera del día.  

{\bfseries\scshape {XI. Los hombres}}

Se ve que hemos nacido para oírnos y vernos,\\
para medirnos (cuánto saltamos, cuánto ganamos, ganamos, etcétera),\\
para ignorarnos (sonriendo), para mentirnos,\\
para el acuerdo, para la indiferencia o para\\
comer juntos.\\
Pero que no nos muestre nadie la tierra, adquirimos\\
olvido, olvido hacia los sueños de aire,\\
y nos quedó sólo un regusto de sangre y polvo\\
en la lengua: nos tragamos el recuerdo\\
entre vino y cerveza, lejos, lejos de aquello,\\
lejos de aquello, de la madre, de la tierra de la vida.  

{\bfseries\scshape {XIII. Los hombres}}

Llegamos hasta lejos, hasta lejos\\
para entender las órbitas de piedra,\\
los ojos apagados que aún siguen mirando,\\
los grandes rostros dispuestos para la eternidad.  

{\bfseries\scshape {XIV. Los hombres}}

Qué lejos, lejos, lejos continuamos,\\
nos alejamos de las duras máscaras\\
erigidas en pleno silencio y nos iremos\\
envueltos en su orgullo, en su distancia.\\
Y para qué vinimos a la isla?\\
No será la sonrisa de los hombres floridos,\\
ni las crepitantes caderas de Ataroa la bella,\\
ni los muchachos a caballo, de ojos impertinentes,\\
lo que nos llevaremos regresando:\\
sino un vacío oceánico, una pobre pregunta\\
con mil contestaciones de labios desdeñosos.  

{\bfseries\scshape {XVI. Los hombres}}

El fatigado, el huérfano\\
de las multitudes, el yo,\\
el triturado, el del cemento,\\
el apátrida de los restauranes repletos,\\
el que quería irse más lejos, siempre,\\
no sabía qué hacer en la isla, quería\\
y no quería quedarse o volver,\\
el vacilante, el híbrido, el enredado en sí mismo\\
aquí no tuvo sitio: la rectitud de piedra,\\
la mirada infinita del prisma de granito,\\
la soledad redonda lo expulsaron:\\
se fue con sus tristezas a otra parte,\\
regresó a sus natales agonías,\\
a las indecisiones del frío y del verano.  

{\bfseries\scshape {XVII. La isla}}

Oh torre de la luz, triste hermosura\\
que dilató en el mar estatuas y collares,\\
ojo calcáreo, insignia del agua extensa, grito\\
de petrel enlutado, diente del mar, esposa\\
del viento de oceanía, oh rosa separada\\
del tronco del rosal despedazado\\
que la profundidad convirtió en archipiélago,\\
oh estrella natural, diadema verde,\\
sola en tu solitaria dinastía,\\
inalcanzable aún, evasiva, desierta\\
como una gota, como una uva, como el mar.  

{\bfseries\scshape {XVIII. Los hombres}}

Como algo que sale del agua, algo desnudo, invicto,\\
párpado de platino, crepitación de sal,\\
alga, pez tembloroso, espada viva,\\
yo, fuera de los otros, me separo\\
de la isla separada, me voy\\
envuelto en luz\\
y si bien pertenezco a los rebaños,\\
a los que entran y salen en manadas,\\
al turismo igualitario, a la prole,\\
confieso mi tenaz adherencia al terreno\\
solicitado por la aurora de Oceanía.  

{\bfseries\scshape {XXI. Los hombres}}

Yo, de los bosques, de los ferrocarriles en invierno,\\
yo, conservador de aquel invierno,\\
del barro\\
en una calle agobiada, miserable,\\
yo, poeta oscuro, recibí el beso de piedra en mi frente  

{\bfseries\scshape {XXII. La isla}}

Amor, amor, oh separada mía\\
por tantas veces mar como nieve y distancia,\\
mínima y misteriosa, rodeada\\
de eternidad, agradezco\\
no sólo tu mirada de doncella,\\
tu blancura escondida, rosa secreta, sino\\
el resplandor moral de tus estatuas,\\
la paz abandonada que impusiste en mis manos\\
el día detenido en tu garganta.  

{\bfseries\scshape {XXIII. Los hombres}}

Porque si coincidiéramos allí\\
como los elefantes moribundos\\
dispuestos al oxigeno total,\\
si armados los satisfechos y los hambrientos,\\
los árabes y los bretones, los de Tehuantepec\\
y los de Hamburgo, los duros de Chicago y los senegaleses,\\
todos, si comprendiéramos que allí guardan las llaves\\
de la respiración, del equilibrio\\
basados en la verdad de la piedra y del viento,\\
si así fuera y corrieran las razas despoblándose\\
las naciones,\\
si navegáramos en tropel hacia la Isla,\\
si todos fueran sabios de golpe y acudiéramos\\
a Rapa Nui, la mataríamos,\\
la mataríamos con inmensas pisadas, con dialectos,\\
escupos, batallas, religiones,\\
y allí también se acabaría el aire,\\
caerían al suelo las estatuas,\\
se harían palos sucios las narices de piedra  

{\bfseries\scshape {XXIV. La isla}}

Adiós, adiós, isla secreta, rosa\\
de purificación, ombligo de oro:\\
volvemos unos y otros a las obligaciones\\
de nuestras enlutadas profesiones y oficios.  

Adiós, que el gran océano te guarde\\
lejos de nuestra estéril aspereza!  

Ha llegado la hora de odiar la soledad:\\
esconde, isla, las llaves antiguas\\
bajo los esqueletos\\
que nos reprocharán hasta que sean polvo\\
en sus cuevas de piedra\\
nuestra invasión inútil.  

Regresamos. Y este adiós, prodigado y perdido\\
es uno más, un adiós\\
sin más solemnidad que la que allí se queda:\\
la indiferencia inmóvil en el centro del mar:\\
cien miradas de piedra que miran hacia adentro  

\end{verse}

\clearpage
\poemtitle {El cobarde (1971)}
\begin{verse}

Y ahora, a dolerme el alma y todo el cuerpo,\\
a gritar, a escondernos en el pozo\\
de la infancia, con miedo y ventarrón:\\
hoy nos trajo el sol joven del invierno\\
una gota de sangre, un signo amargo\\
y ya se acabó todo: no hay remedio,\\
no hay mundo, ni bandera prometida:\\
basta una herida para derribarte:\\
con una sola letra\\
te mata el alfabeto de la muerte,\\
un solo pétalo del gran dolor humano\\
cae en tu orina y crees\\
que el mundo se desangra.  

Así, con sol frío de Francia, en mes de marzo,\\
a fines del invierno dibujado\\
por negros árboles de la Normandía\\
con el cielo entreabierto ya al destello\\
de dulces días, flores venideras,\\
yo encogido, sin calles ni vitrinas,\\
callada mi campana de cristal,\\
con mi pequeña espina lastimosa\\
voy sin vivir, ya mineralizado,\\
inmóvil esperando la agonía,\\
mientras florece el territorio azul\\
predestinado de la primavera.  

Mi verdad o mi fábula revelan\\
que es más tenaz que el hombre\\
el ejercicio de la cobardía.  

\end{verse}

\clearpage
\poemtitle {Sonata con dolores (1971)}
\begin{verse}

Cada vez resurrecto\\
entrando en agonía y alegría,\\
muriendo de una vez\\
y no muriendo,\\
así es, es así y es otra vez así.  

El golpe que te dieron\\
lo repartiste alrededor de tu alma,\\
lo dejaste caer de ropa en ropa\\
manchando los vestuarios\\
con huellas digitales\\
de los dolores que te destinaron\\
y que a ti solo te pertenecían.  

Ay, mientras tú caías\\
en la grieta terrible,\\
la boca que buscabas\\
para vivir y compartir tus besos\\
allí cayó contigo, con tu sombra\\
en la abertura destinada a ti.  

Porque, por qué, por qué te destinaste\\
corona y compañía en el suplicio,\\
por qué se atribuyó la flor azul\\
la participación de tu quebranto?  

Y un día de dolores como espadas\\
se repartió desde tu propia herida?\\
Sí, sobrevives. Sí, sobrevivimos\\
en lo imborrable, haciendo\\
de muchas vidas una cicatriz,\\
de tanta hoguera una ceniza amarga,\\
y de tantas campanas\\
un latido, un sonido bajo el mar.  
\end{verse}
\end{document}
