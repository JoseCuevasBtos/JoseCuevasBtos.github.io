% rubber: module xelatex
\documentclass[12pt]{article}

\usepackage[spanish]{babel}
\usepackage[utf8]{inputenc}
% \usepackage{palatino}
\usepackage{verse, gmverse}

\usepackage{fontspec}
\setmainfont{EB Garamond}

\date{}
\title{Poesía selecta de Pablo Neruda V}
\begin{document}
\maketitle
\tableofcontents

\clearpage
\poemtitle{ Testamento de otoño (1958)}
\begin{verse}
\vspace{\baselineskip}
{\scshape\bfseries El poeta entra a contar su condición y predilecciones}
Entre morir y no morir  
me decidí por la guitarra  
y en esta intensa profesión  
mi corazón no tiene tregua,  
porque donde menos me esperan  
yo llegaré con mi equipaje  
a cosechar el primer vino  
en los sombreros del otoño.  

Entraré si cierran la puerta  
y si me reciben me voy,  
no soy de aquellos navegantes  
que se extravían en el hielo:  
yo me acomodo como el viento,  
con las hojas más amarillas,  
con los capítulos caídos  
de los ojos de las estatuas  
y si en alguna parte descanso  
es en la propia nuez del fuego,  
en lo que palpita y crepita  
y luego viaja sin destino.  

A lo largo de los renglones  
habrás encontrado tu nombre,  
lo siento muchísimo poco,  
no se trataba de otra cosa  
sino de muchísimas más,  
porque eres y porque no eres  
y esto le pasa a todo el mundo,  
nadie se da cuenta de todo  
y cuando se suman las cifras  
todos éramos falsos ricos:  
ahora somos nuevos pobres.  

\vspace{\baselineskip}
{\scshape\bfseries Habla de sus enemigos y les participa su herencia}
He sido cortado en pedazos  
por rencorosas alimañas  
que parecían invencibles.  
Yo me acostumbré en el mar  
a comer pepinos de sombra,  
extrañas variedades de ámbar  
y a entrar en ciudades perdidas  
con camiseta y armadura  
de tal manera que te matan  
y tú te mueres de la risa.  

Dejo pues a los que ladraron  
mis pestañas de caminante,  
mi predilección por la sal,  
la dirección de mi sonrisa  
para que todo lo lleven  
con discreción, si son capaces:  
ya que no pudieron matarme  
no puedo impedirles después  
que no se vistan con mi ropa,  
que no aparezcan los domingos  
con trocitos de mi cadáver,  
certeramente disfrazados.  
Si no dejé tranquilo a nadie  
no me van a dejar tranquilo,  
publicarán mis calcetines.  

\vspace{\baselineskip}
{\scshape\bfseries Se dirige a otros sectores}
Dejé mis bienes terrenales  
a mi Partido y a mi pueblo,  
ahora se trata de otras cosas,  
cosas tan oscuras y claras  
que son sin embargo una sola.  
Así sucede con las uvas,  
y sus dos poderosos hijos,  
el vino blanco, el vino rojo,  
toda la vida es roja y blanca,  
toda claridad es oscura,  
y no todo es tierra y adobe,  
hay en mi herencia sombra y sueños.  

\vspace{\baselineskip}
{\scshape\bfseries Contesta a algunos bien intencionados}
Me preguntaron una vez  
por qué escribía tan oscuro,  
pueden preguntarlo a la noche,  
al mineral, a las raíces.  
Yo no supe qué contestar  
hasta que luego y después  
me agredieron dos desalmados  
acusándome de sencillo:  
que responda el agua que corre,  
y me fui corriendo y cantando.  

\vspace{\baselineskip}
{\scshape\bfseries Destina sus penas}
A quién dejo tanta alegría  
que pululó por mis venas  
y este ser y no ser fecundo  
que me dio la naturaleza?  
He sido un largo río lleno  
de piedras duras que sonaban  
con sonidos claros de noche,  
y a quién puedo dejarle tanto,  
tanto que dejar y tan poco,  
una alegría sin objeto,  
un caballo solo en el mar,  
un telar que tejía viento?  

\vspace{\baselineskip}
{\scshape\bfseries Dispone de sus regocijos}
Mis tristezas se las destino  
a los que me hicieron sufrir,  
pero me olvidé cuáles fueron,  
y no sé dónde las dejé,  
si las ven en medio del bosque  
son como las enredaderas:  
suben del suelo con sus hojas  
y terminan donde terminas,  
en tu cabeza o en el aire,  
y para que no suban más  
hay que cambiar de primavera.  

\vspace{\baselineskip}
{\scshape\bfseries Se pronuncia en contra del odio}
Anduve acercándome al odio,  
son serios sus escalofríos,  
sus nociones vertiginosas.  
El odio es un pez espada,  
se mueve en el agua invisible  
y entonces se le ve venir,  
y tiene sangre en el cuchillo:  
lo desarma la transparencia.  

Entonces para qué odiar  
a los que tanto nos odiaron?  
Allí están debajo del agua  
acechadores y acostados  
preparando espada y alcuza,  
telarañas y telaperros.  
No se trata de cristianismos,  
no es oración ni sastrería,  
sino que el odio perdió:  
se le cayeron las escamas  
en el mercado del veneno,  
y mientras tanto sale el sol  
y uno se pone a trabajar  
y a comprar su pan y su vino.  

\vspace{\baselineskip}
{\scshape\bfseries Pero lo considera en su testamento}
Al odio le dejaré  
mis herraduras de caballo,  
mi camiseta de navío,  
mis zapatos de caminante,  
mi corazón de carpintero,  
todo lo que supe hacer  
y lo que me ayudó a sufrir,  
lo que tuve de duro y puro,  
de indisoluble y emigrante,  
para que se aprenda en el mundo  
que los que tienen bosque y agua  
pueden cortar y navegar,  
pueden ir y pueden volver,  
pueden padecer y amar,  
pueden temer y trabajar,  
pueden ser y pueden seguir,  
pueden florecer y morir,  
pueden ser sencillos y oscuros,  
pueden no tener orejas,  
pueden aguantar la desdicha,  
pueden esperar una flor,  
en fin, podemos existir,  
aunque no acepten nuestras vidas  
unos cuantos hijos de puta.  

\vspace{\baselineskip}
{\scshape\bfseries Finalmente se dirige con arrobamiento a su amada}
Matilde Urrutia, aquí te dejo  
lo que tuve y lo que no tuve,  
lo que soy y lo que no soy.  
Mi amor es un niño que llora,  
no quiere salir de tus brazos,  
yo te lo dejo para siempre:  
eres para mí la más bella.  

Eres para fin la más bella,  
la más tatuada por el viento,  
como un arbolito del sur,  
como un avellano en agosto,  
eres para fin suculenta  
como una panadería,  
es de tierra tu corazón  
pero tus manos son celestes.  

Eres roja y eres picante,  
eres blanca y eres salada  
como escabeche de cebolla,  
eres un piano que ríe  
con todas las notas del alma  
y sobre mí cae la música  
de tus pestañas y tu pelo,  
me baño en tu sombra de oro  
y me deleitan tus orejas  
como si las hubiera visto  
en las mareas de coral:  
por tus uñas luché en las olas  
contra pescados pavorosos.  

De Sur a Sur se abren tus ojos,  
y de Este a Oeste tu sonrisa,  
no se te pueden ver los pies,  
y el sol se entretiene estrellando  
el amanecer en tu pelo.  
Tu cuerpo y tu rostro llegaron  
como yo, de regiones duras,  
de ceremonias lluviosas,  
de antiguas tierras y martirios,  
sigue cantando el Bío-Bío  
en nuestra arcilla ensangrentada,  
pero tú trajiste del bosque,  
todos los secretos perfumes  
y esa manera de lucir  
un perfil de flecha perdida,  
una medalla de guerrero.  
Tú fuiste mi vencedora  
por el amor y por la tierra.  
porque tu boca me traía  
antepasados manantiales,  
citas en bosque de otra edad,  
oscuros tambores mojados:  
de pronto oí que me llamaban:  
era de lejos y de cuando:  
me acerqué al antiguo follaje  
y besé mi sangre en tu boca,  
corazón mío, mi araucana.  

Qué puedo dejarte si tienes,  
Matilde Urrutia, en tu contacto  
ese aroma de hojas quemadas,  
esa fragancia de frutillas  
y entre tus dos pechos marinos  
el crepúsculo de Cauquenes  
y el olor de peumo de Chile?  

Es el alto otoño del mar  
lleno de niebla y cavidades,  
la tierra se extiende y respira,  
se le caen al mes las hojas.  
Y tú inclinada en mi trabajo  
con tu pasión y tu paciencia  
deletreando las patas verdes,  
las telarañas, los insectos  
de mi mortal caligrafía,  
oh leona de pies pequeñitos,  
qué haría sin tus manos breves?  
dónde andaría caminando  
sin corazón y sin objeto?  
en qué lejanos autobuses,  
enfermo de fuego o de nieve?  

Te debo el otoño marino  
con la humedad de las raíces,  
y la niebla como una uva,  
y el sol silvestre y elegante:  
te debo este cajón callado  
en que se pierden los dolores  
y sólo suben a la frente  
las corolas de la alegría.  
Todo te lo debo a ti,  
tórtola desencadenada,  
mi codorniza copetona,  
mi jilguero de las montañas,  
mi campesina de Coihueco.  

Alguna vez si ya no somos,  
si ya no vamos ni venimos  
bajo siete capas de polvo  
los pies secos de la muerte,  
estaremos juntos, amor,  
extrañamente confundidos.  
Nuestras espinas diferentes,  
nuestros ojos maleducados,  
nuestros pies que no se encontraban  
la unidad en un cementerio?  
Que no nos separe la vida  
y se vaya al diablo la muerte!  

Recomendaciones finales  
Aquí me despido, señores,  
después de tantas despedidas  
y como no les dejo nada  
quiero que todos toquen algo:  
lo más inclemente que tuve,  
lo más insano y más ferviente  
vuelve a la tierra y vuelve a ser:  
los pétalos de la bondad  
cayeron como campanadas  
en la boca verde del viento.  

Pero yo recogí con creces  
la bondad de amigos y ajenos.  
Me recibía la bondad  
por donde pasé caminando  
y la encontré por todas partes  
como un corazón repartido.  

Qué fronteras medicinales  
no destronaron mi destierro  
compartiendo conmigo el pan,  
el peligro, el techo y el vino?  
El mundo abrió sus arboledas  
y entré como Juan por su casa  
entre dos filas de ternura.  
Tengo en el Sur tantos amigos  
como los que tengo en el Norte,  
no se puede poner el sol  
entre mis amigos del Este,  
y cuantos son en el Oeste?  
No puedo numerar el trigo.  
No puedo nombrar ni contar  
los Oyarzunes fraternales:  
en América sacudida  
por tanta amenaza nocturna  
no hay luna que no me conozca,  
ni caminos que no me esperen,  
en los pobres pueblos de arcilla  
o en las ciudades de cemento  
hay algún Arce remoto  
que no conozco todavía  
pero que nacimos hermanos.  

En todas partes recogí  
la miel que devoran los osos,  
la sumergida primavera,  
el tesoro del elefante,  
y eso se lo debo a los míos,  
a mis parientes cristalinos.  
El pueblo me identificó  
y nunca dejé de ser pueblo.  
Tuve en la palma de la mano  
el mundo con sus archipiélagos  
y como soy irrenunciable  
no renuncié a mi corazón,  
a las ostras ni a las estrellas.  

\vspace{\baselineskip}
{\scshape\bfseries Termina su libro el poeta hablando de sus variadas transformaciones y confirmando su fe en la poesía}
De tantas veces que he nacido  
tengo una experiencia salobre  
como criaturas del mar  
con celestiales atavismos  
y con destinación terrestre.  
Y así me muevo sin saber  
a qué mundo voy a volver  
o si voy a seguir viviendo.  
Mientras se resuelven las cosas  
aquí dejé mi testimonio,  
mi navegante estravagario  
para que leyéndolo mucho  
nadie pudiera aprender nada,  
sino el movimiento perpetuo  
de un hombre claro y confundido,  
de un hombre lluvioso y alegre,  
enérgico y otoñabundo.  
Y ahora detrás de esta hoja  
me voy y no desaparezco:  
daré un salto en la transparencia  
como un nadador del cielo,  
y luego volveré a crecer  
hasta ser tan pequeño un día  
que el viento me llevará  
y no sabré cómo me llamo  
y no seré cuando despierte:  
entonces cantaré en silencio.  

\end{verse}

\clearpage
\poemtitle{ Soneto I (1959)}
\begin{verse}
Matilde, nombre de planta o piedra o vino,  
de lo que nace de la tierra y dura,  
palabra en cuyo crecimiento amanece,  
en cuyo estío estalla la luz de los limones.  

En ese nombre corren navíos de madera  
rodeados por enjambres de fuego azul marino,  
y esas letras son el agua de un río  
que desemboca en mi corazón calcinado.  

Oh nombre descubierto bajo una enredadera  
como la puerta de un túnel desconocido  
que comunica con la fragancia del mundo!  

Oh invádeme con tu boca abrasadora,  
indágame, si quieres, con tus ojos nocturnos,  
pero en tu nombre déjame navegar y dormir.  

\end{verse}

\clearpage
\poemtitle{ Soneto XLIV (1959)}
\begin{verse}
Sabrás que no te amo y que te amo  
puesto que de dos modos es la vida,  
la palabra es un ala del silencio,  
el fuego tiene una mitad de frío.  

Yo te amo para comenzar a amarte,  
para recomenzar el infinito  
y para no dejar de amarte nunca:  
por eso no te amo todavía.  

Te amo y no te amo como si tuviera  
en mis manos las llaves de la dicha  
y un incierto destino desdichado.  

Mi amor tiene dos vidas para amarte.  
Por eso te amo cuando no te amo  
y por eso te amo cuando te amo.  

\end{verse}

\clearpage
\poemtitle{ Soneto XLVI (1959)}
\begin{verse}
De las estrellas que admiré, mojadas  
por ríos y rocíos diferentes,  
yo no escogí sino la que yo amaba  
y desde entonces duermo con la noche.  

De la ola, una ola y otra ola,  
verde mar, verde frío, rama verde,  
yo no escogí sino una sola ola:  
la ola indivisible de tu cuerpo.  

Todas las gotas, todas las raíces,  
todos los hilos de la luz vinieron,  
me vinieron a ver tarde o temprano.  

Yo quise para mí tu cabellera.  
Y de todos los dones de mi patria  
solo escogí tu corazón salvaje.  

\end{verse}

\clearpage
\poemtitle{ Soneto LXI (1959)}
\begin{verse}
Trajo el amor su cola de dolores,  
su largo rayo estático de espinas  
y cerramos los ojos porque nada,  
porque ninguna herida nos separe.  

No es culpa de tus ojos este llanto:  
tus manos no clavaron esta espada:  
no buscaron tus pies este camino:  
llegó a tu corazón la miel sombría.  

Cuando el amor como una inmensa ola  
nos estrelló contra la piedra dura,  
nos amasó con una sola harina,  

cayó el dolor sobre otro dulce rostro  
y así en la luz de la estación abierta  
se consagró la primavera herida.  

\end{verse}

\clearpage
\poemtitle{ Soneto LXVI (1959)}
\begin{verse}
No te quiero sino porque te quiero  
y de quererte a no quererte llego  
y de esperarte cuando no te espero  
pasa mi corazón del frío al fuego.  

Te quiero solo porque a ti te quiero,  
te odio sin fin, y odiándote te ruego,  
y la medida de mi amor viajero  
es no verte y amarte como un ciego.  

Tal vez consumirá la luz de enero,  
su rayo cruel, mi corazón entero,  
robándome la llave del sosiego.  

En esta historia solo yo me muero  
y moriré de amor porque te quiero,  
porque te quiero, amor, a sangre y fuego.  

\end{verse}

\clearpage
\poemtitle{ Escrito en el año 2000 (1960)}
\begin{verse}
Quiero hablar con las últimas estrellas  
ahora, elevado en este monte humano,  
solo estoy con la noche compañera  
y un corazón gastado por los años.  
Llegué de lejos a estas soledades,  
tengo derecho al sueño soberano,  
a descansar con los ojos abiertos  
entre los ojos de los fatigados,  
y mientras duerme el hombre con su tribu,  
cuando todos los ojos se cerraron,  
los pueblos sumergidos de la noche,  
el cielo de rosales estrellados,  
dejo que el tiempo corra por mi cara  
como aire oscuro o corazón mojado  
y veo lo que viene y lo que nace,  
los dolores que fueron derrotados,  
las pobres esperanzas de mi pueblo:  
los niños en la escuela con zapatos,  
el pan y la justicia repartiéndose  
como el sol se reparte en el verano.  
Veo la sencillez desarrollada,  
la pureza del hombre con su arado  
y entre la agricultura voy y vuelvo  
sin encontrar inmensos hacendados.  
Es tan fácil la luz y no se hallaba:  
el amor parecía tan lejano:  
estuvo siempre cerca la razón:  
nosotros éramos los extraviados  
y ya creíamos en un mundo triste  
lleno de emperadores y soldados  
cuando se vio de pronto que se fueron  
para siempre los crueles y los malos  
y todo el mundo se quedó tranquilo  
en su casa, en la calle, trabajando.  
Y ahora ya se sabe que no es bueno  
que esté la tierra en unas pocas manos,  
que no hay necesidad de andar corriendo  
entre gobernadores y juzgados.  
Qué sencilla es la paz y qué difícil  
embestirse con piedras y con palos  
todos los días y todas las noches  
como si ya no fuéramos cristianos.  

Alta es la noche y pura como piedra  
y con su frío toca mi costado  
como diciéndome que duerma pronto,  
que ya están mis trabajos terminados.  
Pero tengo que hablar con las estrellas,  
hablar en un idioma oscuro y claro  
y con la noche misma conversar  
con sencillez como hermana y hermano.  
Me envuelve con fragancia poderosa  
y me toca la noche con sus manos:  
me doy cuenta que soy aquel nocturno  
que dejé atrás en el tiempo lejano  
cuando la primavera estudiantil  
palpitaba en mi traje provinciano.  
Todo el amor de aquel tiempo perdido,  
el dolor de un aroma arrebatado,  
el color de una calle con cenizas,  
el cielo inextinguible de unas manos!  
Y luego aquellos climas devorantes  
donde mi corazón fue devorado,  
los navíos que huían sin destino,  
los países oscuros o delgados,  
aquella fiebre que tuve en Birmania  
y aquel amor que fue crucificado.  

Soy solo un hombre y llevo mis castigos  
como cualquier mortal apesarado  
de amar, amar, amar sin que lo amaran  
y de no amar habiendo sido amado.  
Y surgen las cenizas de una noche,  
cerca del mar, en un río sagrado,  
y un cadáver oscuro de mujer  
ardiendo en un brasero abandonado:  
el Irrawadhy desde la espesura  
mueve sus aguas y su luz de escualo.  
Los pescadores de Ceylán que alzaban  
conmigo todo el mar y sus pescados  
y las redes chorreando milagrosos  
peces de terciopelo colorado  
mientras los elefantes esperaban  
a que les diera un fruto con mis manos.  
Ay cuánto tiempo es el que en mis mejillas  
se acumuló como un reloj opaco  
que acarrea en su frágil movimiento  
un hilo interminablemente largo  
que comienza con un niño que llora  
y acaba en un viajero con un saco!  

Después llegó la guerra y sus dolores  
y me tocan los ojos y me buscan  
en la noche los muertos españoles,  
los busco y no me ven y sin embargo  
veo sus apagados resplandores:  
Don Antonio morir sin esperanza,  
Miguel Hernández muerto en sus prisiones  
y el pobre Federico asesinado  
por los medioevales malhechores,  
por la caterva infiel de los Paneros:  
los asesinos de los ruiseñores.  
Ay tanta y tanta sombra y tanta sangre  
me llaman esta noche por mi nombre:  
ahora me tocan con alas heladas  
y me señalan su martirio enorme:  
nadie los ha vengado, y me lo piden.  
Y solo mi ternura los conoce.  

Ay cuánta noche cabe en una noche  
sin desbordar esta celeste copa,  
suena el silencio de las lejanías  
como una inaccesible caracola  
y caen en mis manos las estrellas  
llenas aún de música y de sombra.  
En este espacio el tumultuoso peso  
de mi vida no vence ni solloza  
y despido al dolor que me visita  
como si despidiera a una paloma:  
si hay cuentas que sacar hay que sacarlas  
con lo que va a venir y que se asoma,  
con la felicidad de todo el mundo  
y no con lo que el tiempo desmorona.  
Y aquí en el cielo de Sierra Maestra  
yo solo alcanzo a saludar la aurora  
porque se me hizo tarde en mis quehaceres,  
se me pasó la vida en tantas cosas,  
que dejo mis trabajos a otras manos  
y mi canción la cantará otra boca.  
Porque así se encadena la jornada  
y floreciendo seguirá la rosa.  

No se detiene el hombre en su camino:  
otro toma las armas misteriosas:  
no tiene fin la primavera humana,  
del invierno salió la mariposa  
y era mucho más frágil que una flor,  
por eso su belleza no reposa  
y se mueven sus alas de color  
con una matemática radiosa.  
Y un hombre construyó solo una puerta  
y no sacó del mar sino una gota  
hasta que de una vida hasta otra vida  
levantaremos la ciudad dichosa  
con los brazos de los que ya no viven  
y con manos que no han nacido ahora.  
Es esa la unidad que alcanzaremos:  
la luz organizada por la sombra,  
por la continuidad de los deseos  
y el tiempo que camina por las horas  
hasta que ya todos estén contentos.  

Y así comienza una vez más la Historia.  

Y así, pues, en lo alto de estos montes,  
lejos de Chile y de sus cordilleras  
recibo mi pasado en una copa  
y la levanto por la tierra entera,  
y aunque mi patria circule en mi sangre  
sin que nunca se apague su carrera  
en esta hora mi razón nocturna  
señala en Cuba la común bandera  
del hemisferio oscuro que esperaba  
por fin una victoria verdadera.  
La dejo en esta cumbre custodiada,  
alta, ondeando sobre las praderas,  
indicando a los pueblos agobiados  
la dignidad nacida en la pelea:  
Cuba es un mástil claro que divisan  
a través del espacio y las tinieblas,  
es como un árbol que nació en el centro  
del mar Caribe y sus antiguas penas:  
su follaje se ve de todas partes  
y sus semillas van bajo la tierra,  
elevando en la América sombría  
el edificio de la primavera.  

\end{verse}

\clearpage
\poemtitle{ Primer viaje (1962)}
\begin{verse}
No sé cuándo llegamos a Temuco.  
Fue impreciso nacer y fue tardío  
nacer de veras, lento,  
y palpar, conocer, odiar, amar,  
todo esto tiene flor y tiene espinas.  
Del pecho polvoriento de mi patria  
me llevaron sin habla  
hasta la lluvia de la Araucanía.  
Las tablas de la casa  
olían a bosque,  
a selva pura.  
Desde entonces mi amor  
fue maderero  
y lo que toco se convierte en bosque.  
Se me confunden  
los ojos y las hojas,  
ciertas mujeres con la primavera  
del avellano, el hombre con el árbol,  
amo el mundo del viento y del follaje,  
no distingo entre labios y raíces.  

Del hacha y de la lluvia fue creciendo  
la ciudad maderera  
recién cortada como  
nueva estrella con gotas de resina,  
y el serrucho y la sierra  
se amaban noche y día  
cantando,  
trabajando,  
y ese sonido agudo de cigarra  
levantando un lamento  
en la obstinada soledad, regresa  
al propio canto mío:  
mi corazón sigue cortando el bosque,  
cantando con las sierras en la lluvia,  
moliendo frío y aserrín y aroma.  

\end{verse}

\clearpage
\poemtitle{ Rangoon, 1927 (1962)}
\begin{verse}
En Rangoon era tarde para mí.  
Todo lo habían hecho:  
una ciudad  
de sangre,  
sueño y oro.  
El río que bajaba  
de la selva salvaje  
a la ciudad caliente,  
a las calles leprosas  
en donde un hotel blanco para blancos  
y una pagoda de oro para gente dorada  
era cuanto  
pasaba  
y no pasaba.  
Rangoon, gradas heridas  
por los escupitajos  
del betel,  
las doncellas birmanas  
apretando al desnudo  
la seda  
como si el fuego acompañase  
con lenguas de amaranto  
la danza, la suprema  
danza:  
el baile de los pies hacia el Mercado,  
el ballet de las piernas por las calles.  
Suprema luz que abrió sobre mi pelo  
un globo cenital, entró en mis ojos  
y recorrió en mis venas  
los últimos rincones de mi cuerpo  
hasta otorgarse la soberanía  
de un amor desmedido y desterrado.  

Fue así, la encontré cerca  
de los buques de hierro  
junto a las aguas sucias  
de Martabán: miraba  
buscando hombre:  
ella también tenía  
color duro de hierro,  
su pelo era de hierro,  
y el sol pegaba en ella como en una herradura.  

Era mi amor que yo no conocía.  

Yo me senté a su lado  
sin mirarla  
porque yo estaba solo  
y no buscaba río ni crepúsculo,  
no buscaba abanicos,  
ni dinero ni luna,  
sino mujer, quería  
mujer para mis manos y mi pecho,  
mujer para mi amor, para mi lecho,  
mujer plateada, negra, puta o pura,  
carnívora celeste, anaranjada,  
no tenía importancia,  
la quería para amarla y no amarla,  
la quería para plato y cuchara,  
la quería de cerca, tan de cerca  
que pudiera morderle los dientes con mis besos,  
la quería fragante a mujer sola,  
la deseaba con olvido ardiente.  

Ella tal vez quería  
o no quería lo que yo quería,  
pero allí en Martabán, junto al agua de hierro,  
cuando llegó la noche, que allí sale del río,  
como una red repleta de pescados inmensos,  
yo y ella caminamos juntos a sumergirnos  
en el placer amargo de los desesperados.  

\end{verse}

\clearpage
\poemtitle{ El mar (1963)}
\begin{verse}
Necesito del mar porque me enseña:  
no sé si aprendo música o conciencia:  
no sé si es ola sola o ser profundo  
o solo ronca voz o deslumbrante  
suposición de peces y navíos.  
El hecho es que hasta cuando estoy dormido  
de algún modo magnético circulo  
en la universidad del oleaje.  

No son solo las conchas trituradas  
como si algún planeta tembloroso  
participara paulatina muerte,  
no, del fragmento reconstruyo el día,  
de una racha de sal la estalactita  
y de una cucharada el dios inmenso.  

Lo que antes me enseñó lo guardo! Es aire,  
incesante viento, agua y arena.  

Parece poco para el hombre joven  
que aquí llegó a vivir con sus incendios,  
y si embargo el pulso que subía  
y bajaba a su abismo,  
el frío del azul que crepitaba,  
el desmoronamiento de la estrella,  
el tierno desplegarse de la ola  
despilfarrando nieve con la espuma,  
el poder quieto, allí, determinado  
como un trono de piedra en lo profundo,  
substituyó el recinto en que crecían  
tristeza terca, amontonado olvido,  
y cambió bruscamente mi existencia:  
di mi adhesión al puro movimiento.  

\end{verse}

\clearpage
\poemtitle{ Cita de invierno (1963)}
\begin{verse}
\vspace{\baselineskip}
{\scshape\bfseries I}
He esperado este invierno como ningún invierno  
se esperó por un hombre antes de mí,  
todos tenían citas con la dicha:  
solo yo te esperaba, oscura hora.  
Es este como los de antaño, con padre y madre, con fuego  
de carbón y el relincho de un caballo en la calle?  
Es este invierno como el del año futuro,  
el de la inexistencia, con el frío total  
y la naturaleza no sabe que nos fuimos?  
No. Reclamé la soledad circundada  
por un gran cinturón de pura lluvia  
y aquí en mi propio océano me encontró con el viento  
volando como un pájaro entre dos zonas de agua.  
Todo estaba dispuesto para que llore el cielo.  
El fecundo cielo de un solo suave párpado  
dejó caer sus lágrimas como espadas glaciales  
y se cerró como una habitación de hotel  
el mundo: cielo, lluvia y espacio.  

\vspace{\baselineskip}
{\scshape\bfseries II}
Oh centro, oh copa sin latitud ni término!  
Oh corazón celeste del agua derramada!  
Entre el aire y la arena baila y vive  
un cuerpo destinado  
a buscar su alimento transparente  
mientras yo llego y entro con sombrero,  
con cenicientas botas  
gastadas por la sed de los caminos.  
Nadie había llegado  
para la solitaria ceremonia.  
Me siento apenas solo  
ahora que la pureza es perceptible.  
Sé que no tengo fondo, como el pozo  
que nos llenó de espanto cuando niños,  
y que rodeado por la transparencia  
y la palpitación de las agujas  
hablo con el invierno,  
con la dominación y el poderío  
de su vago elemento,  
con la extensión y la salpicadura  
de su rosa tardía  
hasta que pronto no había luz  
y bajo el techo  
de la casa oscura  
yo seguiré sin que nadie responda  
hablando con la tierra.  

\vspace{\baselineskip}
{\scshape\bfseries III}
Quién no desea un alma dura?  
Quién no se practicó en el alma un filo?  
Cuando a poco de ver vimos el odio  
y de empezar a andar nos tropezaron  
y de querer amar nos desamaron  
y solo de tocar fuimos heridos,  
quién no hizo algo por armar sus manos  
y para subsistir hacerse duro  
como el cuchillo, y devolver la herida?  
El delicado pretendió aspereza,  
el más tierno buscaba empuñadura,  
el que solo quería que lo amaran  
con un tal vez, con la mitad de un beso,  
pasó arrogante sin mirar a aquella  
que lo esperaba abierta y desdichada:  
no hubo nada que hacer: de calle en calle  
se establecieron mercados de máscaras  
y el mercader probaba a cada uno  
un rostro de crepúsculo o de tigre,  
de austero, de virtud, de antepasado,  
hasta que terminó la luna llena  
y en la noche sin luz fuimos iguales.  

\vspace{\baselineskip}
{\scshape\bfseries IV}
Yo tuve un rostro que perdí en la arena,  
un pálido papel de pesaroso  
y me costó cambiar la piel del alma  
hasta llegar a ser el verdadero,  
a conquistar este derecho triste:  
esperar el invierno sin testigos.  
Esperar una ola bajo el vuelo  
del oxidado cormorán marino  
en plena soledad restituida.  
Esperar y encontrarme con un síntoma  
de luz o luto  
o nada:  
lo que percibe apenas mi razón,  
mi sinrazón, mi corazón, mis dudas.  

\vspace{\baselineskip}
{\scshape\bfseries V}
Ahora ya tiene el agua tanto tiempo  
que es nueva, el agua antigua se fugó  
a romper su cristal en otra vida  
y la arena tampoco recogió  
el tiempo, es otro el mar y su camisa,  
la identidad perdió el espejo  
y crecimos cambiando de camino.  

\vspace{\baselineskip}
{\scshape\bfseries VI}
Invierno, no me busques. He partido.  
Estoy después, en lo que llega ahora  
y desarrollará la lluvia fina,  
las agujas sin fin, el matrimonio  
del alma con los árboles mojados,  
la ceniza del mar, el estallido  
de una cápsula de oro en el follaje,  
y mis ojos tardíos  
solo preocupados por la tierra.  

\vspace{\baselineskip}
{\scshape\bfseries VII}
Solo por tierra, viento, agua y arena  
que me otorgaron claridad plenaria.  

\end{verse}

\clearpage
\poemtitle{ Para la envidia (1964)}
\begin{verse}
De uno a uno saqué los envidiosos  
de mi propia camisa, de mi piel,  
los vi junto a mí mismo cada día,  
los contemplé  
en el reino transparente  
de una gota de agua:  
los amé cuanto pude: en su desdicha  
o en la ecuanimidad de sus trabajos:  
y hasta ahora no sé  
cómo ni cuándo  
substituyeron nardo o limonero  
por silenciosa arruga  
y una grieta anidó donde se abriera  
la estrella regular de la sonrisa.  

Aquella grieta de un hombre en la boca!  

Aquella miel que fue substituida!  

El grave viento de la edad  
volando  
trajo polvo, alimentos,  
semillas separadas del amor,  
pétalos enrollados de serpiente,  
ceniza cruel del odio muerto  
y todo  
fructificó en la herida de la boca,  
funcionó la pasión generatriz  
y el triste sedimento del olvido  
germinó, levantando la corola,  
la medusa violeta de la envidia.  
Qué haces tú, Pedro, cuando sacas peces?  
Los devuelves al mar, rompes la red,  
cierras los ojos ante el incentivo  
de la profundidad procreadora?  

Ay! Yo confieso mi pecado puro!  
Cuanto saqué del mar,  
coral, escama,  
cola del arcoíris,  
pez o palabra o planta plateada  
o simplemente piedra submarina,  
yo la erigí, le di la luz de mi alma.  

Yo, pescador, recogí lo perdido  
y no hice daño a nadie en mis trabajos.  

No hice daño, o tal vez herí de muerte  
al que quiso nacer y recibió  
el canto de mi desembocadura  
que silenció su condición bravía:  
al que no quiso  
navegar en mi pecho,  
y desató  
su propia fuerza,  
pero vino el viento  
y se llevó su voz y no nacieron  
aquellos que querían ver la luz.  

Tal vez el hombre crece y no respeta,  
como el árbol del bosque, el albedrío  
de lo que lo rodea,  
y es de pronto  
no solo la raíz, sino la noche,  
y no solo da frutos, sino sombra,  
sombra y noche que el tiempo y el follaje  
abandonaron en el crecimiento  
hasta que desde la humedad yacente  
en donde esperan las germinaciones  
no se divisan dedos de la luz:  
el gratuito sol le fue negado  
a la semilla hambrienta  
y a plena oscuridad desencadena  
el alma un desarrollo atormentado.  

Tal vez no sé, no supe, no sabía.  

No tuve tiempo en mis preocupaciones  
de ver, de oír, de acechar y palpar  
lo que estaba pasando, y por amor  
pensé que mi deber era cantar,  
cantar creciendo y olvidando siempre,  
agonizando como resistiendo:  
era mi amor, mi oficio  
en la mañana entre los carpinteros,  
bebiendo con los húsares, de noche,  
desatar la escritura de mi canto  
y yo creí cumplir,  
ardiente o separado  
del fuego,  
cerca del manantial o en la ceniza,  
creí que dando cuanto yo tenía,  
hiriéndome para no dormir,  
a todo sueño, a toda hora, a toda vida,  
con mi sangre y con mis meditaciones,  
y con lo que aprendí de cada cosa,  
del clavel, de su generosidad,  
de la madera y su paz olorosa,  
del propio amor, del río, de la muerte,  
con lo que me otorgó la ciudad y la tierra,  
con lo que yo arranqué de una ola verde,  
o de una casa que dejó vacía  
la guerra, o de una lámpara  
que halló encendida en medio del otoño,  
así como del hombre y de sus máquinas,  
del pequeño empleado y su aflicción,  
o del navío navegando en la niebla:  
con todo y, más que todo, con lo que yo debía  
a cada hombre por su propia vida  
hice yo lo posible por pagar, y no tuve  
otra moneda que mi propia sangre.  

Ahora qué hago con este y con el otro?  
Qué puedo hacer para restituir  
lo que yo no robé? Por qué la primavera  
me trajo a mí una corona amarilla  
y quién anduvo hostil y enmarañado  
buscándola en el bosque? Ahora  
tal vez es tarde ya para encontrar  
y volcar en la copa del rencor  
la verdad atrasada y cristalina.  

Tal vez el tiempo endureció la voz,  
la boca, la piedad del ofendido,  
y ya el reloj no podrá volver  
a la consagración de la ternura.  

El odio despiadado tuvo tiempo  
de construir un pabellón furioso  
y destinarme una corona cruel  
con espinas sangrientas y oxidadas.  
Y no fue por orgullo que guardé  
el corazón ausente del terror:  
ni de mi dolor ensimismado,  
ni de las alegrías que sostengo  
dispersé  
en la venganza  
el poderío.  

Fue por otra razón, por indefenso.  

Fue porque a cada mordedura  
el día  
que llegaba  
me separaba de un nuevo dolor,  
me amarraba las manos y crecía  
el liquen en la piedra de mi pecho,  
la enredadera se me derramaba,  
pequeñas manos verdes me cubrían,  
y me fui ya sin puños a los bosques  
o me dormí en el título del trébol.  
Oh, yo resguardo en mí mismo la avaricia  
de mis espadas, lento  
en la ira,  
gozo  
en mi dureza,  
pero cuando la tórtola en la torre  
trina, y agacha el brazo el alfarero  
hacia su barro, haciéndolo vasija,  
yo tiemblo y me traspasa  
el aire lancinante:  
mi corazón se va con la paloma.  

Llueve y salgo a probar el aguacero.  

Yo salgo a ser lo que amo, la desnuda  
existencia del sol en el peñasco,  
y lo que crece y crece sin saber  
que no puede abolir su crecimiento:  
dar grano el trigo: ser innumerable  
sin razón: porque así le fue ordenado:  
sin orden, sin mandato,  
y, entre las rosas que no se reparten,  
tal vez esta secreta voluntad,  
esta trepidación de pan y arena,  
llegaron a imponer su condición  
y no soy yo sino materia viva  
que fermenta y levanta sus insignias  
en la fecundación de cada día.  

Tal vez la envidia, cuando  
sacó a brillar contra mí la navaja  
y se hizo profesión de algunos cuantos,  
agregó a mi substancia un alimento  
que yo necesitaba en mis trabajos,  
un ácido agresivo que me dio  
el estímulo brusco de una hora,  
la corrosiva lengua contra el agua.  

Tal vez la envidia, estrella  
hecha de vidrios rotos  
caídos  
en una calle amarga,  
fue una medalla que condecoró  
el pan que doy cantando cada día  
y a mi buen corazón de panadero.  

\end{verse}

\clearpage
\poemtitle{ La chascona (1964)}
\begin{verse}
La piedra y los clavos, la tabla, la teja se unieron: he aquí levantada  
la casa chascona con agua que corre escribiendo en su idioma,  
las zarzas guardaban el sitio con su sanguinario ramaje  
hasta que la escala y sus muros supieron tu nombre  
y la flor encrespada, la vid y su alado zarcillo,  
las hojas de higuera que como estandartes de razas remotas  
cernían sus alas oscuras sobre tu cabeza,  
el muro de azul victorioso, el ónix abstracto del suelo,  
tus ojos, mis ojos, están derramados en roca y madera  
por todos los sitios, los días febriles, la paz que construye  
y sigue ordenada la casa con tu transparencia.  

Mi casa, tu casa, tu sueño en mis ojos, tu sangre siguiendo el camino del cuerpo que duerme  
como una paloma cerrada en sus alas inmóvil persigue su vuelo  
y el tiempo recoge en su copa tu sueño y el mío  
en la casa que apenas nació de las manos despiertas.  

La noche encontrada por fin en la nave que tú construimos,  
la paz de madera olorosa que sigue con pájaros,  
que sigue el susurro del viento perdido en las hojas  
y de las raíces que comen la paz suculenta del humus  
mientras sobreviene sobre mí dormida la luna del agua  
como una paloma del bosque del Sur que dirige el dominio  
del cielo, del aire, del viento sombrío que te pertenece,  
dormida, durmiendo en la casa que hicieron tus manos,  
delgada en el sueño, en el germen del humus nocturno  
y multiplicada en la sombra como el crecimiento del trigo.  

Dorada, la tierra te dio la armadura del trigo,  
el color que los hornos cocieron con barro y delicia,  
la piel que no es blanca ni es negra ni roja ni verde,  
que tiene el color de la arena, del pan, de la lluvia,  
del sol, de la pura madera, del viento,  
tu carne color de campana, color de alimento fragante,  
tu carne que forma la nave y encierra la ola!  

De tantas delgadas estrellas que mi alma recoge en la noche  
recibo el rocío que el día convierte en ceniza  
y bebo la copa de estrellas difuntas llorando las lágrimas  
de todos los hombres, de los prisioneros, de los carceleros,  
y todas las manos me buscan mostrando una llaga,  
mostrando el dolor, el suplicio o la brusca esperanza,  
y así sin que el cielo y la tierra me dejen tranquilo,  
así consumido por otros dolores que cambian de rostro,  
recibo en el sol y en el día la estatua de tu claridad  
y en la sombra, en la luna, en el sueño, el racimo del reino,  
el contacto que induce a mi sangre a cantar en la muerte.  

La miel, bienamada, la ilustre dulzura del viaje completo  
y aún, entre largos caminos, fundamos en Valparaíso una torre,  
por más que en tus pies encontré mis raíces perdidas  
tú y yo mantuvimos abierta la puerta del mar insepulto  
y así destinamos a La Sebastiana el deber de llamar los navíos  
y ver bajo el humo del puerto la rosa incitante,  
el camino cortado en el agua por el hombre y sus mercaderías.  

Pero azul y rosado, roído y amargo entreabierto entre sus telarañas,  
he aquí, sosteniéndose en hilos, en uñas, en enredaderas,  
he aquí victorioso, harapiento, color de campana y de miel,  
he aquí, bermellón y amarillo, purpúreo, plateado, violeta,  
sombrío y alegre, secreto y abierto como una sandía  
el puerto y la puerta de Chile, el manto radiante de Valparaíso,  
el sonoro estupor de la lluvia en los cerros cargados de padecimientos,  
el sol resbalando en la oscura mirada, en los ojos más bellos del mundo.  

Yo te convidé a la alegría de un puerto agarrado a la furia del alto oleaje,  
metido en el frío del último océano, viviendo en peligro,  
hermosa es la nave sombría, la luz vesperal de los meses antárticos,  
la nave de techo amaranto, el puñado de velas o casas o vidas  
que aquí se vistieron con trajes de honor y banderas  
y se sostuvieron cayéndose en el terremoto que abría y cerraba el infierno,  
tomándose al fin de la mano los hombres, los muros, las cosas,  
unidos y desvencijados en el estertor planetario.  

Cada hombre contó con sus manos los bienes funestos, el río  
de sus extensiones, su espada, su rienda, su ganadería,  
y dijo a la esposa: «Defiende tu páramo ardiente o tu campo de nieve»  
o «Cuida la vaca, los viejos telares, la sierra o el oro».  

Muy bien, bienamada, es la ley de los siglos que fueron atándose  
adentro del hombre, en un hilo que ataba también sus cabezas:  
el príncipe echaba las redes con el sacerdote enlutado,  
y mientras los dioses callaban, caían al cofre monedas  
que allí acumularon la ira y la sangre del hombre desnudo.  

Por eso, erigida la base y bendita por cuervos oscuros  
subió el interés y dispuso en el zócalo su pie mercenario,  
después a la Estatua impusieron medallas y música,  
periódicos, radios y televisores cantaron la loa del Santo Dinero,  
y así hasta el probable, hasta el que no pudo ser hombre,  
el manumitido, el desnudo y hambriento, el pastor lacerado,  
el empleado nocturno que roe en tinieblas su pan disputado a las ratas,  
creyeron que aquel era Dios, defendieron el Arca suprema  
y se sepultaron en el humillado individuo, ahítos de orgullo prestado.  

\end{verse}

\clearpage
\poemtitle{ Amor para éste libro (1965)}
\begin{verse}
\itshape
En estas soledades he sido poderoso  
de la misma manera que una herramienta alegre  
o como hierba impune que suelta sus espigas  
o como un perro que se revuelca en el rocío.  
Matilde, el tiempo pasará gastando y encendiendo  
otra piel, otras uñas, otros ojos, y entonces  
el alga que azotaba nuestras piedras bravías,  
la ola que construye, sin cesar, su blancura,  
todo tendrá firmeza sin nosotros,  
todo estará dispuesto para los nuevos días  
que no conocerán nuestro destino.  

Qué dejamos aquí sino el grito perdido  
del queltehve, en la arena del invierno, en la racha  
que nos cortó la cara y nos mantuvo  
erguidos en la luz de la pureza,  
como en el corazón de una estrella preclara?  

Qué dejamos viviendo como un nido  
de ásperas aves, vivas, entre los matorrales  
o estáticas, encima de los fríos peñascos?  
Así pues, si vivir fue solo anticiparse  
a la tierra, a este suelo y su aspereza,  
líbrame tú, amor mío, de no cumplir, y ayúdame  
a volver a mi puesto bajo la tierra hambrienta.  

Pedimos al océano su rosa,  
su estrella abierta, su contacto amargo,  
y al agobiado, al ser hermano, al herido  
dimos la libertad recogida en el viento.  
Es tarde ya. Tal vez  
solo fue un largo día color de miel y azul,  
tal vez solo una noche, como el párpado  
de una grave mirada que abarcó  
la medida del mar que nos rodeaba,  
y en este territorio fundamos solo un beso,  
solo inasible amor que aquí se quedará  
vagando entre la espuma del mar y las raíces.  

\end{verse}

\clearpage
\poemtitle{ Primavera en Chile (1967)}
\begin{verse}
Hermoso es septiembre en mi patria cubierto con una corona de mimbre y violetas  
y con un canasto colgando en los brazos colmado de dones terrestres:  
septiembre adelanta sus ojos mapuches matando el invierno  
y vuelve el chileno a la resurrección de la carne y el vino.  
Amable es el sábado y apenas se abrieron las manos del viernes  
voló transportando ciruelas y caldos de luna y pescado.  

Oh amor en la tierra que tú recorrieras que yo atravesamos  
no tuve en mi boca un fulgor de sandía como en Talagante  
y en vano busqué entre los dedos de la geografía  
el mar clamoroso, el vestido que el viento y la piedra otorgaron a Chile,  
y no hallé duraznos de enero redondos de luz y delicia  
como el terciopelo que guarda y desgrana la miel de mi patria.  
Y en los matorrales del Sur sigiloso conozco el rocío  
por sus penetrantes diamantes de menta, y me embriaga el aroma  
del vino central que estalló desde tu cinturón de racimos  
y el olor de tus aguas pesqueras que te llena de olfato  
porque se abren las valvas del mar en tu pecho de plata abundante,  
y encumbrado arrastrando los pies cuando marcho en los montes más duros  
yo diviso en la nieve invencible la razón de tu soberanía.  

\end{verse}

\clearpage
\poemtitle{ Habla un transeúnte de las américas llamado Chivilcoy (1967)}
\begin{verse}
\vspace{\baselineskip}
{\scshape\bfseries I}
Yo cambio de rumbo, de empleo, de bar y de barco, de pelo  
de tienda y mujer, lancinante, exprofeso no existo,  
tal vez soy mexibiano, argentuayo, bolivio,  
caribián, panamante, colomvenechilenomalteco:  
aprendí en los mercados a vender y comprar caminando:  
me inscribí en los partidos dispares y cambié de camisa impulsado  
por las necesidades rituales que echan a la mierda el escrúpulo  
y confieso saber más que todos sin haber aprendido:  
lo que ignoro no vale la pena, no se paga en la plaza, señores.  

Acostumbro zapatos quebrados, corbatas raídas, cuidado,  
cuando menos lo piensen llevo un gran solitario en un dedo  
y me planchan por dentro y por fuera, me perfuman, me cuidan, me  
peinan.  

Me casé en Nicaragua: pregunten ustedes por el general Allegado  
que tuvo el honor de suegro de su servidor, y más tarde  
en Colombia fui esposo legítimo de una Jaramillo Restrepo.  
Si mis matrimonios terminan cambiando de clima, no importa.  
(Hablando entre hombres: Mi chola de Tambo! Algo serio en la cama).  

\vspace{\baselineskip}
{\scshape\bfseries II}
Vendí mantequilla y chancaca en los puertos peruanos  
y medicamentos de un poblado a otro de la Patagonia:  
voy llegando a viejo en las malas pensiones sin plata, pasando por rico,  
y pasando por pobre entre ricos, sin haber ganado ni perdido nada.  

\vspace{\baselineskip}
{\scshape\bfseries III}
Desde la ventana que me corresponde en la vida  
veo el mismo jardín polvoriento de tierra mezquina  
con perros errantes que orinan y siguen buscando la felicidad,  
o excrementicios y eróticos gatos que no se interesan por vidas ajenas.  

\vspace{\baselineskip}
{\scshape\bfseries IV}
Yo soy aquel hombre rodado por tantos kilómetros y sin existencia:  
soy piedra en un río que no tiene nombre en el mapa:  
soy el pasajero de los autobuses gastados de Oruro  
y aunque pertenezco a las cervecerías de Montevideo  
en la Boca anduve vendiendo guitarras de Chile  
y sin pasaporte entraba y salía por las cordilleras.  
Supongo que todos los hombres dejan equipaje:  
yo voy a dejar como herencia lo mismo que el perro:  
es lo que llevé entre las piernas: mis bienes son esos.  

\vspace{\baselineskip}
{\scshape\bfseries V}
Si desaparezco aparezco con otra mirada: es lo mismo.  
Soy un héroe imperecedero: no tengo comienzo ni fin  
y mi moraleja consiste en un plato de pescado frito.  
\end{verse}
\end{document}
