% rubber: module xelatex
\documentclass[12pt]{article}

\usepackage[spanish]{babel}
\usepackage[utf8]{inputenc}
% \usepackage{palatino}
\usepackage{verse}

\usepackage{fontspec}
\setmainfont{EB Garamond}

\date{}
\title{Sonetos selectos de Shakespeare}
\begin{document}
\maketitle
\tableofcontents
\clearpage
\poemtitle {III}
\begin{verse}

Contémplate al espejo y di a tu rostro\\
que ya se reproduzca sin demora;\\
si no renuevas tu frescura en otro\\
al mundo y a una madre desazonas.\\
Pues ¿qué doncella habrá tan altanera\\
para vedar su huerto a tu simiente?\\
¿Y quién tan vanidoso que prefiera\\
privarnos de belleza con su muerte?\\
Tú eres la viva imagen de tu madre\\
y ella ve en ti el frescor de sus abriles;\\
también tú en tu vejez podrás mirarte\\
y ver la edad de oro que ahora vives.\\
Mas si prefieres que no te recuerden,\\
no engendres y tu imagen con ti muere.

\end{verse}

\clearpage
\poemtitle {VI}
\begin{verse}

No dejes pues que el tosco invierno borre,\\
si no te has destilado, tu verano:\\
endulza una vasija; busca dónde\\
incrementar tu erario y no enterrarlo.\\
Ese uso no es usura mal mirada\\
pues llena de alborozo a quienes paguen\\
y a ti te beneficia de la crianza\\
de uno igual a ti o diez si cabe.\\
Serás diez veces más feliz que ahora\\
al verte reflejado en otros diez;\\
la muerte no podrá con tu persona\\
pues si ellos viven vives tú también.\\
Mas no disfrutes solo tu legado\\
o heredarán tu encanto los gusanos. 

\end{verse}

\clearpage
\poemtitle {VIII}
\begin{verse}

Si tú eres música, ¿te apena oírla?\\
Si el dulce es dulce y es gozoso el gozo,\\
¿por qué amas lo que tomas con inquina\\
y tomas con placer lo ignominioso?\\
Si no te es grato oír el maridaje\\
de notas que armonizan y se suman\\
es porque te regañan con voz suave:\\
no es solo para ti esta partitura.\\
Las cuerdas, como sabes, se disponen\\
por melodiosos pares y al pulsarlas,\\
al mismo tiempo que nos cantan un acorde\\
parecen padre, hijo y madre amada.\\
Y su canción, sin letra y con donaire,\\
te canta: ``Tú, solista, no eres nadie''.

\end{verse}

\clearpage
\poemtitle {XV}
\begin{verse}

Si pienso que la perfección le dura\\
apenas un instante a lo que crece;\\
que en este inmenso teatro perpetúan\\
los astros sus arcanos intereses;\\
si veo que los hombres y las plantas\\
florecen al albur del mismo cielo,\\
de jóvenes se jactan de su savia\\
y menguan al llegar el apogeo;\\
entonces, aunque la inconstante vida,\\
al ver tu juventud te muestre grande,\\
me consta que conspiran tiempo y ruina\\
por inclinar tu día hacia la tarde.\\
En nombre de mi amor, combato el tiempo:\\
lo que te quita él, yo lo reinjerto.

\end{verse}

\clearpage
\poemtitle {XVIII}
\begin{verse}

¿Por qué igualarte a un día de verano\\
si tú eres más hermoso y apacible?\\
El viento azota los capullos mayos\\
y el término estival no tarda en irse;\\
si a veces arde el óculo solar,\\
más veces su dorada faz se nubla\\
y es norma que, por obra natural\\
o del azar, lo bello al fin sucumba.\\
Mas no se nublará tu estío eterno\\
ni perderá la gracia que posee,\\
ni te tendrá la muerte por trofeo\\
si eternas son las líneas donde creces.\\
Habiendo quien respira y pueda ver,\\
todo esto sigue vivo y tú también.

\end{verse}

\clearpage
\poemtitle {XXIII}
\begin{verse}

Así como el actor con pocas tablas\\
el miedo le confunde los papeles,\\
o al ser feroz el cúmulo de rabia\\
le mina el corazón aun siendo fuerte,\\
por miedo de confiarme se me olvidan\\
los ritos amorosos del cortejo\\
y el peso de mi amor es tal que mina\\
la fuerza que mi amor retiene dentro.\\
Prefiero la elocuencia de mis ojos,\\
voceros mudos de mi pecho hablante,\\
que piden por amor y exigen poco,\\
que la de aquella lengua que habla en balde.\\
Lee bien lo que el amor silente escribe,\\
pies la mirada que oye de amor vive.

\end{verse}

\clearpage
\poemtitle {XLIII}
\begin{verse}

Mis ojos ven mejor si están cerrados,\\
así no se distraen con simplezas;\\
mas al dormir, te ven en sueños claros\\
y brillan en lo oscuro como estelas.\\
Y tú, sombra que alumbras a otras sombras,\\
si a ojos que no ven reluces tanto,\\
¿podrá lucir aún más tu dulce forma\\
en plena claridad y a pleno campo?\\
Pues si en la noche inerte tus borrosos\\
contornos engalanan mi pupila,\\
¿podrán embelesarse más mis ojos\\
al verte a la luz viva de los días?\\
El día es noche cuando no te veo\\
y los días son las noches que te sueño.

\end{verse}

\clearpage
\poemtitle {L}
\begin{verse}

Con cuánta pesadumbre emprendo el viaje\\
sabiendo que el descanso del camino\\
traerá consigo el eco de esta frase:\\
"¡Estás a tantas millas de tu amigo!''.\\
Lastrada bajo el peso de mi pena,\\
mi bestia se desplaza a paso lento;\\
parece que su instinto le dijera\\
que nos aleja más si va ligero.\\
La espuela sanguinaria no la azuza\\
y, cuando siente el filo del puyazo,\\
su queja pesarosa me es más dura\\
que el hierro que se clava en su costado,\\
pues hace que mi mente me repita:\\
delante está el pesar, detrás, la dicha.

\end{verse}

\clearpage
\poemtitle {LVI}
\begin{verse}

Reponte, dulce amor, que no se diga,\\
que no eres tan punzante como el hambre,\\
que pese a que se sacia cada día,\\
despunta al día siguiente igual de grande.\\
Por eso, amor, aunque hoy tus ojos queden\\
ahítos de comer hasta el agobio,\\
mañana vuelve a abrirlos si no quieres\\
matar de tedio el hálito amoroso.\\
Hagamos de esta triste ausencia un mar\\
que parte en dos la costa donde a diario\\
acuden las dos partes a esperar\\
la vuelta del amor rectificado:\\
así el invierno y su rigor consiguen\\
hacer que ansiemos el verano el triple.

\end{verse}

\clearpage
\poemtitle {LXII}
\begin{verse}

Pequé de amarme a mí con todo empeño,\\
con todo mi ojo, mi alma y con mi mente,\\
y ese pecado arraiga tan adentro\\
que no hay ninguna cura que lo enmiende.\\
Creí que no había rostro como el mío\\
ni nadie con un porte tan donoso\\
y me juzgué a mí mismo, convencido\\
de que era superior a cualquier otro.\\
Mas cuando en el espejo veo mi cara\\
curtida y maltratada por los años,\\
entonces mi lectura es la contraria\\
y considero inicuo amarme tanto.\\
Y así te alabo en mí, por si pudiera\\
pintar mi edad con tu belleza fresca.

\end{verse}

\clearpage
\poemtitle {LXVI}
\begin{verse}

Que venga ya la muerte: estoy cansado\\
de ver hecho un mendigo al que más vale,\\
y que el don nadie vista con boato,\\
y al cándido lo engañe el miserable,\\
y que el honor recaiga en el indigno,\\
y que el perfecto sufra la desdicha,\\
y la doncella se hunda en el ludibrio,\\
y al fuerte lo invaliden las intrigas,\\
y que la autoridad censure el arte,\\
y la locura cure lo sensato,\\
y tachen de simpleza a las verdades,\\
y viva el bien cautivo de lo malo.\\
Mas en la muerte no hallaré reposo\\
si, muerto yo, mi amor se queda solo.

\end{verse}

\clearpage
\poemtitle {LXXII}
\begin{verse}

Olvídame, no vaya a ser que el mundo\\
te pida que recites lo que en vida\\
mostré de bueno para que, difunto,\\
me quieras aún, pues nada encontrarías\\
a menos que repares mis carencias\\
y, a fuerza de mentir virtuosamente,\\
me otorgues los halagos que me niega\\
la cruda realidad hasta en la muerte.\\
Mi nombre ha de yacer junto a mi cuerpo\\
en vez de seguir vivo y mancillarnos,\\
no vaya a ser que al fin tu amor sincero,\\
si ha de mentir por mí, parezca falso.\\
Mi oprobio es lo que pongo por delante;\\
el tuyo, amar aquello que no vale.

\end{verse}

\clearpage
\poemtitle {LXXXI}
\begin{verse}

O vivo para hacerte el epitafio\\
o vives tú y se pudrirá mi carne.\\
Si mueres, tu recuerdo estará a salvo;\\
de mí habrán olvidado cada parte.\\
Tendrá tu nombre vida para siempre\\
y a mí no habrá en el mundo quien me llore;\\
la tierra me reserva un hoyo inerte;\\
tú yaces en los ojos de los hombres.\\
Mi verso fiel será tu monumento,\\
lectura de los ojos que aún no existen;\\
y cuando estén, los que hoy suspiran, muertos\\
no faltarán las lenguas que te imiten.\\
Tú vivirás --mi pluma es garantía--\\
en tanto haya una boca que respira.

\end{verse}

\clearpage
\poemtitle {XCVII}
\begin{verse}

¡Qué invierno fue no estar junto a tu lado,\\
placer fugaz del año fulminante!\\
¡Qué fríos padecí, qué días magros!\\
¡Diciembre y su escasez en todas partes!\\
Todo ello fue en verano, y a las puertas\\
del pingüe otoño, que en su vientre orondo\\
portaba ya la mies de primavera\\
como una viuda encinta sin su esposo.\\
Quimeras, orfandad, frutos perdidos\\
es cuanto yo veía en la abundancia,\\
pues tienes al verano a tu servicio\\
y si no estás, los pájaros no cantan\\
o trinan, al cantar tan tristemente\\
que, trémulas, las hojas palidecen.

\end{verse}

\clearpage
\poemtitle {CV}
\begin{verse}

Mi amor no ha de llamarse idolatría\\
ni es ídolo mi amado si mis himnos\\
y todas mis canciones se originan\\
en uno, solo en uno y siempre el mismo.\\
Tan buenos es hoy mi amor como mañana\\
y tal es su excelencia que es constante;\\
por eso, si me ciño a la constancia,\\
me expreso sobre un todo, no su parte.\\
"Hermoso, bueno, honesto'' es mi argumento;\\
"hermoso, bueno, honesto'' o cosa igual;\\
en esas variaciones me entretengo,\\
fundiendo tres en uno sin parar.\\
Hermoso, bueno, honesto: salvo en ti,\\
se han dado separados, nunca así.

\end{verse}

\clearpage
\poemtitle {CXVI}
\begin{verse}

No admito que se impida el matrimonio\\
sincero entre dos almas. No es amor\\
el que ante la mudanza muda el modo\\
o marcha con aquel que se marchó.\\
Amor es como un faro impertinente\\
que arrostra las tormentas sin bandearse;\\
lucero de las barcas que se pierden,\\
ignoto a la razón, no a los compases.\\
Por más que hienda labios y mejillas\\
con su guadaña, no es bufón del tiempo,\\
ni muda con el paso de los días\\
ni cesa hasta las lindes de lo eterno.\\
Si no es así y estoy errado, entonces\\
ni yo escribí ni amó jamás un hombre.

\end{verse}

\clearpage
\poemtitle {CXXIII}
\begin{verse}

No, tiempo, no te jactes de que cambio:\\
sé bien que las pirámides que alzaste\\
son meras construcciones del pasado\\
que tú disfrazas como novedades.\\
Tan breve es nuestro tránsito que aquello\\
que pasa por antiguo nos admita;\\
nos place más colmar nuestro deseo\\
que recordar que es cosa conocida.\\
Te desafío a ti y a tus archivos:\\
recelo del pasado y del presente\\
pues corres a tal ritmo enloquecido\\
que tu visión y lo que vemos mienten.\\
Ni tu hoz ni tú, lo juro, impedirán\\
que siga siendo fiel a la verdad.

\end{verse}

\clearpage
\poemtitle {CXXIX}
\begin{verse}

Si en antros de impudicicia se derrama\\
vigor, eso es lascivia, que en su afán\\
es cruel, sangrienta, indigna de confianza,\\
atroz, salvaje, sórdida y brutal;\\
fugaz y aborrecida casi al vuelo,\\
buscada sin razón y al consumarse,\\
odiada sin razón, como ese cebo\\
capaz de volver loco a quien lo trague:\\
tan loco cuando busca y cuando tuvo,\\
posea, poseería o poseyera;\\
en acto, un gozo, luego, un infortunio;\\
un sueño que no pasa de promesa.\\
Lo saben todos y ninguno evita\\
el cielo que al infierno los destina.

\end{verse}

\clearpage
\poemtitle {CXXXVIII}
\begin{verse}

Mi amor jura estar hecha de verdad\\
y, aun cuando sé que me engaña, yo le creo,\\
que así ella me ve joven, virginal\\
y ajeno a sutilezas y camelos.\\
Si pienso que ella piensa que soy joven,\\
sabiendo que me sabe muy vivido,\\
es por dar crédito a su lengua innoble:\\
de la verdad, los dos nos deshicimos.\\
¿Por qué razón no dice que es injusta?\\
¿O yo que ya pasó mi primavera?\\
Amar es simular confianza mutua\\
y, en el amor maduro, no echar cuentas:\\
por eso retozamos engañados,\\
pues solo en el engaño está el halago.
\end{verse}
\end{document}
