% rubber: module xelatex
\documentclass[12pt]{article}

\usepackage[spanish]{babel}
\usepackage[utf8]{inputenc}
% \usepackage{palatino}
\usepackage{verse, gmverse}

\usepackage{fontspec}
\setmainfont{EB Garamond}

\date{}
\title{Poesía selecta de Pablo Neruda I}
\begin{document}
\maketitle
\tableofcontents

\clearpage
\poemtitle{ Nocturno (1918)}
\begin{verse}
Es de noche: medito triste y solo  
a la luz de una vela titilante  
y pienso en la alegría y en el dolo,  
en la vejez cansada  
y en la juventud gallarda y arrogante.  
	
Pienso en el mar, quizás porque en mi oído  
siento el tropel bravío de las olas:  
estoy muy lejos de ese mar temido  
del pescador que lucha por su vida  
y de su madre que lo espera sola.  
	
No solo pienso en eso, pienso en todo:  
en el pequeño insecto que camina  
en la charca de lodo  
y en el arroyo serpenteando deja de correr sus aguas cristalinas...  
	
Cuando la noche llega y es oscura  
como boca de lobo, yo me pierdo  
en reflexiones llenas de amargura  
y ensombrezco mi mente  
en la infinita edad de los recuerdos.  
	
Se concluye la vela: sus fulgores  
semejan los espasmos de agonía  
de un moribundo. Pálidos colores  
el nuevo día anuncian y con ellos  
terminan mis aladas utopías.

\end{verse}

\clearpage
\poemtitle{ Oración (1921)}
\begin{verse}
Carne doliente y machacada,  
raudal de llanto sobre cada  
noche de jergón malsano:  
en este hora yo quisiera  
ver encantarse mis quimeras  
a flor de labio, pecho y mano,  
para que desciendan ellas  
--las puras y únicas estrellas  
de los jardines de mi amor--  
en caravanas impolutas  
sobre las almas de las putas  
de estas ciudades del dolor.  
	
Mal de amor, sensual laceria:  
campana negra de miseria:  
rosas del lecho de arrabal,  
abierto al mal como un camino  
por donde va el placer y el vino  
desde la gloria al hospital.  
	
En esta hora en que las lilas  
sacuden sus hojas tranquilas  
para botar el polvo impuro,  
vuela mi espíritu intocado,  
traspasa el huerto y el vallado,  
abre la puerta, salta el muro  
	
y va enredando en su camino  
el mal dolor, el agrio sino,  
y desnudando la raigambre  
de las mujeres que lucharon  
y cayeron  
y pecaron  
y murieron  
bajo los látigos del hambre.  
	
No solo es seda lo que escribo:  
que el verso mío sea vivo  
como recuerdo en tierra ajena  
para alumbrar la mala suerte  
de los que van hacia la muerte  
como la sangre por las venas.  
	
De los que van desde la vida  
rotas las manos doloridas  
en todas las zarzas ajenas:  
de los que en estas horas quietas  
no tienen madres ni poetas  
para la pena.  
	
Porque la frente en esta hora  
se dobla y la mirada llora  
saltando dolores y muros:  
en esta hora en que las lilas  
sacuden sus hojas tranquilas  
para botar el polvo impuro.

\end{verse}

\clearpage
\poemtitle{ Un hombre anda bajo la lluvia (1922)}
\begin{verse}
Pena de mala fortuna  
que cae en mi alma y la llena.  
Pena.  
Luna.  
	
Calles blancas, calles blancas...  
...Siempre ha de haber luna cuando  
por ver si la pena arranca  
ando  
y ando...  
	
Recuerdo el rincón oscuro  
en que lloraba en mi infancia:  
los líquenes en los muros,  
las risas a la distancia.  
	
...Sombra.. silencio... una voz  
que se perdía...  
La lluvia en el techo. Atroz  
lluvia que siempre caía...  
y mi llanto, húmeda voz  
que se perdía.  
	
...Se llama y nadie responde,  
se anda por seguir andando...  
	
Andar... Andar... Hacia dónde?...  
Y hasta cuándo?...  
Nadie responde  
y se sigue andando.  
	
Amor perdido y hallado  
y otra vez la vida trunca.  
Lo que siempre se ha buscado  
no debiera hallarse nunca!  
	
Uno se cansa de amar...  
Uno vive y se ha de ir...  
Soñar... Para qué soñar?  
Vivir... Para qué vivir?  
	
...Siempre ha de haber calles blancas  
cuando por la tierra grande  
por ver si la pena arranca  
ande  
y ande...  
	
...Ande en noches sin fortuna  
bajo el vellón de la luna,  
como las almas en pena...  
	
Pena de mala fortuna  
que cae en mi alma y la llena.  
Pena.  
Luna.

\end{verse}

\clearpage
\poemtitle{ Farewell (1922)}
\begin{verse}
\textbf{ 1}
Desde el fondo de ti, y arrodillado,  
un niño triste, como yo, nos mira.  
	
Por esa vida que arderá en sus venas  
tendrían que amarrarse nuestras vidas.  
	
Por esas manos, hijas de tus manos,  
tendrían que matar las manos mías.  
	
Por sus ojos abiertos en la tierra  
veré en los tuyos lágrimas un día.  
	
\textbf{ 2}
Yo no lo quiero, Amada.  
	
Para que nada nos amarre  
que no nos una nada.  
	
Ni la palabra que aromó tu boca,  
ni lo que no dijeron las palabras.  
	
Ni la fiesta de amor que no tuvimos,  
ni tus sollozos junto a la ventana.  
	
\textbf{ 3}
(Amo el amor de los marineros  
que besan y se van.  
	
Dejan una promesa.  
No vuelven nunca más.  
	
En cada puerto una mujer espera:  
los marineros besan y se van.  
	
Una noche se acuestan con la muerte  
en el lecho del mar.  
	
\textbf{ 4}
Amo el amor que se reparte  
en besos, lecho y pan.  
	
Amor que puede ser eterno  
y puede ser fugaz.  
	
Amor que quiere liberarse  
para volver a amar.  
	
Amor divinizado que se acerca.  
Amor divinizado que se va).  
	
\textbf{ 5}
Ya no se encantarán mis ojos en tus ojos,  
ya no se endulzará junto a ti mi dolor.  
	
Pero hacia donde vaya llevaré tu mirada  
y hacia donde camines llevarás mi dolor.  
	
Fui tuyo, fuiste mía. Qué más? Juntos hicimos  
un recodo en la ruta donde el amor pasó.  
	
Fui tuyo, fuiste mía. Tú serás del que te ame,  
del que corte en tu huerto lo que he sembrado yo.  
	
Yo me voy. Estoy triste: siempre estoy triste.  
Vengo desde tus brazos. No sé hacia dónde voy.  
	
...Desde tu corazón me dice adiós un niño.  
Y yo le digo adiós.

\end{verse}

\clearpage
\poemtitle{ Amiga, no te mueras (1923)}
\begin{verse}
Amiga, no te mueras.  
	
Óyeme estas palabras que me salen ardiendo,  
y que nadie diría si yo no las dijera.  
	
Amiga, no te mueras.  
	
Yo soy el que te espera en la estrellada noche.  
El que bajo el sangriento sol poniente te espera.  
	
Miro caer los frutos en la tierra sombría.  
Miro bailar las gotas del rocío en las hierbas.  
	
En la noche al espeso perfume de las rosas,  
cuando danza la ronda de las sombras inmensas.  
	
Bajo el cielo del Sur, el que te espera cuando  
el aire de la tarde como una boca besa.  
	
Amiga, no te mueras.  
	
Yo soy el que cortó las guirnaldas rebeldes  
para el lecho selvático fragante a sol y a selva.  
	
El que trajo en los brazos jacintos amarillos.  
Y rosas desgarradas. Y amapolas sangrientas.  
	
El que cruzó los brazos por esperarte, ahora.  
El que quebró sus arcos. El que dobló sus flechas.  
	
Yo soy el que en los labios guarda sabor de uvas.  
Racimos refregados. Mordeduras bermejas.  
	
El que te llama desde las llanuras brotadas.  
Yo soy el que en la hora del amor te desea.  
	
El aire de la tarde cimbra las ramas altas.  
Ebrio, mi corazón, bajo Dios, tambalea.  
	
El río desatado rompe a llorar y a veces  
se adelgaza su voz y se hace pura y trémula.  
	
Retumba, atardecida, la queja azul del agua.  
Amiga, no te mueras!  
	
Yo soy el que te espera en la estrellada noche,  
sobre las playas áureas, sobre las rubias eras.  
	
El que cortó jacintos para tu lecho, y rosas.  
Tendido entre las hierbas yo soy el que te espera!

\end{verse}

\clearpage
\poemtitle{ Canción del macho y de la hembra! (1923)}
\begin{verse}
Canción del macho y de la hembra!  
La fruta de los siglos  
exprimiendo su jugo  
en nuestras venas.  
	
Mi alma derramándose en tu carne extendida  
para salir de ti más buena,  
el corazón desparramándose,  
estirándose como un pantera,  
y mi vida, hecha astillas, anudándose  
a tu como la luz a las estrellas!  
	
Me recibes  
como al viento la vela.  
	
Te recibo  
como el surco a la siembra.  
	
Duérmete sobre mis dolores  
si mis dolores no te queman,  
amárrate a mis alas,  
acaso mis alas te llevan,  
endereza mis deseos,  
acaso te lastima su pelea.  
	
Tú eres lo único que tengo  
desde que perdí mi tristeza!  
	
Desgárrame como una espada  
o táctame como una antena!  
	
Bésame,  
muérdeme,  
incéndiame,  
que yo vengo a tu tierra  
solo por el naufragio de mis ojos de macho  
en el agua infinita de tus ojos de hembra!

\end{verse}

\clearpage
\poemtitle{ Te recuerdo como eras en el último otoño (1923)}
\begin{verse}
Te recuerdo como eras en el último otoño.  
Eras la boina gris y el corazón en calma.  
En tus ojos peleaban las llamas del crepúsculo.  
Y las hojas caían en el agua de tu alma.  
	
Apegada a mis brazos como una enredadera,  
las hojas recogían tu voz lenta y en calma.  
Hoguera de estupor en que mi sed ardía.  
Dulce jacinto azul torcido sobre mi alma.  
	
Siento viajar tus ojos y es distante el otoño:  
boina gris, voz de pájaro y corazón de casa  
hacia donde emigraban mis profundos anhelos  
y caían mis besos alegres como brasas.  
	
Cielo desde un navío. Campo desde los cerros.  
Tu recuerdo es de luz, de humo, de estanque en calma!  
Más allá de tus ojos ardían los crepúsculos.  
Hojas secas de otoño giraban en tu alma.

\end{verse}

\clearpage
\poemtitle{ Juegas todos los días con la luz del universo (1923)}
\begin{verse}
Juegas todos los días con la luz del universo.  
Sutil visitadora, llegas en la flor y en el agua.  
Eres más que esta blanca cabecita que aprieto  
como un racimo entre mis manos cada día.  
	
A nadie te pareces desde que yo te amo.  
Déjame tenderte entre guirnaldas amarillas.  
Quién escribe tu nombre con letra de humo entre las estrellas del sur?  
Ah déjame recordarte cómo eras entonces, cuándo aún no existías.  
	
De pronto el viento aúlla y golpea mi ventana cerrada.  
El cielo es una red cuajada de peces sombríos.  
Aquí vienen a dar todos los vientos, todos.  
Se desviste la lluvia.  
	
Pasan huyendo los pájaros.  
El viento. El viento.  
Yo solo puedo luchar contra la fuerza de los hombres.  
El temporal arremolina hojas oscuras  
y suelta todas las barcas que anoche amarraron al cielo.  
	
Tú estás aquí. Ah tú no huyes.  
Tú me responderás hasta el último grito.  
Ovíllate a mi lado como si tuvieras miedo.  
Sin embargo alguna vez corrió una sombra extrañada por tus ojos.  
	
Ahora, ahora también, pequeña, me traes madreselvas,  
y tienes hasta los senos perfumados.  
Mientras el viento triste galopa matando mariposas  
yo te amo, y mi alegría muerde tu boca de ciruela.  
	
Cuánto te habrá dolido acostumbrarte a mí,  
a mi alma sola y salvaje, a mi nombre que todos ahuyentan.  
Hemos visto arder tantas veces el lucero besándonos los ojos  
y sobre nuestras cabezas destorcerse los crepúsculos en abanicos girantes.  
Mis palabras llovieron sobre ti acariciándote.  
Amé desde hace tiempo tu cuerpo de nácar soleado.  
Hasta te creo dueña del universo.  
Te traeré de las montañas flores alegres, copihues,  
avellanas oscuras, y cestas silvestres de besos.  
	
Quiero hacer contigo  
lo que la primavera hace con los cerezos.

\end{verse}

\clearpage
\poemtitle{ Me gustas cuando callas (1923)}
\begin{verse}
Me gustas cuando callas porque estás como ausente,  
y me oyes desde lejos, y mi voz no te toca.  
Parece que los ojos se te hubieran volado  
y parece que un beso te cerrara la boca.  
	
Como todas las cosas están llenas de mi alma  
emerges de las cosas, llena del alma mía.  
Mariposa de sueño, te pareces a mi alma,  
y te pareces a la palabra melancolía.  
	
Me gustas cuando callas y estás como distante.  
Y estás como quejándote, mariposa en arrullo.  
Y me oyes desde lejos, y mi voz no te alcanza:  
déjame que me calle con el silencio tuyo.  
	
Déjame que te hable también con tu silencio  
claro como una lámpara, simple como un anillo.  
Eres como la noche, callada y constelada.  
Tu silencio es de estrella, tan lejano y sencillo.  
	
Me gustas cuando callas porque estás como ausente.  
Distante y dolorosa como si hubieras muerto.  
Una palabra entonces, una sonrisa bastan.  
Y estoy alegre, alegre de que no sea cierto.

\end{verse}

\clearpage
\poemtitle{ Niña morena y ágil (1923)}
\begin{verse}
Niña morena y ágil, el sol que hace las frutas,  
el que cuaja los trigos, el que tuerce las algas,  
hizo tu cuerpo alegre, tus luminosos ojos  
y tu boca que tiene la sonrisa del agua.  
	
Un sol negro y ansioso se te arrolla en las hebras  
de la negra melena, cuando estiras los brazos.  
Tú juegas con el sol como con un estero  
y él te deja en los ojos dos oscuros remansos.  
	
Niña morena y ágil, nada hacia ti me acerca.  
Todo de ti me aleja, como del mediodía.  
Eres la delirante juventud de la abeja,  
la embriaguez de la ola, la fuerza de la espiga.  
	
Mi corazón sombrío te busca, sin embargo,  
y amo tu cuerpo alegre, tu voz suelta y delgada.  
Mariposa morena dulce y definitiva,  
como el trigal y el sol, la amapola y el agua.  
	 
Puedo escribir los versos más tristes esta noche.  
Escribir, por ejemplo: "La noche esta estrellada,  
y tiritan, azules, los astros, a lo lejos".  
El viento de la noche gira en el cielo y canta.

\end{verse}

\clearpage
\poemtitle{ Puedo escribir los versos más tristes esta noche (1923)}
\begin{verse}
Puedo escribir los versos más tristes esta noche.  
	
Escribir, por ejemplo: "La noche esta estrellada,  
y tiritan, azules, los astros, a lo lejos".  
	
El viento de la noche gira en el cielo y canta.  
	
Puedo escribir los versos más tristes esta noche.  
Yo la quise, y a veces ella también me quiso.  
	
En las noches como ésta la tuve entre mis brazos.  
La besé tantas veces bajo el cielo infinito.  
	
Ella me quiso, a veces yo también la quería.  
Cómo no haber amado sus grandes ojos fijos.  
	
Puedo escribir los versos más tristes esta noche.  
Pensar que no la tengo. Sentir que la he perdido.  
	
Oír la noche inmensa, más inmensa sin ella.  
Y el verso cae al alma como al pasto el rocío.  
	
Qué importa que mi amor no pudiera guardarla.  
La noche está estrellada y ella no está conmigo.  
	
Eso es todo. A lo lejos alguien canta. A lo lejos.  
Mi alma no se contenta con haberla perdido.  
	
Como para acercarla mi mirada la busca.  
Mi corazón la busca, y ella no está conmigo.  
	
La misma noche que hace blanquear los mismos árboles.  
Nosotros, los de entonces, ya no somos los mismos.  
	
Ya no la quiero, es cierto, pero cuánto la quise.  
Mi voz buscaba el viento para tocar su oído.  
	
De otro. Será de otro. Como antes de mis besos.  
Su voz, su cuerpo claro. Sus ojos infinitos.  
	
Ya no la quiero, es cierto, pero tal vez la quiero.  
Es tan corto el amor, y es tan largo el olvido.  
	
Porque en noches como esta la tuve entre mis brazos,  
mi alma no se contenta con haberla perdido.  
	
Aunque éste sea el último dolor que ella me causa,  
y éstos sean los últimos versos que yo le escribo.

\end{verse}

\clearpage
\poemtitle{ Alianza (Sonata) (1926)}
\begin{verse}
De miradas polvorientas caídas al suelo  
o de hojas sin sonido y sepultándose.  
De metales sin luz, con el vacío,  
con la ausencia del día muerto de golpe.  
En lo alto de las manos el deslumbrar de mariposas,  
el arrancar de mariposas cuya luz no tiene término.  
	
Tú guardabas la estela de luz, de seres rotos  
que el sol abandonado, atardeciendo, arroja a las iglesias.  
Teñida con miradas, con objeto de abejas,  
tu material de inesperada llama huyendo  
precede y sigue al día y a su familia de oro.  
Los días acechando cruzan el sigilo  
pero caen adentro de tu voz de luz.  
Oh dueña del amor, en tu descanso  
fundé mi sueño, mi actitud callada.  
	
Con tu cuerpo de número tímido, extendido de pronto  
hasta cantidades que definen la tierra,  
detrás de la pelea de los días blancos de espacio  
y fríos de muertes lentas y estímulos marchitos,  
siento arder tu regazo y transitar tus besos  
haciendo golondrinas frescas en mi sueño.  
	
A veces el destino de tus lágrimas asciende  
como la edad hasta mi frente, allí  
están golpeando las olas, destruyéndome de muerte:  
su movimiento es húmedo, decaído, final.

\end{verse}

\clearpage
\poemtitle{ Sonata y destrucciones (1928)}
\begin{verse}
Después de mucho, después de vagas leguas,  
confuso de dominios, incierto de territorios,  
acompañado de pobres esperanzas,  
y compañías infieles, y desconfiados sueños,  
amo lo tenaz que aún sobrevive en mis ojos,  
oigo en mi corazón mis pasos de jinete,  
muerdo el fuego dormido y la sal arruinada,  
y de noche, de atmósfera obscura y luto prófugo,  
aquel que vela a la orilla de los campamentos,  
el viajero armado de estériles resistencias,  
detenido entre sombras que crecen y alas que tiemblan,  
me siento ser, y mi brazo de piedra me defiende.  
	
Hay entre ciencias de llanto un altar confuso,  
y en mi sesión de atardeceres sin perfume,  
en mis abandonados dormitorios donde habita la luna,  
y arañas de mi propiedad, y destrucciones que me son queridas,  
adoro mi propio ser perdido, mi substancia imperfecta,  
mi golpe de plata y mi pérdida eterna.  
Ardió la uva húmeda, y su agua funeral  
aún vacila, aún reside,  
y el patrimonio estéril, y el domicilio traidor.  
¿Quién hizo ceremonia de cenizas?  
¿Quién amó lo perdido, quién protegió lo último?  
¿El hueso del padre, la madera del buque muerto,  
y su propio final, su misma huida,  
su fuerza triste, su dios miserable?  
Acecho, pues, lo inanimado y lo doliente,  
y el testimonio extraño que sostengo,  
con eficiencia cruel y escrito en cenizas,  
es la forma de olvido que prefiero,  
el nombre que doy a la tierra, el valor de mis sueños,  
la cantidad interminable que divido  
con mis ojos de invierno, durante cada día de este mundo.

\end{verse}

\clearpage
\poemtitle{ Diurno doliente (1928)}
\begin{verse}
De pasión sobrante y sueños de ceniza  
un pálido palio llevo, un cortejo evidente,  
un viento de metal que vive solo,  
un sirviente mortal vestido de hambre,  
y en lo fresco que baja del árbol, en la esencia del sol  
que su salud de astro implanta en las flores,  
cuando a mi piel parecida al oro llega el placer,  
tú, fantasma coral con pies de tigre,  
tú, ocasión funeral, reunión ígnea,  
acechando la patria en que sobrevivo  
con tus lanzas lunares que tiemblan un poco.  
	
Porque la ventana que el medio día vacío atraviesa  
tiene un día cualquiera mayor aire en sus alas,  
el frenesí hincha el traje y el sueño al sombrero,  
una abeja extremada arde sin tregua.  
Ahora, ¿qué imprevisto paso hace crujir los caminos?  
¿Qué vapor de estación lúgubre, qué rostro de cristal,  
y aún más, qué sonido de carro viejo con espigas?  
Ay, una a una, la ola que llora y la sal que se triza,  
y el tiempo del amor celestial que pasa volando,  
han tenido voz de huéspedes y espacio en la espera.  
	
De distancias llevadas a cabo, de resentimientos infieles,  
de hereditarias esperanzas mezcladas con sombra,  
de asistencias desgarradoramente dulces  
y días de transparente veta y estatura floral,  
¿que subsiste en mi término escaso, en mi débil producto?  
De mi lecho amarillo y de mi substancia estrellada,  
¿quién no es vecino y ausente a la vez?  
Un esfuerzo que salta, un flechazo de trigo  
tengo, y un arco en mi pecho manifiestamente espera,  
y un latido delgado, de agua y tenacidad,  
como algo que se quiebra perpetuamente,  
atraviesa hasta el fondo mis separaciones,  
apaga mi poder y propaga mi duelo.

\end{verse}

\clearpage
\poemtitle{ Arte poética (1928)}
\begin{verse}
\itshape
Entre sombra y espacio, entre guarniciones y doncellas,  
dotado de corazón singular y sueños funestos,  
precipitadamente pálido, marchito en la frente  
y con luto de viudo furioso por cada día de vida,  
ay, para cada agua invisible que bebo soñolientamente  
y de todo sonido que acojo temblando,  
tengo la misma sed ausente y la misma fiebre fría  
un oído que nace, una angustia indirecta,  
como si llegaran ladrones o fantasmas,  
y en una cáscara de extensión fija y profunda,  
como un camarero humillado, como una campana un poco ronca,  
como un espejo viejo, como un olor de casa sola  
en la que los huéspedes entran de noche perdidamente ebrios,  
y hay un olor de ropa tirada al suelo, y una ausencia de flores  
—posiblemente de otro modo aún menos melancólico—,  
pero, la verdad, de pronto, el viento que azota mi pecho,  
las noches de substancia infinita caídas en mi dormitorio,  
el ruido de un día que arde con sacrificio  
me piden lo profético que hay en mí, con melancolía  
y un golpe de objetos que llaman sin ser respondidos  
hay, y un movimiento sin tregua, y un nombre confuso.

\end{verse}

\clearpage
\poemtitle{ Significa sombras (1929)}
\begin{verse}
¿Qué esperanza considerar, qué presagio puro,  
qué definitivo beso enterrar en el corazón,  
someter en los orígenes del desamparo y la inteligencia,  
suave y seguro sobre las aguas eternamente turbadas?  
	
¿Qué vitales, rápidas alas de un nuevo ángel de sueños  
instalar en mis hombros dormidos para seguridad perpetua,  
de tal manera que el camino entre las estrellas de la muerte  
sea un violento vuelo comenzado desde hace muchos días y meses y siglos?  
	
Tal vez la debilidad natural de los seres recelosos y ansiosos  
busca de súbito permanencia en el tiempo y límites en la tierra,  
tal vez las fatigas y las edades acumuladas implacablemente  
se extienden como la ola lunar de un océano recién creado  
sobre litorales y tierras angustiosamente desiertas.  
	
Ay, que lo que soy siga existiendo y cesando de existir,  
y que mi obediencia se ordene con tales condiciones de hierro  
que el temblor de las muertes y de los nacimientos no conmueva  
el profundo sitio que quiero reservar para mí eternamente.  
	
Sea, pues, lo que soy, en alguna parte y en todo tiempo,  
establecido y asegurado y ardiente testigo,  
cuidadosamente destruyéndose y preservándose incesantemente,  
evidentemente empeñado en su deber original.

\end{verse}

\clearpage
\poemtitle{ Ritual de mis piernas (1930)}
\begin{verse}
Largamente he permanecido mirando mis largas piernas,  
con ternura infinita y curiosa, con mi acostumbrada pasión,  
como si hubieran sido las piernas de una mujer «divina»  
profundamente sumida en el abismo de mi tórax:  
y es que, la verdad, cuando el tiempo, el tiempo pasa,  
sobre la tierra, sobre el techo, sobre mi impura cabeza,  
y pasa, el tiempo pasa, y en mi lecho no siento de noche que una  
mujer está respirando, durmiendo desnuda y a mi lado,  
entonces, extrañas, oscuras cosas toman el lugar de la ausente,  
viciosos, melancólicos pensamientos  
siembran pesadas posibilidades en mi dormitorio,  
y así, pues, miro mis piernas como si pertenecieran a otro cuerpo,  
y fuerte y dulcemente estuvieran pegadas a mis entrañas.  
	
Como tallos o femeninas, adorables cosas,  
desde las rodillas suben, cilíndricas y espesas,  
con turbado y compacto material de existencia;  
como brutales, gruesos brazos de diosa,  
como árboles monstruosamente vestidos de seres humanos,  
como fatales, inmensos labios sedientos y tranquilos,  
son allí la mejor parte de mi cuerpo:  
lo enteramente sustancial, sin complicado contenido  
de sentidos o tráqueas o intestinos o ganglios:  
nada, sino lo puro, lo dulce y espeso de mi propia vida,  
guardando la vida, sin embargo, de una manera completa.  
	
Las gentes cruzan el mundo en la actualidad  
sin apenas recordar que poseen un cuerpo y en él la vida,  
y hay miedo, hay miedo en el mundo de las palabras que designan el cuerpo,  
y se habla favorablemente de la ropa,  
de pantalones es posible hablar, de trajes,  
y de ropa interior de mujer (de medias y ligas de «señora»),  
como si por las calles fueran las prendas y los trajes vacíos por completo  
y un oscuro y obsceno guardarropas ocupara el mundo.  
	
Tienen existencia los trajes, color, forma, designio,  
y profundo lugar en nuestros mitos, demasiado lugar,  
demasiados muebles y demasiadas habitaciones hay en el mundo,  
y mi cuerpo vive entre y bajo tantas cosas abatido,  
con un pensamiento fijo de esclavitud y de cadenas.  
Bueno, mis rodillas, como nudos,  
particulares, funcionarios, evidentes,  
separan las mitades de mis piernas en forma seca:  
y en realidad dos mundos diferentes, dos sexos diferentes  
no son tan diferentes como las dos mitades de mis piernas.  
Desde la rodilla hasta el pie una forma dura,  
mineral, fríamente útil, aparece,  
una criatura de hueso y persistencia,  
y los tobillos no son ya sino el propósito desnudo,  
la exactitud y lo necesario dispuestos en definitiva.  
	
Sin sensualidad, cortas y duras, y masculinas,  
son allí mis piernas, y dotadas  
de grupos musculares como animales complementarios,  
y allí también una vida, una sólida, sutil, aguda vida  
sin temblar permanece, aguardando y actuando.  
En mis pies cosquillosos,  
y duros como el sol, y abiertos como flores,  
y perpetuos, magníficos soldados  
en la guerra gris del espacio,  
todo termina, la vida termina definitivamente en mis pies,  
lo extranjero y lo hostil allí comienza:  
los nombres del mundo, lo fronterizo y lo remoto,  
lo sustantivo y lo adjetivo que no caben en mi corazón  
con densa y fría constancia allí se originan.  
	
Siempre,  
productos manufacturados, medias, zapatos,  
o simplemente aire infinito,  
habrá entre mis pies y la tierra  
extremando lo aislado y lo solitario de mi ser,  
algo tenazmente supuesto entre mi vida y la tierra,  
algo abiertamente invencible y enemigo.

\end{verse}

\clearpage
\poemtitle{ Lamento lento (1931)}
\begin{verse}
\itshape
En la noche del corazón  
la gota de tu nombre lento  
en silencio circula y cae  
y rompe y desarrolla su agua.  
	
Algo quiere su leve daño  
y su estima infinita y corta,  
como el paso de un ser perdido  
de pronto oído.  
	
De pronto, de pronto escuchado  
y repartido en el corazón  
con triste insistencia y aumento  
como un sueño frío de otoño.  
	
La espesa rueda de la tierra  
su llanta húmeda de olvido  
hace rodar, cortando el tiempo  
en mitades inaccesibles.  
	
Sus copas duras cubren tu alma  
derramada en la tierra fría  
con sus pobres chispas azules  
volando en la voz de la lluvia.
\end{verse}
\end{document}
