% rubber: module xelatex
\documentclass[12pt]{article}

\usepackage[spanish]{babel}
\usepackage[utf8]{inputenc}
% \usepackage{palatino}
\usepackage{verse}

\usepackage{fontspec}
\setmainfont{EB Garamond}

\date{}
\title{Poesía selecta de Pablo Neruda III}
\begin{document}
\maketitle
\tableofcontents
\clearpage
\poemtitle {Alturas de Macchu Picchu (1946)}
\begin{verse}

{\bfseries\scshape {I}}

Del aire al aire, como una red vacía,\\
iba yo entre las calles y la atmósfera, llegando y despidiendo,\\
en el advenimiento del otoño la moneda extendida\\
de las hojas, y entre la primavera y las espigas,\\
lo que el más grande amor, como dentro de un guante\\
que cae, nos entrega como una larga luna.  

(Días de fulgor vivo en la intemperie\\
de los cuerpos: aceros convertidos\\
al silencio del ácido:\\
noches deshilachadas hasta la última harina:\\
estambres agredidos de la patria nupcial.)  

Alguien que me esperó entre los violines\\
encontró un mundo como una torre enterrada\\
hundiendo su espiral más abajo de todas\\
las hojas de color de ronco azufre:\\
más abajo, en el oro de la geología,\\
como una espada envuelta en meteoros,\\
hundí la mano turbulenta y dulce\\
en lo más genital de lo terrestre.  

Puse la frente entre las olas profundas,\\
descendí como gota entre la paz sulfúrica,\\
y, como un ciego, regresé al jazmín\\
de la gastada primavera humana.  

{\bfseries\scshape {II}}

Si la flor a la flor entrega el alto germen\\
y la roca mantiene su flor diseminada\\
en su golpeado traje de diamante y arena,\\
el hombre arruga el pétalo de la luz que recoge\\
en los determinados manantiales marinos\\
y taladra el metal palpitante en sus manos.\\
Y pronto, entre la ropa y el humo, sobre la mesa hundida\\
como una barajada cantidad, queda el alma:\\
cuarzo y desvelo, lágrimas en el océano\\
como estanques de frío: pero aún\\
mátala y agonízala con papel y con odio,\\
sumérgela en la alfombra cotidiana, desgárrala\\
entre las vestiduras hostiles del alambre.  

No: por los corredores, aire, mar o caminos,\\
quién guarda sin puñal (como las encarnadas\\
amapolas) su sangre? La cólera ha extenuado\\
la triste mercancía del vendedor de seres,\\
y, mientras en la altura del ciruelo, el rocío\\
desde mil años deja su carta transparente\\
sobre la misma rama que lo espera, oh corazón, oh frente triturada\\
entre las cavidades del otoño:  

Cuántas veces en las calles de invierno de una ciudad o en\\
un autobús o un barco en el crepúsculo, o en la soledad\\
más espesa, la de la noche de fiesta, bajo el sonido\\
de sombras y campanas, en la misma gruta del placer humano,\\
me quise detener a buscar la eterna veta insondable\\
que antes toqué en la piedra o en el relámpago que el beso desprendía.  

(Lo que en el cereal como una historia amarilla\\
de pequeños pechos preñados va repitiendo un número\\
que sin cesar es ternura en las capas germinales,\\
y que, idéntica siempre, se desgrana en marfil\\
y lo que en el agua es patria transparente, campana\\
desde la nieve aislada hasta las olas sangrientas.)  

No pude asir sino un racimo de rostros o de máscaras\\
precipitadas, como anillos de oro vacío,\\
como ropas dispersas hijas de un otoño rabioso\\
que hiciera temblar el miserable árbol de las razas asustadas.  

No tuve sitio donde descansar la mano\\
y que, corriente como agua de manantial encadenado,\\
o firme como grumo de antracita o cristal,\\
hubiera devuelto el calor o el frío de mi mano extendida.\\
Qué era el hombre? En qué parte de su conversación abierta\\
entre los almacenes y los silbidos, en cuál de sus movimientos metálicos\\
vivía lo indestructible, lo imperecedero, la vida?  

{\bfseries\scshape {III}}

El ser como el maíz se desgranaba en el inacabable\\
granero de los hechos perdidos, de los acontecimientos\\
miserables, del uno al siete, al ocho,\\
y no una muerte, sino muchas muertes llegaba a cada uno:\\
cada día una muerte pequeña, polvo, gusano, lámpara\\
que se apaga en el lodo del suburbio, una pequeña muerte de alas gruesas\\
entraba en cada hombre como una corta lanza\\
y era el hombre asediado del pan o del cuchillo,\\
el ganadero: el hijo de los puertos, o el capitán oscuro del arado,\\
o el roedor de las calles espesas:  

todos fallecieron esperando su muerte, su corta muerte diaria:\\
y su quebranto aciago de cada día era\\
como una copa negra que bebían temblando.  

{\bfseries\scshape {IV}}

La poderosa muerte me invitó muchas veces:\\
era como la sal invisible en las olas,\\
y lo que su invisible sabor diseminaba\\
era como mitades de hundimientos y altura\\
o vastas construcciones de viento y ventisquero.\\
Yo al férreo filo vine, a la angostura\\
del aire, a la mortaja de agricultura y piedra,\\
al estelar vacío de los pasos finales\\
y a la vertiginosa carretera espiral:\\
pero, ancho mar, ¡oh muerte!, de ola en ola no vienes,\\
sino como un galope de claridad nocturna\\
o como los totales números de la noche.\\
Nunca llegaste a hurgar en el bolsillo, no era\\
posible tu visita sin vestimenta roja:\\
sin auroral alfombra de cercado silencio:\\
sin altos o enterrados patrimonios de lágrimas.  

No pude amar en cada ser un árbol\\
con su pequeño otoño a cuestas (la muerte de mil hojas),\\
todas las falsas muertes y las resurrecciones\\
sin tierra, sin abismo:\\
quise nadar en las más anchas vidas,\\
en las más sueltas desembocaduras,\\
y cuando poco a poco el hombre fue negándome\\
y fue cerrando paso y puerta para que no tocaran\\
mis manos manantiales su inexistencia herida,\\
entonces fui por calle y calle y río y río,\\
y ciudad y ciudad y cama y cama,\\
y atravesó el desierto mi máscara salobre,\\
y en las últimas casas humilladas, sin lámpara, sin fuego,\\
sin pan, sin piedra, sin silencio, solo,\\
rodé muriendo de mi propia muerte.  

{\bfseries\scshape {V}}

No eres tú, muerte grave, ave de plumas férreas,\\
la que el pobre heredero de las habitaciones\\
llevaba entre alimentos apresurados, bajo la piel vacía:\\
era algo, un pobre pétalo de cuerda exterminada:\\
un átomo del pecho que no vino al embate\\
o el áspero rocío que no cayó en la frente.\\
Era lo que no pudo renacer, un pedazo\\
de la pequeña muerte sin paz ni territorio:\\
un hueso, una campana que morían en él.\\
Yo levanté las vendas del yodo, hundí las manos\\
en los pobres dolores que mataban la muerte,\\
y no encontré en la herida sino una racha fría\\
que entraba por los vagos intersticios del alma.  

{\bfseries\scshape {VI}}

Entonces en la escala de la tierra he subido\\
entre la atroz maraña de las selvas perdidas\\
hasta ti, Macchu Picchu.\\
Alta ciudad de piedras escalares,\\
por fin morada del que lo terrestre\\
no escondió en las dormidas vestiduras.\\
En ti, como dos líneas paralelas,\\
la cuna del relámpago y del hombre\\
se mecían en un viento de espinas.  

Madre de piedra, espuma de los cóndores.\\
Alto arrecife de la aurora humana.  

Pala perdida en la primera arena.  

Esta fue la morada, este es el sitio:\\
aquí los anchos granos del maíz ascendieron\\
y bajaron de nuevo como granizo rojo.  

Aquí la hebra dorada salió de la vicuña\\
a vestir los amores, los túmulos, las madres,\\
el rey, las oraciones, los guerreros.  

Aquí los pies del hombre descansaron de noche\\
junto a los pies del águila en las altas guaridas\\
carniceras, y en la aurora\\
pisaron con los pies del trueno la niebla enrarecida\\
y tocaron las tierras y las piedras\\
hasta reconocerlas en la noche o la muerte.  

Miro las vestiduras y las manos,\\
el vestigio del agua en la oquedad sonora,\\
la pared suavizada por el tacto de un rostro\\
que miró con mis ojos las lámparas terrestres,\\
que aceitó con mis manos las desaparecidas\\
maderas: porque todo, ropaje, piel, vasijas,\\
palabras, vino, panes,\\
se fue, cayó a la tierra.  

Y el aire entró con dedos\\
de azahar sobre todos los dormidos:\\
mil años de aire, meses, semanas de aire,\\
de viento azul, de cordillera férrea,\\
que fueron como suaves huracanes de pasos\\
lustrando el solitario recinto de la piedra.  

{\bfseries\scshape {VII}}

Muertos de un solo abismo, sombras de una hondonada,\\
la profunda, es así como al tamaño\\
de vuestra magnitud\\
vino la verdadera, la más abrasadora\\
muerte y desde las rocas taladradas,\\
desde los capiteles escarlata,\\
desde los acueductos escalares\\
os desplomasteis corto en un otoño\\
en una sola muerte.\\
Hoy el aire vacío ya no llora,\\
ya no conoce vuestros pies de arcilla,\\
ya olvidó vuestros cántaros que filtraban el cielo\\
cuando lo derramaban los cuchillos del rayo,\\
y el árbol poderoso fue comido\\
por la niebla, y cortado por la racha.\\
Él sostuvo una mano que cayó de repente\\
desde la altura hasta el final del tiempo.\\
Ya no sois, manos de araña, débiles\\
hebras, tela enmarañada:\\
cuanto fuisteis cayó: costumbres, sílabas\\
raídas, máscaras de luz deslumbradora.  

Pero una permanencia de piedra y de palabra:\\
la ciudad como un vaso se levantó en las manos\\
de todos, vivos, muertos, callados, sostenidos\\
de tanta muerte, un muro, de tanta vida un golpe\\
de pétalos de piedra: la rosa permanente, la morada:\\
este arrecife andino de colonias glaciales.  

Cuando la mano de color de arcilla\\
se convirtió en arcilla, y cuando los pequeños párpados se cerraron\\
llenos de ásperos muros, poblados de castillos,\\
y cuando todo el hombre se enredó en su agujero,\\
quedó la exactitud enarbolada:\\
el alto sitio de la aurora humana:\\
la más alta vasija que contuvo el silencio:\\
una vida de piedra después de tantas vidas.  

{\bfseries\scshape {VIII}}

Sube conmigo, amor americano.\\
Besa conmigo las piedras secretas.  

La plata torrencial del Urubamba\\
hace volar el polen a su copa amarilla.\\
Vuela el vacío de la enredadera,\\
la planta pétrea, la guirnalda dura\\
sobre el silencio del cajón serrano.  

Ven, minúscula vida, entre las alas\\
de la tierra, mientras --cristal y frío, aire golpeado--\\
apartando esmeraldas combatidas,\\
oh, agua salvaje, bajas de la nieve.  

Amor, amor, hasta la noche abrupta,\\
desde el sonoro pedernal andino,\\
hacia la aurora de rodillas rojas,\\
contempla el hijo ciego de la nieve.  

Oh, Wilkamayu de sonoros hilos,\\
cuando rompes tus truenos lineales\\
en blanca espuma, como herida nieve,\\
cuando tu vendaval acantilado\\
canta y castiga despertando al cielo,\\
qué idioma traes a la oreja apenas\\
desarraigada de tu espuma andina?  

Quién apresó el relámpago del frío\\
y lo dejó en la altura encadenado,\\
repartido en sus lágrimas glaciales,\\
sacudido en sus rápidas espadas,\\
golpeando sus estambres aguerridos,\\
conducido en su cama de guerrero,\\
sobresaltado en su final de roca?  

Qué dicen tus destellos acosados?\\
Tu secreto relámpago rebelde\\
antes viajó poblado de palabras?\\
Quién va rompiendo sílabas heladas,\\
idiomas negros, estandartes de oro,\\
bocas profundas, gritos sometidos,\\
en tus delgadas aguas arteriales?  

Quién va cortando párpados florales\\
que vienen a mirar desde la tierra?\\
Quién precipita los racimos muertos\\
que bajan en tus manos de cascada\\
a desgranar su noche desgranada\\
en el carbón de la geología?  

Quién despeña la rama de los vínculos?\\
Quién otra vez sepulta los adioses?  

Amor, amor, no toques la frontera,\\
ni adores la cabeza sumergida:\\
deja que el tiempo cumpla su estatura\\
en su salón de manantiales rotos,\\
y, entre el agua veloz y las murallas,\\
recoge el aire del desfiladero,\\
las paralelas láminas del viento,\\
el canal ciego de las cordilleras,\\
el áspero saludo del rocío,\\
y sube, flor a flor, por la espesura,\\
pisando la serpiente despeñada.  

En la escarpada zona, piedra y bosque,\\
polvo de estrellas verdes, selva clara,\\
Mantur estalla como un lago vivo\\
o como un nuevo piso del silencio.  

Ven a mi propio ser, al alba mía,\\
hasta las soledades coronadas.\\
El reino muerto vive todavía.\\
Y en el Reloj la sombra sanguinaria\\
del cóndor cruza como una nave negra.  

{\bfseries\scshape {IX}}

Águila sideral, viña de bruma.\\
Bastión perdido, cimitarra ciega.\\
Cinturón estrellado, pan solemne.\\
Escala torrencial, párpado inmenso.\\
Túnica triangular, polen de piedra.\\
Lámpara de granito, pan de piedra.\\
Serpiente mineral, rosa de piedra.\\
Nave enterrada, manantial de piedra.\\
Caballo de la luna, luz de piedra.\\
Escuadra equinoccial, vapor de piedra.\\
Geometría final, libro de piedra.\\
Témpano entre las ráfagas labrado.\\
Madrépora del tiempo sumergido.\\
Muralla por los dedos suavizada.\\
Techumbre por las plumas combatida.\\
Ramos de espejo, bases de tormenta.\\
Tronos volcados por la enredadera.\\
Régimen de la garra encarnizada.\\
Vendaval sostenido en la vertiente.\\
Inmóvil catarata de turquesa.\\
Campana patriarcal de los dormidos.\\
Argolla de las nieves dominadas.\\
Hierro acostado sobre sus estatuas.\\
Inaccesible temporal cerrado.\\
Manos de puma, roca sanguinaria.\\
Torre sombrera, discusión de nieve.\\
Noche elevada en dedos y raíces.\\
Ventanas de las nieblas, paloma endurecida.\\
Planta nocturna, estatua de los truenos.\\
Cordillera esencial, techo marino.\\
Arquitectura de águilas perdidas.\\
Cuerda del cielo, abeja de la altura.\\
Nivel sangriento, estrella construida.\\
Burbuja mineral, luna de cuarzo.\\
Serpiente andina, frente de amaranto.\\
Cúpula del silencio, patria pura.\\
Novia del mar, árbol de catedrales.\\
Ramo de sal, cerezo de alas negras.\\
Dentadura nevada, trueno frío.\\
Luna arañada, piedra amenazante.\\
Cabellera del frío, acción del aire.\\
Volcán de manos, catarata oscura.\\
Ola de plata, dirección del tiempo.  

{\bfseries\scshape {X}}

Piedra en la piedra, el hombre, dónde estuvo?\\
Aire en el aire, el hombre, dónde estuvo?\\
Tiempo en el tiempo, el hombre, dónde estuvo?\\
Fuiste también el pedacito roto\\
del hombre inconcluso, de águila vacía\\
que por las calles de hoy, que por las huellas,\\
que por las hojas del otoño muerto\\
va machacando el alma hasta la tumba?\\
La pobre mano, el pie, la pobre vida\ldots{}\\
Los días de la luz deshilachada\\
en ti, como la lluvia\\
sobre las banderillas de la fiesta,\\
dieron pétalo a pétalo de su alimento oscuro\\
en la boca vacía?\\
Hambre, coral del hombre,\\
hambre, planta secreta, raíz de los leñadores,\\
hambre, subió tu raya de arrecife\\
hasta estas altas torres desprendidas?  

Yo te interrogo, sal de los caminos,\\
muéstrame la cuchara, déjame, arquitectura,\\
roer con un palito los estambres de piedra,\\
subir todos los escalones del aire hasta el vacío,\\
rascar la entraña hasta tocar el hombre.  

Macchu Picchu, pusiste\\
piedras en la piedra, y en la base, harapo?\\
Carbón sobre carbón, y en el fondo la lágrima?\\
Fuego en el oro, y en él, temblando el rojo\\
goterón de la sangre?\\
Devuélveme el esclavo que enterraste!\\
Sacude de las tierras el pan duro\\
del miserable, muéstrame los vestidos\\
del siervo y su ventana.\\
Dime cómo durmió cuando vivía.\\
Dime si fue su sueño\\
ronco, entreabierto, como un hoyo negro\\
hecho por la fatiga sobre el muro.\\
El muro, el muro! Si sobre su sueño\\
gravitó cada piso de piedra, y si cayó bajo ella\\
como bajo una luna, con el sueño!  

Antigua América, novia sumergida,\\
también tus dedos,\\
al salir de la selva hacia el alto vacío de los dioses,\\
bajo los estandartes nupciales de la luz y el decoro,\\
mezclándose al trueno de los tambores y de las lanzas,\\
también, también tus dedos,\\
los que la rosa abstracta y la línea del frío, los\\
que el pecho sangriento del nuevo cereal trasladaron\\
hasta la tela de materia radiante, hasta las duras cavidades,\\
también, también, América enterrada, guardaste en lo más bajo,\\
en el amargo intestino, como un águila, el hambre?  

{\bfseries\scshape {XI}}

A través del confuso esplendor,\\
a través de la noche de piedra, déjame hundir la mano\\
y deja que en mí palpite, como un ave mil años prisionera,\\
el viejo corazón del olvidado!\\
Déjame olvidar hoy esta dicha, que es más ancha que el mar,\\
porque el hombre es más ancho que el mar y que sus islas,\\
y hay que caer en él como en u n pozo para salir del fondo\\
con un ramo de agua secreta y de verdades sumergidas.\\
Déjame olvidar, ancha piedra, la proporción poderosa,\\
la trascendente medida, las piedras del panal,\\
y de la escuadra déjame hoy resbalar\\
la mano sobre la hipotenusa de áspera sangre y cilicio.\\
Cuando, como una herradura de élitros rojos, el cóndor furibundo\\
me golpea las sienes en el orden del vuelo\\
y el huracán de plumas carniceras barre el polvo sombrío\\
de las escalinatas diagonales, no veo a la bestia veloz,\\
no veo el ciego ciclo de sus garras,\\
veo el antiguo ser, servidor, el dormido\\
en los campos, veo un cuerpo, mil cuerpos, un hombre, mil mujeres,\\
bajo la racha negra, negros de lluvia y noche,\\
con la piedra pesada de la estatua:\\
Juan Cortapiedras, hijo de Wiracocha,\\
Juan Comefrío, hijo de estrella verde,\\
Juan Piesdescalzos, nieto de la turquesa,\\
sube a nacer conmigo, hermano.  

{\bfseries\scshape {XII}}

Sube a nacer conmigo, hermano.  

Dame la mano desde la profunda\\
zona de tu dolor diseminado.\\
No volverás del fondo de las rocas.\\
No volverás del tiempo subterráneo.\\
No volverá tu voz endurecida.\\
No volverán tus ojos taladrados.\\
Mírame desde el fondo de la tierra,\\
labrador, tejedor, pastor callado:\\
domador de guanacos tutelares:\\
albañil del andamio desafiado:\\
aguador de las lágrimas andinas:\\
joyero de los dedos machacados:\\
agricultor temblando en la semilla:\\
alfarero en tu greda derramado:\\
traed a la copa de esta nueva vida\\
vuestros viejos dolores enterrados.\\
Mostradme vuestra sangre y vuestro surco,\\
decidme: aquí fui castigado,\\
porque la joya no brilló o la tierra\\
no entregó a tiempo la piedra o el grano:\\
señaladme la piedra en que caísteis\\
y la madera en que os crucificaron,\\
encendedme los viejos pedernales,\\
las viejas lámparas, los látigos pegados\\
a través de los siglos en las llagas\\
y las hachas de brillo ensangrentado.\\
Yo vengo a hablar por vuestra boca muerta.\\
A través de la tierra juntad todos\\
los silenciosos labios derramados\\
y desde el fondo habladme toda esta larga noche,\\
como si yo estuviera con vosotros anclado,\\
contadme todo, cadena a cadena,\\
eslabón a eslabón, y paso a paso,\\
afilad los cuchillos que guardasteis,\\
ponedlos en mi pecho y en mi mano,\\
como un río de rayos amarillos,\\
como un río de tigres enterrados,\\
y dejadme llorar, horas, días, años,\\
edades ciegas, siglos estelares.  

Dadme el silencio, el agua, la esperanza.  

Dadme la lucha, el hierro, los volcanes.  

Apagadme los cuerpos como imanes.  

Acudid a mis venas y a mi boca.  

Hablad por mis palabras y mi sangre.  

\end{verse}

\clearpage
\poemtitle {La letra (1946)}
\begin{verse}

Así fue. Y así será. En las sierras\\
calcáreas, y a la orilla\\
del humo, en los talleres,\\
hay un mensaje escrito en las paredes\\
y el pueblo, sólo el pueblo, puede verlo.\\
Sus letras transparentes se formaron\\
con sudor y silencio. Están escritas.\\
Las amasaste, pueblo, en tu camino\\
y están sobre la noche como el fuego\\
abrasador y oculto de la aurora.\\
Entra, pueblo, en las márgenes del día.\\
Anda como un ejército, reunido,\\
y golpea la tierra con tus pasos\\
y con la misma identidad sonora.\\
Sea uniforme tu camino como\\
es uniforme el sudor en la batalla,\\
uniforme la sangre polvorienta\\
del pueblo fusilado en los caminos.  

Sobre esta claridad irá naciendo\\
la granja, la ciudad, la minería,\\
y sobre esta unidad como la tierra\\
firme y germinadora se ha dispuesto\\
la creadora permanencia, el germen\\
de la nueva ciudad para las vidas.\\
Luz de los gremios maltratados, patria\\
amasada por manos metalúrgicas,\\
orden salido de los pescadores\\
como un ramo del mar, muros armados\\
por la albañilería desbordante,\\
escuelas cereales, armaduras\\
de fábricas amadas por el hombre.\\
Paz desterrada que regresas, pan\\
compartido, aurora, sortilegio\\
del amor terrenal, edificado\\
sobre los cuatro vientos del planeta.  

\end{verse}

\clearpage
\poemtitle {Salitre (1946)}
\begin{verse}

Salitre, harina de la luna llena,\\
cereal de la pampa calcinada,\\
espuma de las ásperas arenas,\\
jazminero de flores enterradas.  

Polvo de estrella hundida en tierra oscura,\\
nieve de soledades abrasadas,\\
cuchillo de nevada empuñadura,\\
rosa blanca de sangre salpicada.  

Junto a tu nívea luz de estalactita,\\
duelo, viento y dolor, el hombre habita:\\
harapo y soledad son su medalla.  

Hermanos de la tierras desoladas:\\
aquí tenéis como un montón de espadas\\
mi corazón dispuesto a la batalla.  

\end{verse}

\clearpage
\poemtitle {La patria prisionera (1947)}
\begin{verse}

Patria de mi ternura y mis dolores,\\
patria de amor, de primavera y agua,\\
hoy sangran tus banderas tricolores\\
sobre las alambradas de Pisagua.  

Existes, patria, sobre los temores\\
y arde tu corazón de fuego y fragua\\
hoy, entre carceleros y traidores,\\
ayer, entre los muros de Rancagua.  

Pero saldrás al aire, a la alegría,\\
saldrás del duelo de estas agonías,\\
y de esta sumergida primavera,  

libre en la dignidad de tu derecho\\
y cantará en la luz, y a pleno pecho,\\
tu dulce voz, oh, patria prisionera!  

\end{verse}

\clearpage
\poemtitle {Margarita Naranjo (Salitrera ``María Elena'', Antofagasta) (1948)}
\begin{verse}

Estoy muerta. Soy de María Elena.\\
Toda mi vida la viví en la pampa.\\
Dimos la sangre para la Compañía\\
norteamericana, mis padres antes, mis hermanos.\\
Sin que hubiera huelga, sin nada nos rodearon.\\
Era de noche, vino todo el Ejército,\\
iban de casa en casa despertando a la gente,\\
llevándola al campo de concentración.\\
Yo esperaba que nosotros no fuéramos.\\
Mi marido ha trabajado tanto para la Compañía,\\
y para el Presidente, fue el más esforzado,\\
consiguiendo los votos aquí, es tan querido,\\
nadie tiene nada que decir de él, él lucha\\
por sus ideales, es puro y honrado\\
como pocos. Entonces vinieron a nuestra puerta,\\
mandados por el coronel Urízar,\\
y lo sacaron a medio vestir y a empellones\\
lo tiraron al camión que partió en la noche,\\
hacia Pisagua, hacia la oscuridad. Entonces\\
me pareció que no podía respirar más, me parecía\\
que la tierra faltaba debajo de los pies,\\
es tanta la traición, tanta la injusticia,\\
que me subió a la garganta algo como un sollozo\\
que no me dejó vivir. Me trajeron comida\\
las compañeras, y les dije: ``No comeré hasta que vuelva''.\\
Al tercer día hablaron al señor Urízar,\\
que se rió con grandes carcajadas, enviaron\\
telegramas y telegramas que el tirano en Santiago\\
no contestó. Me fui durmiendo y muriendo,\\
sin comer, apreté los dientes para no recibir\\
ni siquiera la sopa o el agua. No volvió, no volvió,\\
y poco a poco me quedé muerta, y me enterraron:\\
aquí, en el cementerio de la oficina salitrera,\\
había en esa tarde un viento de arena,\\
lloraban los viejos y las mujeres y cantaban\\
las canciones que tantas veces canté con ellos.\\
Si hubiera podido, habría mirado a ver si estaba\\
Antonio, mi marido, pero no estaba, no estaba,\\
no lo dejaron venir ni a mi muerte: ahora,\\
aquí estoy muerta, en el cementerio de la pampa\\
no hay más que soledad en torno a mí, que ya no existo,  

\end{verse}

\clearpage
\poemtitle {Amo, Valparaíso, cuanto encierras (1948)}
\begin{verse}

Amo, Valparaíso, cuanto encierras,\\
y cuanto irradias, novia del océano,\\
hasta más lejos de tu nimbo sordo.\\
Amo la luz violeta con que acudes\\
al marinero en la noche del mar,\\
y entonces eres --rosa de azahares--\\
luminosa y desnuda, fuego y niebla.\\
Que nadie venga con un martillo turbio\\
a golpear lo que amo, a defenderte:\\
nadie sino mi ser por tus secretos:\\
nadie sino mi voz por tus abiertas\\
hileras de rocío, por tus escalones\\
en donde la maternidad salobre\\
del mar te besa, nadie sino mis labios\\
en tu corona fría de sirena,\\
elevada en el aire de la altura,\\
oceánico amor, Valparaíso,\\
reina de todas las costas del mundo,\\
verdadera central de olas y barcos,\\
eres en mí como la luna o como\\
la dirección del aire en la arboleda.\\
Amo tus criminales callejones,\\
tu luna de puñal sobre los cerros,\\
y entre tus plazas la marinería\\
revistiendo de azul la primavera.\\
Que se entienda, te pido, puerto mío,\\
que yo tengo derecho\\
a escribirte lo bueno y lo malvado\\
y soy como las lámparas amargas\\
cuando iluminan las botellas rotas.  

\end{verse}

\clearpage
\poemtitle {Lautaro (1550) (1948)}
\begin{verse}

La sangre toca un corredor de cuarzo.\\
Así nace Lautaro de la tierra.\\
La piedra crece donde cae la gota.  

\end{verse}

\clearpage
\poemtitle {La educación del cacique (1948)}
\begin{verse}

Lautaro era una flecha delgada.\\
Elástico y azul fue nuestro padre.\\
Fue su primera edad solo silencio.\\
Su adolescencia fue dominio.\\
Su juventud fue un viento dirigido.\\
Se preparó como una larga lanza.\\
Acostumbró los pies en las cascadas.\\
Educó la cabeza en las espinas.\\
Ejecutó las pruebas del guanaco.\\
Vivió en las madrigueras de la nieve.\\
Acechó la comida de las águilas.\\
Arañó los secretos del peñasco.\\
Entretuvo los pétalos del fuego.\\
Se amamantó de primavera fría.\\
Se quemó en las gargantas infernales.\\
Fue cazador entre las aves crueles.\\
Se tiñeron sus manos de victorias.\\
Leyó las agresiones de la noche.\\
Sostuvo los derrumbes del azufre.  

Se hizo velocidad, luz repentina.  

Tomó las lentitudes del otoño.\\
Trabajó en las guaridas invisibles.\\
Durmió en las sábanas del ventisquero.\\
Igualó la conducta de las flechas.  

Bebió la sangre agreste en los caminos.\\
Arrebató el tesoro de las olas.\\
Se hizo amenaza como un dios sombrío.\\
Comió en cada cocina de su pueblo.\\
Aprendió el alfabeto del relámpago.\\
Olfateó las cenizas esparcidas.\\
Envolvió el corazón con pieles negras.  

Descifró el espiral hilo del humo.\\
Se construyó de fibras taciturnas.\\
Se aceitó como el alma de la oliva.\\
Se hizo cristal de transparencia dura.\\
Estudió para viento huracanado.\\
Se combatió hasta apagar la sangre.  

Solo entonces fue digno de su pueblo.  

\end{verse}

\clearpage
\poemtitle {El gran océano (1949)}
\begin{verse}

Si de tus dones y de tus destrucciones, Océano, a mis manos\\
pudiera destinar una medida, una fruta, un fermento,\\
escogería tu reposo distante, las líneas de tu acero,\\
tu extensión vigilada por el aire y la noche,\\
y la energía de tu idioma blanco\\
que destroza y derriba sus columnas\\
en su propia pureza demolida.  

No es la última ola con su salado peso\\
la que tritura costas y produce\\
la paz de arena que rodea el mundo:\\
es el central volumen de la fuerza,\\
la potencia extendida de las aguas,\\
la inmóvil soledad llena de vidas.\\
Tiempo, tal vez, o copa acumulada\\
de todo movimiento, unidad pura\\
que no selló la muerte, verde víscera\\
de la totalidad abrasadora.  

Del brazo sumergido que levanta una gota\\
no queda sino un beso de la sal. De los cuerpos\\
del hombre en tus orillas una húmeda fragancia\\
de flor mojada permanece. Tu energía\\
parece resbalar sin ser gastada,\\
parece regresar a su reposo.  

La ola que desprendes,\\
arco de identidad, pluma estrellada,\\
cuando se despeñó fue solo espuma,\\
y regresó a nacer sin consumirse.  

Toda tu fuerza vuelve a ser origen.\\
Solo entregas despojos triturados,\\
cáscaras que apartó tu cargamento,\\
lo que expulsó la acción de tu abundancia,\\
todo lo que dejó de ser racimo.  

Tu estatua está extendida más allá de las olas.  

Viviente y ordenada como el pecho y el manto\\
de un solo ser y sus respiraciones,\\
en la materia de la luz izadas,\\
llanuras levantadas por las olas,\\
forman la piel desnuda del planeta.\\
Llenas tu propio ser con tu substancia.  

Colmas la curvatura del silencio.  

Con tu sal y tu miel tiembla la copa,\\
la cavidad universal del agua,\\
y nada falta en ti como en el cráter\\
desollado, en el vaso cerril:\\
cumbres vacías, cicatrices, señales\\
que vigilan el aire mutilado.  

Tus pétalos palpitan contra el mundo,\\
tiemblan tus cereales submarinos,\\
las suaves ovas cuelgan su amenaza,\\
navegan y pululan las escuelas,\\
y solo sube al hilo de las redes\\
el relámpago muerto de la escama,\\
un milímetro herido en la distancia\\
de tus totalidades cristalinas.  

\end{verse}

\clearpage
\poemtitle {Los enigmas (1949)}
\begin{verse}

Me habéis preguntado qué hila el crustáceo entre sus patas de oro\\
y os respondo: El mar lo sabe.\\
Me decís qué espera la ascidia en su campana transparente? Qué espera?\\
Yo os digo, espera como vosotros el tiempo.\\
Me preguntáis a quién alcanza el abrazo del alga Macrocustis?\\
Indagadlo, indagadlo a cierta hora, en cierto mar que conozco.\\
Sin duda me preguntaréis por el marfil maldito del narwhal, para que yo os conteste\\
de qué modo el unicornio marino agoniza arponeado.\\
Me preguntáis tal vez por las plumas alcionarias que tiemblan\\
en los puros orígenes de la marea austral?\\
Y sobre la construcción cristalina del pólipo habéis barajado, sin duda,\\
una pregunta más, desgranándola ahora?\\
Queréis saber la eléctrica materia de las púas del fondo?\\
La armada estalactita que camina quebrándose?\\
El anzuelo del pez pescador, la música extendida\\
en la profundidad como un hilo en el agua?  

Yo os quiero decir que esto lo sabe el mar, que la vida en sus arcas\\
es ancha como la arena, innumerable y pura\\
y entre las uvas sanguinarias el tiempo ha pulido\\
la dureza de un pétalo, la luz de la medusa\\
y ha desgranado el ramo de sus hebras corales\\
desde una cornucopia de nácar infinito.  

Yo no soy sino la red vacía que adelanta\\
ojos humanos, muertos en aquellas tinieblas,\\
dedos acostumbrados al triángulo, medidas\\
de un tímido hemisferio de naranja.  

Anduve como vosotros escarbando\\
la estrella interminable,\\
y en mi red, en la noche, me desperté desnudo,  

\end{verse}

\clearpage
\poemtitle {La frontera (1949)}
\begin{verse}

Lo primero que vi fueron árboles, barrancas\\
decoradas con flores de salvaje hermosura,\\
húmedo territorio, bosques que se incendiaban\\
y el invierno detrás del mundo, desbordado.\\
Mi infancia son zapatos mojados, troncos rotos\\
caídos en la selva, devorados por lianas\\
y escarabajos, dulces días sobre la avena,\\
y la barba dorada de mi padre saliendo\\
hacia la majestad de los ferrocarriles.  

Frente a mi casa el agua austral cavaba\\
hondas derrotas, ciénagas de arcillas enlutadas,\\
que en el verano eran atmósfera amarilla\\
por donde las carretas crujían y lloraban\\
embarazadas con nueve meses de trigo.\\
Rápido sol del Sur:\\
rastrojos, humaredas\\
en caminos de tierras escarlatas, riberas\\
de ríos de redondo linaje, corrales y potreros\\
en que reverberaba la miel del mediodía.  

El mundo polvoriento entraba grado a grado\\
en los galpones, entre barricas y cordeles,\\
a bodegas cargadas con el resumen rojo\\
del avellano, todos los párpados del bosque.  

Me pareció ascender en el tórrido traje\\
del verano, con las máquinas trilladoras,\\
por las cuestas, en la tierra barnizada de boldos,\\
erguida entre los robles, indeleble,\\
pegándose en las ruedas como carne aplastada.  

Mi infancia recorrió las estaciones: entre\\
los rieles, los castillos de madera reciente,\\
la casa sin ciudad, apenas protegida\\
por reses y manzanos de perfume indecible\\
fui yo, delgado niño cuya pálida forma\\
se impregnaba de bosques vacíos y bodegas.  

\end{verse}

\clearpage
\poemtitle {Carta a Miguel Otero Silva. En Caracas (1949)}
\begin{verse}

Un amigo me trajo tu carta escrita\\
con palabras invisibles, sobre su traje, en sus ojos.\\
Qué alegre eres. Miguel, qué alegres somos!\\
Ya no queda en un mundo de úlceras estucadas\\
sino nosotros, indefinidamente alegres.\\
Veo pasar al cuervo y no me puede hacer daño.\\
Tú observas al escorpión y limpias tu guitarra.\\
Vivimos entre las fieras, cantando, y cuando tocamos\\
un hombre, la materia de alguien en quien creíamos,\\
y éste se desmorona como un pastel podrido,\\
tú en tu venezolano patrimonio recoges\\
lo que puede salvarse, mientras que yo defiendo\\
la brasa de la vida.\\
Qué alegría, Miguel!\\
Tú me preguntarás dónde estoy? Te contaré\\
--dando sólo detalles útiles al Gobierno--\\
que en esta costa llena de piedras salvajes\\
se unen el mar y el campo, olas y pinos,\\
águilas y petreles, espumas y praderas.\\
Has visto desde muy cerca y todo el día\\
cómo vuelan los pájaros del mar? Parece\\
que llevaran las cartas del mundo a sus destinos.\\
Pasan los alcatraces como barcos del viento,\\
otras aves que vuelan como flechas y traen\\
los mensajes de reyes difuntos, de los príncipes\\
enterrados con hilos de turquesa en las costas andinas\\
y las gaviotas hechas de blancura redonda,\\
que olvidan continuamente sus mensajes.\\
Qué azul es la vida, Miguel, cuando hemos puesto en ella\\
amor y lucha, palabras que son el pan y el vino,\\
palabras que ellos no pueden deshonrar todavía,\\
porque nosotros salimos a la calle con escopeta y cantos.\\
Están perdidos con nosotros, Miguel.\\
Qué pueden hacer sino matarnos y aun así\\
les resulta un mal negocio, sólo pueden\\
tratar de alquilar un piso frente a nosotros y seguirnos\\
para aprender a reír y a llorar como nosotros.\\
Cuando yo escribía versos de amor, que me brotaban\\
por todas partes, y me moría de tristeza,\\
errante, abandonado, royendo el alfabeto,\\
me decían: «Qué grande eres, oh Teócrito!»\\
Yo no soy Teócrito: tomé a la vida,\\
me puse frente a ella, la besé hasta vencerla,\\
y luego me fui por los callejones de las minas\\
a ver cómo vivían oros hombres.\\
Y cuando salí con las manos teñidas de basura y dolores,\\
las levanté mostrándolas en las cuerdas de oro,\\
y dije: «Yo no comparto el crimen.»\\
Tosieron, se disgustaron mucho, me quitaron el saludo,\\
me dejaron de llamar Teócrito, y terminaron\\
por insultarme y mandar toda la policía a encarcelarme,\\
porque no seguía preocupado exclusivamente de asuntos metafísicos.\\
Pero yo había conquistado la alegría.\\
Desde entonces me levanté leyendo las cartas\\
que traen las aves del mar desde tan lejos,\\
cartas que vienen mojadas, mensajes que poco a poco\\
voy traduciendo con lentitud y seguridad: soy meticuloso\\
como un ingeniero en este extraño oficio.\\
Y salgo de repente a la ventana. Es un cuadrado\\
de transparencia, es pura la distancia\\
de hierbas y peñascos, y así voy trabajando\\
entre las cosas que amo: olas, piedras, avispas,\\
con una embriagadora felicidad marina.\\
Pero a nadie le gusta que estemos alegres, a ti te asignaron\\
un papel bonachón: «Pero no exagere, no se preocupe»,\\
y a mí me quisieron clavar en un insectario, entre las lágrimas,\\
para que éstas me ahogaran y ellos pudieran decir sus discursos en mi tumba.  

Yo recuerdo un día en la pampa arenosa\\
del salitre, había quinientos hombres\\
en huelga. Era la tarde abrasadora\\
de Tarapacá. Y cuando los rostros habían recogido\\
toda la arena y el desangrado sol seco del desierto,\\
yo vi llegar a mi corazón, como una copa que odio,\\
la vieja melancolía. Aquella hora de crisis,\\
en la desolación de los salares, en ese minuto débil de\\
la lucha, en que podríamos haber sido vencidos,\\
una niña pequeñita y pálida venida de las minas\\
dijo con una voz valiente en que se juntaban el cristal y el acero\\
un poema tuyo, un viejo poema tuyo, que rueda entre los ojos arrugados\\
de todos los obreros y labradores de mi patria, de América.\\
Y aquel trozo de canto tuyo refulgió de repente\\
en mi boca como una flor purpúrea\\
y bajó hacia mi sangre, llenándola de nuevo\\
con una alegría desbordante nacida de tu canto.\\
Y yo pensé no sólo en ti, sino en tu Venezuela amarga.\\
Hace años, vi un estudiante que tenía en los tobillos\\
la señal de las cadenas que un general le había impuesto,\\
y me contó cómo los encadenados trabajaban en los caminos\\
y los calabozos donde la gente se perdía. Porque así ha sido nuestra América:\\
una llanura con ríos devorantes y constelaciones\\
de mariposas (en algunos sitios, las esmeraldas son espesas como manzanas),\\
pero siempre a lo largo de la noche y de los ríos\\
hay tobillos que sangran, antes cerca del petróleo,\\
hoy cerca del nitrato, en Pisagua, donde un déspota sucio\\
ha enterrado la flor de mi patria para que muera, y él pueda comerciar con los huesos.\\
Por eso cantas, por eso, para que América deshonrada y herida\\
haga temblar sus mariposas y recoja sus esmeraldas\\
sin la espantosa sangre del castigo, coagulada\\
en las manos de los verdugos y de los mercaderes.\\
Yo comprendí qué alegre estarías, cerca del Orinoco, cantando,\\
seguramente, o bien comprando vino para tu casa,\\
ocupando tu puesto en la lucha y en la alegría,\\
ancho de hombros, como son los poetas de este tiempo,\\
--con trajes claros y zapatos de camino--.\\
Desde entonces, he ido pensando que alguna vez te escribiría,\\
y cuando el amigo llegó, todo lleno de historias tuyas\\
que se te desprendían de todo el traje\\
y que bajo los castaños de mi casa se derramaron,\\
me dijo: «Ahora», y tampoco comencé a escribirte.\\
Pero hoy ha sido demasiado: pasó por mi ventana\\
no sólo un ave del mar, sino millares,\\
y recogí las cartas que nadie lee y que ellas llevan\\
por las orillas del mundo, hasta perderlas.\\
Y entonces, en cada una leía palabras tuyas\\
y eran como las que yo escribo y sueño y canto,\\
y entonces decidí enviarte esta carta, que termino aquí\\
para mirar por la ventana el mundo que nos pertenece.  
\end{verse}
\end{document}
