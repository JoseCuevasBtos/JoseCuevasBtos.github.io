% rubber: module xelatex
\documentclass[12pt]{article}

\usepackage[spanish]{babel}
\usepackage[utf8]{inputenc}
% \usepackage{palatino}
\usepackage{verse}

\usepackage{fontspec}
\setmainfont{EB Garamond}

\date{}
\title{Poesía selecta de Ruben Darío I}
\begin{document}
\maketitle
\tableofcontents
\clearpage
\poemtitle {Era un aire suave}
\begin{verse}

Era un aire suave, de pausados giros;\\
El hada Harmonía ritmaba sus vuelos;\\
E iban frases vagas y tenues suspiros\\
Entre los sollozos de los violoncelos.  

Sobre la terraza, junto a los ramajes,\\
Diríase un trémolo de liras eolias\\
Cuando acariciaban los sedosos trajes,\\
Sobre el tallo erguidas, las blancas magnolias.  

La marquesa Eulalia risas y desvíos\\
Daba a un tiempo mismo para dos rivales:\\
El vizconde rubio de los desafíos\\
Y el abate joven de los madrigales.  

Cerca, coronado con hojas de viña,\\
Reía en su máscara Término barbudo,\\
Y, como un efebo que fuese una niña,\\
Mostraba una Diana su mármol desnudo.  

Y bajo un boscaje del amor palestra,\\
Sobre rico zócalo al modo de Jonia,\\
Con un candelabro prendido en la diestra\\
Volaba el Mercurio de Juan de Bolonia.  

La orquesta perlaba sus mágicas notas;\\
Un coro de sones alados se oía;\\
Galantes pavanas fugaces gavotas,\\
Cantaban los dulces violines de Hungría.  

Al oír las quejas de sus caballeros\\
Ríe, ríe, ríe, la divina Eulalia,\\
Pues son su tesoro las flechas de Eros,\\
El cinto de Cipria, la rueca de Onfalia.  

¡Ay de quien sus mieles y frases recoja!\\
¡Ay de quien del canto de su amor se fíe!\\
Con sus ojos lindos y su boca roja,\\
La divina Eulalia, ríe, ríe, ríe.  

Tiene azules ojos, es maligna y bella;\\
Cuando mira vierte viva luz extraña;\\
Se asoma a sus húmedas pupilas de estrella\\
El alma del rubio cristal de Champaña.  

Es noche de fiesta, y el baile de trajes\\
Ostenta su gloria de triunfos mundanos.\\
La divina Eulalia, vestida de encajes,\\
Una flor destroza con sus tersas manos.  

El teclado armónico de su risa fina\\
A la alegre música de un pájaro iguala.\\
Con los staccati de una bailarina\\
Y las locas fugas de una colegiala.  

¡Amoroso pájaro que trinos exhala\\
Bajo el ala a veces ocultando el pico;\\
Que desdenes rudos lanza bajo el ala,\\
Bajo el ala aleve del leve abanico!  

Cuando a media noche sus notas arranque\\
Y en arpegios áureos gima Filomela,\\
Y el ebúrneo cisne, sobre el quieto estanque,\\
Como blanca góndola imprima su estela,  

La marquesa alegre llegará al boscaje,\\
Boscaje que cubre la amable glorieta\\
Donde han de estrecharla los brazos de un paje,\\
que siendo su paje será su poeta.  

Al compás de un canto de artista de Italia\\
Que en la brisa errante la orquesta deslíe,\\
Junto a los rivales, la divina Eulalia,\\
La divina Eulalia, ríe, ríe, ríe.  

¿Fue acaso en el tiempo del rey Luis de Francia,\\
Sol con corte de astros, en campos de azur,\\
Cuando los alcázares llenó de fragancia\\
La regia y pomposa rosa Pompadour?  

¿Fue cuando la bella su falda cogía\\
Con dedos de ninfa, bailando el minué,\\
Y de los compases el ritmo seguía\\
Sobre el tacón rojo, lindo y leve el pie?  

¿O cuando pastoras de floridos valles\\
Ornaban con cintas sus albos corderos,\\
Y oían, divinas Tirsis de Versalles,\\
Las declaraciones de sus caballeros?  

¿Fue en ese buen tiempo de duques pastores,\\
De amantes princesas y tiernos galanes,\\
Cuando entre sonrisas y perlas y flores\\
Iban las casacas de los chambelanes?  

¿Fue acaso en el Norte o en el Mediodía?\\
Yo el tiempo y el día y el país ignoro;\\
Pero sé que Eulalia ríe todavía,\\
¡Y es cruel y eterna su risa de oro!  

\end{verse}

\clearpage
\poemtitle {Divagación}
\begin{verse}

¿Vienes? me llega aquí, pues que suspiras,\\
Un soplo de las mágicas fragancias\\
Que hicieran los delirios de las liras\\
En las Grecias, las Romas y las Francias.  

¡Suspira así! Revuelen las abejas\\
Al olor de la olímpica ambrosía,\\
En los perfumes que en el aire dejas;\\
Y el dios de piedra se despierte y ría.  

Y el dios de piedra se despierte y cante\\
La gloria de los tirsos florecientes\\
En el gesto ritual de la bacante\\
De rojos labios y nevados dientes;  

En el gesto ritual que en las hermosas\\
Ninfalias guía a la divina hoguera,\\
Hoguera que hace llamear las rosas\\
En las manchadas pieles de pantera.  

Y pues amas reír, ríe, y la brisa\\
Lleve el son de los líricos cristales\\
De tu reír, y haga temblar la risa\\
La barba de los Términos joviales.  

Mira hacia el lado del boscaje, mira\\
Blanquear el muslo de marfil de Diana,\\
Y después de la Virgen, la Hetaira\\
Diosa, su blanca, rosa y rubia hermana.  

Pasa en busca de Adonis; sus aromas\\
Deleitan a las rosas y los nardos;\\
Síguela una pareja de palomas\\
Y hay tras ella una fuga de leopardos.  

¿Te gusta amar en griego? Yo las fiestas\\
Galantes busco, en donde se recuerde,\\
Al suave son de rítmicas orquestas,\\
La tierra de la luz y el mirlo verde.  

(Los abates refieren aventuras\\
A las rubias marquesas. Soñolientos\\
Filósofos defienden las ternuras\\
Del amor, con sutiles argumentos,  

Mientras que surge de la verde grama,\\
En la mano el acanto de Corinto,\\
Una ninfa a quien puso un epigrama\\
Beaumarchais, sobre el mármol de su plinto.  

Amo más que la Grecia de los griegos\\
La Grecia de la Francia, porque en Francia,\\
Al eco de las Risas y los Juegos,\\
Su más dulce licor Venus escancia.  

Demuestran más encantos y perfidias\\
Coronadas de flores y desnudas,\\
Las diosas de Clodión que las de Fidias;\\
Unas cantan francés, otras son mudas.  

Verlaine es más que Sócrates; y Arsenio\\
Houssaye supera al viejo Anacreonte.\\
En París reinan el Amor y el Genio.\\
Ha perdido su imperio el dios bifronte.  

Monsieur Prudhomme y Homais no saben nada.\\
Hay Chipres, Pafos, Tempes y Amatuntes,\\
Donde el amor de mi madrina, un hada,\\
Tus frescos labios a los míos juntes).  

Sones de bandolín. El rojo vino\\
Conduce un paje rojo. ¿Amas los sones\\
Del bandolín, y un amor florentino?\\
Serás la reina en los decamerones.  

(Un coro de poetas y pintores\\
Cuenta historias picantes. Con maligna\\
Sonrisa alegre aprueban los señores.\\
Clelia enrojece, una dueña se signa).  

¿O un amor alemán? —que no han sentido\\
Jamás los alemanes—: la celeste\\
Gretchen; claro de luna; el aria; el nido\\
Del ruiseñor; y en una roca agreste,  

La luz de nieve que del cielo llega\\
Y baña a una hermosura que suspira\\
La queja vaga que a la noche entrega\\
Loreley en la lengua de la lira.  

Y sobre el agua azul el caballero\\
Lohengrín; y su cisne, cual si fuese\\
Un cincelado témpano viajero,\\
Con su cuello enarcado en forma de S.  

Y del divino Enrique Heine un canto,\\
A la orilla del Rhin; y del divino\\
Wolfgang la larga cabellera, el manto;\\
Y de la uva teutona el blanco vino.  

O amor lleno de sol, amor de España,\\
Amor lleno de púrpuras y oros;\\
Amor que da el clavel, la flor extraña\\
Regada con la sangre de los toros;  

Flor de gitanas, flor que amor recela\\
Amor de sangre y luz, pasiones locas;\\
Flor que trasciende a clavo y a canela,\\
Roja cual las heridas y las bocas.  

¿Los amores exóticos acaso…?\\
Como rosa de Oriente me fascinas:\\
Me deleitan la seda, el oro, el raso.\\
Gautier adoraba a las princesas chinas.  

¡Oh, bello amor de mil genuflexiones;\\
Torres de kaolín, pies imposibles,\\
Tazas de té, tortugas y dragones,\\
Y verdes arrozales apacibles!  

Ámame en chino, en el sonoro chino\\
De Li-Tai-Pe. Yo igualaré a los sabios\\
Poetas que interpretan el destino;\\
Madrigalizaré junto a tus labios.  

Diré que eres más bella que la luna;\\
Que el tesoro del cielo es menos rico\\
Que el tesoro que vela la importuna\\
Caricia de Marfil de tu abanico.  

Ámame, japonesa, japonesa\\
Antigua, que no sepa de naciones\\
Occidentales; tal una princesa\\
Con las pupilas llenas de visiones,  

Que aún ignorase en la sagrada Kioto,\\
En su labrado camarín de plata,\\
Ornado al par de crisantemo y loto,\\
La civilización de Yamagata.  

O con amor hindú que alza sus llamas\\
En la visión suprema de los mitos,\\
Y hace temblar en misteriosas bramas\\
La iniciación de los sagrados ritos,  

En tanto mueven tigres y panteras\\
Sus hierros, y en los fuertes elefantes\\
Sueñan con ideales bayaderas\\
Los rajahs, constelados de brillantes.  

O negra, negra como la que canta\\
En su Jerusalem el rey hermoso,\\
Negra que haga brotar bajo su planta\\
La rosa y la cicuta del reposo…  

Amor, en fin, que todo diga y cante,\\
Amor que encante y deje sorprendida\\
A la serpiente de ojos de diamante\\
Que está enroscada al árbol de la vida.  

Ámame así, fatal cosmopolita,\\
Universal, inmensa, única, sola\\
y todas; misteriosa y erudita:\\
Ámame mar y nube, espuma y ola.  

Sé mi reina de Saba, mi tesoro;\\
Descansa en mis palacios solitarios.\\
Duerme. Yo encenderé los incensarios.\\
Y junto a mi unicornio cuerno de oro,\\
Tendrán rosas y miel tus dromedarios.  

\end{verse}

\clearpage
\poemtitle {Sonatina}
\begin{verse}

La princesa está triste\ldots{} ¿qué tendrá la princesa?\\
Los suspiros se escapan de su boca de fresa,\\
Que ha perdido la risa, que ha perdido el color.\\
La princesa está pálida en su silla de oro,\\
Está mudo el teclado de su clave sonoro;\\
Y en un vaso olvidada se desmaya una flor.  

El jardín puebla el triunfo de los pavos reales.\\
Parlanchina, la dueña dice cosas banales,\\
Y vestido de rojo piruetea el bufón.\\
La princesa no ríe, la princesa no siente;\\
La princesa persigue por el cielo de Oriente\\
La libélula vaga de una vaga ilusión.  

¿Piensa acaso en el príncipe de Golconda o de China,\\
O en el que ha detenido su carroza argentina\\
Para ver de sus ojos la dulzura de luz,\\
O en el rey de las Islas de las rosas fragantes,\\
O en el que es soberano de los claros diamantes,\\
O en el dueño orgulloso de las perlas de Ormuz?  

¡Ay! la pobre princesa de la boca de rosa,\\
Quiere ser golondrina, quiere ser mariposa.\\
Tener alas ligeras, bajo el cielo volar,\\
Ir al sol por la escala luminosa de un rayo,\\
Saludar a los lirios con los versos de Mayo,\\
O perderse en el viento sobre el trueno del mar.  

Ya no quiere el palacio, ni la rueca de plata,\\
Ni el halcón encantado, ni el bufón escarlata,\\
Ni los cisnes unánimes en el lago de azur.\\
Y están tristes las flores por la flor de la corte;\\
Los jazmines de Oriente, los nelumbos del Norte,\\
De Occidente las dalias y las rosas del Sur.  

¡Pobrecita princesa de los ojos azules!\\
Está presa en sus oros, está presa en sus tules,\\
En la jaula de mármol del palacio real;\\
El palacio soberbio que vigilan los guardas,\\
Que custodian cien negros con sus cien alabardas,\\
Un lebrel que no duerme y un dragón colosal.  

¡Oh, quién fuera hipsipila que dejó la crisálida!\\
(La princesa está triste. La princesa está pálida)\\
¡Oh, visión adorada de oro, rosa y marfil!\\
¡Quien volara a la tierra donde un príncipe existe\\
(La princesa está pálida. La princesa está triste)\\
Más brillante que el alba, más hermoso que Abril!  

--¡Calla, calla, princesa --dice el hada madrina--,\\
En caballo con alas, hacia acá se encamina,\\
En el cinto la espada y en la mano el azor,\\
El feliz caballero que te adora sin verte,\\
Y que llega de lejos, vencedor de la Muerte,\\
A encenderte los labios con su beso de amor!  

\end{verse}

\clearpage
\poemtitle {Bouquet}
\begin{verse}

Un poeta egregio del país de Francia,\\
Que con versos áureos alabó el amor,\\
Formó un ramo armónico, lleno de elegancia,\\
En su \emph{Sinfonía en Blanco Mayor}.  

Yo por ti formara, Blanca deliciosa,\\
El regalo lírico de un blanco bouquet,\\
Con la blanca estrella, con la blanca rosa\\
Que en los bellos parques del azul se ve.  

Hoy que tú celebras tus bodas de nieve,\\
(Tus bodas de virgen con el sueño son)\\
Todas sus blancuras, Primavera, llueve\\
Sobre la blancura de tu corazón.  

Cirios, cirios blancos, blancos, blancos lirios.\\
Cuellos de los cisnes, margarita en flor,\\
Galas de la espuma, ceras de los cirios\\
Y estrellas celestes tienen tu color.  

Yo al enviarte versos de mi vida arranco\\
La flor que te ofrezco, blanco serafín:\\
¡Mira cómo mancha tu corpiño blanco\\
La más roja rosa que hay en mi jardín!  

\end{verse}

\clearpage
\poemtitle {El faisán}
\begin{verse}

Dijo sus secretos el faisán de oro:\\
En el gabinete mi blanco tesoro,\\
De sus claras risas el divino coro  

Las bellas figuras de los gobelinos,\\
Los cristales llenos de aromados vinos,\\
Las rosas francesas en los vasos chinos.  

(Las rosas francesas, porque fue allá en Francia\\
Donde en el retiro de la dulce estancia\\
Esas frescas rosas dieron su fragancia).  

La cena esperaba. Quitadas las vendas,\\
Iban mil amores de flechas tremendas\\
En aquella noche de Carnestolendas.  

La careta negra se quitó la niña,\\
y tras el preludio de una alegre riña\\
Apuró mi boca vino de su viña  

Vino de la viña de la boca loca,\\
Que hace arder el beso, que el mordisco invoca.\\
¡Oh los blancos dientes de la loca boca!  

En su boca ardiente yo bebí los vinos,\\
y pinzas rosadas, sus dedos divinos,\\
Me dieron las fresas y los langostinos.  

Yo la vestimenta de Pierrot tenía,\\
Y aunque me alegraba y aunque me reía,\\
Moraba en mi alma la melancolía.  

La carnavalesca noche luminosa\\
Dio a mi triste espíritu la mujer hermosa,\\
Sus ojos de fuego, sus labios de rosa.  

Y en el gabinete del café galante\\
Ella se encontraba con su nuevo amante,\\
Peregrino pálido de un país distante.  

Llegaban los ecos de vagos cantares;\\
Y se despedían de sus azahares\\
Miles de purezas en los bulevares.  

Y cuando el champaña me cantó su canto,\\
Por una ventana vi que un negro manto\\
De nube, de Febo cubría el encanto.  

Y dije a la amada de un día: —¿No viste\\
De pronto ponerse la noche tan triste?\\
¿Acaso la Reina de luz ya no existe?  

Ella me miraba. Y el faisán cubierto de plumas de oro:\\
--«¡Pierrot! ¡ten por cierto\\
Que tu fiel amada, que la Luna, ha muerto!»  

\end{verse}

\clearpage
\poemtitle {Margarita}
\begin{verse}

¿Recuerdas que querías ser una Margarita\\
Gautier? Fijo en mi mente tu extraño rostro está,\\
Cuando cenamos juntos, en la primera cita,\\
En una noche alegre que nunca volverá.  

Tus labios escarlatas de púrpura maldita\\
Sorbían el champaña del fino baccarat;\\
Tus dedos deshojaban la blanca margarita\\
«¡Sí\ldots{} no\ldots{} sí\ldots{} no\ldots{}» y sabías que te adoraba ya!  

Después, ¡oh flor de Histeria! llorabas y reías;\\
Tus besos y tus lágrimas tuve en mi boca yo;\\
Tus risas, tus fragancias, tus quejas, eran mías.  

Y en una tarde triste de los más dulces días,\\
La Muerte, la celosa, por ver si me querías,\\
¡Como a una margarita de amor, te deshojó!  

\end{verse}

\clearpage
\poemtitle {Dice Mía}
\begin{verse}

Mi pobre alma pálida\\
Era un crisálida.\\
Luego mariposa\\
De color de rosa.  

Un céfiro inquieto\\
Dijo mi secreto…\\
--¿Has sabido tu secreto un día?  

¡Oh Mía!\\
Tu secreto es una\\
Melodía en un rayo de luna\ldots{}\\
--¿Una melodía?  

\end{verse}

\clearpage
\poemtitle {Ite, missa est}
\begin{verse}

Yo adoro a una sonámbula con alma de Eloísa\\
Virgen como la nieve y honda como la mar;\\
Su espíritu es la hostia de mi amorosa misa\\
Y alzo al son de una dulce lira crepuscular.  

Ojos de evocadora, gesto de profetisa,\\
En ella hay la sagrada frecuencia del altar;\\
Su risa es la sonrisa suave de Monna Lisa.\\
Sus labios son los únicos labios para besar.  

Y he de besarla un día con rojo beso ardiente;\\
Apoyada en mi brazo como convaleciente\\
Me mirará asombrada con íntimo pavor;  

La enamorada esfinge quedará estupefacta,\\
Apagaré la llama de la vestal intacta\\
¡Y la faunesa antigua me rugirá de amor!  

\end{verse}

\clearpage
\poemtitle {Pórtico}
\begin{verse}

Libre la frente que el casco rehúsa,\\
Casi desnuda en la gloria del día,\\
Alza su tirso de rosas la musa\\
Bajo el gran sol de la eterna Harmonía.  

Es Floreal, eres tú, Primavera,\\
Quien la sandalia calzó a su pie breve;\\
Ella, de tristes nostalgias muriera\\
En el país de los cisnes de nieve.  

Griega es su sangre, su abuelo era ciego;\\
Sobre la cumbre del Pindo sonoro\\
El sagitario del carro de fuego\\
Puso en su lira las cuerdas de oro.  

Y bajo el pórtico blanco de Paros,\\
Y en los boscajes de frescos laureles,\\
Píndaro dióle sus ritmos preclaros,\\
Dióle Anacreonte sus vinos y mieles.  

Toda desnuda, en los claros diamantes\\
Que en la Castalia recaman las linfas,\\
Viéronla tropas de faunos saltantes,\\
Cual la más fresca y gentil de las ninfas.  

Y en la fragante, harmoniosa floresta,\\
Puesto a los ecos su oído de musa,\\
Pan sorprendióla escuchando la orquesta\\
Que él daba al viento con su cornamusa.  

Ella resurge después en el Lacio,\\
Siendo del tedio su lengua exterminio;\\
Lleva a sus labios la copa de Horacio,\\
Bebe falerno en su ebúrneo triclinio.  

Pájaro errante, ideal golondrina,\\
Vuela de Arabia a un confín solitario,\\
Y ve pasar en su torre argentina\\
A un rey de Oriente sobre un dromedario;  

Rey misterioso, magnífico y mago.\\
Dueño opulento de cien Estambules,\\
Y a quien un genio brindara en un lago\\
Góndolas de oro en las aguas azules.  

Ese es el rey más hermoso que el día,\\
Que abre a la musa las puertas de Oriente;\\
Ese es el rey del país Fantasía,\\
Que lleva un claro lucero en la frente.  

Es en Oriente donde ella se inspira\\
En las moriscas exóticas zambras;\\
Donde primero contempla y admira\\
Las cinceladas divinas alhambras;  

Las muelles danzas en las alcatifas\\
Donde la mora sus velos desata,\\
Los pensativos y viejos kalifas\\
De ojos obscuros y barbas de plata.  

Es una bella y alegre mañana\\
Cuando su vuelo la musa confía\\
A una errabunda y fugaz caravana\\
Que hace del viento su brújula y guía.  

Era la errante familia bohemia,\\
Sabia en extraños conjuros y estigmas,\\
Que une en su boca plegaria y blasfemia,\\
Nombres sonoros y raros enigmas;  

Que ama los largos y negros cabellos,\\
Danzas lascivas y finos puñales,\\
Ojos llameantes de vivos destellos,\\
Flores sangrientas de labios carnales.  

Y con la gente morena y huraña\\
Que a los caprichos del aire se entrega,\\
Hace su entrada triunfal en España\\
Fresca y riente la rítmica griega.  

Mira las cumbres de Sierra Nevada,\\
Las bocas rojas de Málaga, lindas,\\
Y en un pandero su mano rosada\\
Fresas recoge, claveles y guindas.  

Canta y resuena su verso de oro,\\
Ve de Sevilla las hembras de llama,\\
Sueña y habita en la Alhambra del moro;\\
Y en sus cabellos perfumes derrama.  

Busca del pueblo las penas, la flores,\\
Mantos bordados de alhajas de seda,\\
Y la guitarra que sabe de amores,\\
Cálida y triste querida de Rueda;  

(Urna amorosa de voz femenina,\\
Caja de música de duelo y placer:\\
Tiene el acento de un alma divina,\\
Talle y caderas como una mujer).  

Va del tablado flamenco a la orilla\\
Y ase en sus palmas los crótalos negros,\\
Mientras derrocha la audaz seguidilla\\
Bruscos acordes y raudos alegros.  

Ritma los pasos, modula los sones,\\
Ebria risueña de un vino de luz,\\
Hace que brillen los ojos gachones,\\
Negros diamantes del patio andaluz.  

Campo y pleno aire refrescan sus alas;\\
Ama los nidos, las cumbres, las cimas;\\
Vuelve del campo vestida de galas,\\
Cuelga a su cuello collares de rimas.  

En su tesoro de reina de Saba,\\
Guarda en secreto celestes emblemas;\\
Flechas de fuego en su mágica aljaba,\\
Perlas, rubíes, zafiros y gemas.  

Tiene una corte pomposa de majas.\\
Suya es la chula de rostro risueño,\\
Suyas las juergas, las curvas navajas\\
Ebrias de sangre y licor malagueño.  

Tiene por templo un alcázar marmóreo,\\
Guárdalo esfinge de rostro egipciaco,\\
Y cual labrada en un bloque hiperbóreo,\\
Venus enfrente de un triunfo de Baco.  

Dentro presenta sus formas de nieve,\\
Brinda su amable sonrisa de piedra,\\
Mientras se enlaza en un bajo-relieve\\
A una driada ceñida de hiedra.  

Un joven fauno robusto y violento,\\
Dulce terror de las ninfas incautas,\\
Al son triunfante que lanzan al viento\\
Tímpanos, liras y sistros y flautas.  

Ornan los muros mosaicos y frescos,\\
Áureos pedazos de un sol fragmentario,\\
Iris trenzados en mil arabescos,\\
Joyas de un hábil cincel lapidario.  

Y de la eterna Belleza en el ara,\\
Ante su sacra y grandiosa escultura,\\
Hay una lámpara en albo carrara.\\
De una eucarística y casta blancura.  

Fuera, el frondoso jardín del poeta\\
Ríe en su fresca y gentil hermosura;\\
Ágata, perla, amatista, violeta,\\
Verdor eclógico y tibia espesura.  

Una andaluza despliega su manto\\
Para el poeta de música eximia;\\
Rústicos Títiros cantan su canto;\\
Bulle el hervor de la alegre vendimia.  

Ya es un tropel de bacantes modernas\\
El que despierta las locas lujurias;\\
Ya húmeda y triste de lágrimas tiernas,\\
Da su gemido la gaita de Asturias.  

Francas fanfarrias de cobres sonoros,\\
Labios quemantes de humanas sirenas,\\
Ocres y rojos de plazas de toros,\\
Fuegos y chispas de locas verbenas.  

Joven homérida, un día su tierra\\
Vióle que alzaba soberbio estandarte,\\
Buen capitán de la lírica guerra,\\
Regio cruzado del reino del arte.  

Vióle con yelmo de acero brillante,\\
Rica armadura sonora a su paso,\\
Firme tizona, broncíneo olifante,\\
Listo y piafante su excelso pegaso.  

Y de la brega tornar vióle un día\\
De su victoria en los bravos tropeles,\\
Bajo el gran sol de la eterna Harmonía,\\
Dueño de verdes y nobles laureles.  

Fue aborrecido de Zoilo, el verdugo.\\
Fue por la gloria su estrella encendida.\\
Y esto pasó en el reinado de Hugo,\\
Emperador de la barba florida.  

\end{verse}

\clearpage
\poemtitle {El cisne}
\begin{verse}

Fué en una hora divina para el género humano.\\
El Cisne antes cantaba sólo para morir.\\
Cuando se oyó el acento del Cisne wagneriano\\
Fue en medio de una aurora, fue para revivir.  

Sobre las tempestades del humano oceano\\
Se oye el canto del Cisne; no se cesa de oír,\\
Dominando el martillo del viejo Thor germano\\
O las trompas que cantan la espada de Argantir.  

¡Oh Cisne! ¡Oh sacro pájaro! Si antes la blanca Helena\\
Del huevo azul de Leda brotó de gracia llena,\\
Siendo de la Hermosura la princesa inmortal,  

Bajo tus blancas alas la nueva Poesía,\\
Concibe en una gloria de luz y de harmonía\\
La Helena eterna y pura que encarna el ideal.  

\end{verse}

\clearpage
\poemtitle {Sinfonía en Gris mayor}
\begin{verse}

El mar como un vasto cristal azogado\\
Refleja la lámina de un cielo de zinc;\\
Lejanas bandadas de pájaros marchan\\
El fondo bruñido de pálido gris.  

El sol como un vidrio redondo y opaco\\
Con paso de enfermo camina al cenit;\\
El viento marino descansa en la sombra\\
Teniendo de almohada su negro clarín.  

Las ondas que mueven su vientre de plomo\\
Debajo del muelle parecen gemir.\\
Sentado en un cable, fumando su pipa,\\
Está un marinero pensando en las playas\\
De un vago, lejano, brumoso país.  

Es viejo ese lobo. Tostaron su cara\\
Los rayos de fuego del sol del Brasil;\\
Los recios tifones del mar de la China\\
Le han visto bebiendo su frasco de gin.  

La espuma impregnada de yodo y salitre\\
Ha tiempo conoce su roja nariz,\\
Sus crespos cabellos, sus bíceps de atleta,\\
Su gorra de lona, su blusa de dril.  

En medio del humo que forma el tabaco\\
Ve el viejo el lejano, brumoso país,\\
A donde una tarde caliente y dorada\\
Tendidas las velas partió el bergantín…  

La siesta del trópico. El lobo se aduerme.\\
Ya todo lo envuelve la gama del gris.\\
Parece que un suave y enorme esfumino\\
Del curvo horizonte borrara el confín.  

La siesta del trópico. La vieja cigarra\\
Ensaya su ronca guitarra senil,\\
Y el grillo preludia un solo monótono\\
En la única cuerda que está en su violín.  

\end{verse}

\clearpage
\poemtitle {Epitalamio bárbaro}
\begin{verse}

El alba aun no aparece en su gloria de oro.\\
Canta el mar con la música de sus ninfas en coro\\
y el aliento del campo se va cuajando en bruma.\\
Teje la náyade el encaje de su espuma\\
Y el bosque inicia el himno de sus flautas de pluma.  

Es el momento en que el salvaje caballero\\
Se ve pasar. La tribu aúlla y el ligero\\
Caballo es un relámpago, veloz como una idea.\\
A su paso, asustada, se para la marea;\\
La náyade interrumpe la labor que ejecuta\\
Y el director del bosque detiene la batuta.  

--«¿Qué pasa?» desde el lecho pregunta Venus bella,\\
Y Apolo:\\
--«Es Sagitario que ha robado una estrella».  

\end{verse}

\clearpage
\poemtitle {Canto de la sangre}
\begin{verse}

Sangre de Abel. Clarín de las batallas.\\
Luchas fraternales; estruendos, horrores;\\
Flotan las banderas, hieren las metrallas,\\
Y visten la púrpura los emperadores.  

Sangre del Cristo. El órgano sonoro.\\
La viña celeste da el celeste vino;\\
Y en el labio sacro del cáliz de oro\\
Las almas se abrevan del vino divino.  

Sangre de los martirios. El salterio.\\
Hogueras; leones, palmas vencedoras;\\
Los heraldos rojos con que del misterio\\
Vienen precedidas las grandes auroras.  

Sangre que vierte el cazador. El cuerno.\\
Furias escarlatas y rojos destinos\\
Forjan en las fraguas del obscuro Infierno\\
Las fatales armas de los asesinos.  

¡Oh sangre de las vírgenes! La lira.\\
Encanto de abejas y de mariposas.\\
La estrella de Venus desde el cielo mira\\
El purpúreo triunfo de las reinas rosas.  

Sangre que la Ley vierte.\\
Tambor a la sordina.\\
Brotan las adelfas que riega la Muerte\\
y el rojo cometa que anuncia la ruina.  

Sangre de los suicidas. Organillo.\\
Fanfarrias macabras, responsos corales,\\
Con que de Saturno celébrase el brillo\\
En los manicomios y en los hospitales.  

\end{verse}

\clearpage
\poemtitle {II -- Palimpsesto}
\begin{verse}

\emph{Escrita en viejo dialecto eolio\\
Hallé esta página dentro un infolio\\
Y entre los libros de un monasterio\\
Del venerable San Agustín,\\
Un fraile acaso puso el escolio\\
Que allí se encuentra; dómine serio\\
De flacas manos y buen latín.\\
Hay sus lagunas.}  

\ldots{}Cuando los toros\\
De las campañas, bajo los oros\\
Que vierte el hijo de Hiperión,\\
Pasan mugiendo, y en las eternas\\
Rocas salvajes de las cavernas\\
Esperezándose ruge el león;  

Cuando en las vírgenes y verdes parras\\
Sus secas notas dan las cigarras,\\
Y en los panales de Himeto deja\\
Su rubia carga la leve abeja\\
Que en bocas rojas chupa la miel,\\
Junto a los mirtos, bajo los lauros,\\
En grupo lírico van los centauros\\
Con la harmonía de su tropel.  

Uno las patas rítmicas mueve,\\
Otro alza el cuello con gallardía\\
Como en hermoso bajo-relieve\\
Que a golpes mágicos Scopas haría;\\
Otro alza al aire las manos blancas\\
Mientras le dora las fincas ancas\\
Con baño cálido la luz del sol;\\
Y otro saltando piedras y troncos\\
Va dando alegres sus gritos roncos\\
Como el ruido de un caracol.  

Silencio. Señas hace ligero\\
El que en la tropa va delantero;\\
Porque a un recodo de la campaña\\
Llegan en donde Diana se baña.\\
Se oye el ruido de claras linfas\\
Y a la algazara que hacen las ninfas.\\
Risa de plata que el aire riega\\
Hasta sus ávidos oídos llega;\\
Golpes en la onda, palabras locas,\\
Gritos joviales de frescas bocas,\\
Y los ladridos de la traílla\\
Que diana tiene junto a la orilla\\
Del fresco río, donde está ella\\
Blanca y desnuda como una estrella.  

Tanta blancura que al cisne injuria\\
Abre los ojos de la lujuria:\\
Sobre las márgenes y rocas áridas\\
Vuela el enjambre de las cantáridas\\
Con su bruñido verde metálico,\\
Siempre propicias al culto fálico.\\
Amplias caderas, pie fino y breve;\\
Las dos colinas de rosa y nieve…\\
¡Cuadro soberbio de tentación!\\
¡Ay del cuitado que a ver se atreve\\
Lo que fue espanto para Acteón!\\
Cabellos rubios, mejillas tiernas,\\
Marmóreos cuellos, rosadas piernas,\\
Gracias ocultas del lindo coro,\\
En el herido cristal sonoro;\\
Seno en que hiciérase sagrada copa;\\
Tal ve en silencio la ardiente tropa.  

¿Quién adelanta su firme busto?\\
¿Quirón experto? ¿Folo robusto?\\
Es el más joven y es el más bello;\\
Su piel es blanca, crespo el cabello,\\
Los cascos finos, y en la mirada\\
Brilla del sátiro la llamarada.\\
En un instante, veloz y listo,\\
A una tan bella como Kalisto,\\
Ninfa que a la alta diosa acompaña,\\
Saca de la onda donde se baña:\\
La grupa vuelve, raudo galopa;\\
Tal iba el toro raptor de Europa\\
Con el orgullo de su conquista.  

¿A dó va Diana? Viva la vista\\
La planta alada, la cabellera\\
Mojada y suelta; terrible, fiera,\\
Corre del monte por la extensión;\\
Ladran sus perros enfurecidos;\\
Entre sus dedos humedecidos\\
Lleva una flecha para el ladrón.  

Ya a los centauros a ver alcanza\\
La cazadora; ya el dardo lanza,\\
Y un grito se oye de hondo dolor:\\
La casta divina de la venganza\\
Mató al raptor\ldots{}\\
La tropa rápida se esparce huyendo,\\
Forman los cascos sonoro estruendo.\\
Llegan las ninfas. Lloran. ¿Qué ven?\\
En la carrera la cazadora\\
Con su saeta castigadora\\
A la robada mató También.  

\end{verse}

\clearpage
\poemtitle {Cosas del Cid}
\begin{verse}

Cuenta Barbey, en versos que valen bien su prosa\\
Una hazaña del Cid, fresca como una rosa,\\
Pura como una perla. No se oyen en la hazaña\\
Resonar en el viento las trompetas de España,\\
Ni el azorado moro las tiendas abandona\\
Al ver al sol el alma de acero de Tizona.  

\emph{Babieca} descansando del huracán guerrero,\\
Tranquilo pace, mientras el bravo caballero\\
Sale a gozar del aire de la estación florida.\\
Ríe la primavera, y el vuelo de vida\\
Abre lirios y sueños en el jardín del mundo.\\
Rodrigo de Vivar pasa, meditabundo.\\
Por una senda en donde, bajo el sol glorioso,\\
Tendiéndole la mano, le detiene un leproso.  

Frente a frente, el soberbio príncipe del estrago\\
Y la victoria, joven, bello como Santiago,\\
Y el horror animado, la viviente carroña\\
Que infecta los suburbios de hedor y de ponzoña.  

Y al Cid tiende la mano el siniestro mendigo,\\
Y su escarcela busca y no encuentra Rodrigo.\\
--¡Oh, Cid, una limosna! --dice el precito.\\
--Hermano\\
¡Te ofrezco la desnuda limosna de mi mano!--\\
Dice el Cid; y, quitando su férreo guante, extiende\\
La diestra al miserable, que llora y que comprende.  

*  

Tal es el sucedido que el Condestable escancia\\
Como un vino precioso en su copa de Francia.\\
Yo agregaré este sorbo de licor castellano:  

*  

Cuando su guantelete hubo vuelto a la mano\\
El Cid, siguió su rumbo por la primaveral\\
Senda. Un pájaro daba su nota de cristal\\
En un árbol. El cielo profundo desleía\\
Un perfume de gracia en la gloria del día.\\
Las ermitas lanzaban en el aire sonoro\\
Su melodiosa lluvia de tórtolas de oro;\\
El alma de las flores iba por los caminos\\
A unirse a la piadosa voz de los peregrinos,\\
Y el gran Rodrigo Díaz de Vivar, satisfecho,\\
Iba cual si llevase una estrella en el pecho.\\
Cuando de la campiña, aromada de esencia\\
Sutil, salió una niña vestida de inocencia,\\
Una niña que fuera una mujer, de franca\\
Y angélica pupila, y muy dulce y muy blanca.\\
Una niña que fuera un hada o que surgiera\\
Encarnación de la divina primavera.  

Y fue al Cid y le dijo: «Alma de amor y fuego,\\
Por Jimena y por Dios un regalo te entrego,\\
Esta rosa naciente y este fresco laurel».  

Y el Cid, sobre su yelmo las frescas hojas siente\\
En su guante de hierro hay una flor naciente,\\
Y en lo íntimo del alma como un dulzor de miel.  

\end{verse}

\clearpage
\poemtitle {Dezires, layes y canciones}
\begin{verse}

{\bfseries\scshape {Dezir}}

Reina Venus, soberana\\
capitana\\
de deseos y pasiones,\\
en la tempestad humana\\
por ti mana\\
sangre de los corazones.\\
Una copa me dio el sino\\
y en ella bebí tu vino\\
y me embriagué de dolor,\\
pues me hizo experimentar\\
que en el vino del amor\\
hay la amargura del mar.  

Di al olvido turbulento\\
sentimiento,\\
y hallé un sátiro ladino\\
que dio a mi labio sediento\\
nuevo aliento,\\
nueva copa y nuevo vino.\\
Y al llegar la primavera,\\
en mi roja sangre fiera\\
triple llama fue encendida:\\
yo al flamante amor entrego\\
la vendimia de mi vida\\
bajo pámpanos de fuego.  

En la fruta misteriosa,\\
ámbar, rosa,\\
su deseo sacia el labio,\\
y en viva rosa se posa,\\
mariposa,\\
beso ardiente o beso sabio.\\
¡Bien haya el sátiro griego\\
que me enseñó el dulce juego!\\
En el reino de mi aurora\\
no hay ayer, hoy ni mañana;\\
danzo las danzas de ahora\\
con la música pagana.  

{\bfseries\scshape {Otro dezir}}

Ponte el traje azul que más\\
conviene a tu rubio encanto.\\
Luego, Mía, te pondrás\\
otro, color de amaranto,\\
y el que rima con tus ojos\\
y aquel de reflejos rojos\\
que a tu blancor sienta tanto.  

En el obscuro cabello\\
pon las perlas que conquistas;\\
en el columbino cuello\\
pon el collar de amatistas,\\
y ajorcas en los tobillos\\
de topacios amarillos\\
y esmeraldas nunca vistas.  

Un camarín te decoro\\
donde sabrás la lección\\
que dio a Angélica Medoro\\
y a Belkiss dio Salomón;\\
arderá mi sangre loca,\\
y en el vaso de tu boca\\
te sorberé el corazón.  

Luz de sueño, flor de mito,\\
tu admirable cuerpo canta\\
la gracia de Hermafrodito\\
con lo aéreo de Atalanta;\\
y de tu beldad ambigua\\
la evocada musa antigua\\
su himno de carne levanta.  

Del ánfora en que está el viejo\\
vino anacreóntico bebe;\\
Febo arruga el entrecejo\\
y Juno arrugarlo debe,\\
mas la joven Venus ríe\\
y Eros su filtro deslíe\\
en los cálices de Hebe.  

{\bfseries\scshape {Canción}}

Amor tu ventana enflora\\
y tu amante esta mañana\\
preludia por ti una diana\\
en la lira de la Aurora.  

Desnuda sale la bella,\\
y del cabello el tesoro\\
pone una nube de oro\\
en la desnudez de estrella:\\
y en la matutina hora\\
de la clara fuente mana\\
la salutación pagana\\
de las náyades a Flora.  

En el baño al beso incita\\
sobre el cristal de la onda\\
la sonrisa de Gioconda\\
en el rostro de Afrodita;\\
y el cuerpo que la luz dora,\\
adolescente, se hermana\\
con las formas de Diana\\
la celeste cazadora.  

Y mientras la hermosa juega\\
con el sonoro diamante,\\
más encendido que amante\\
el fogoso amante llega\\
a su divina señora.  

{\bfseries\scshape {Que el amor no admite cuerdas reflexiones}}

Señora, amor es violento,\\
y cuando nos transfigura\\
nos enciende el pensamiento\\
la locura.  

No pidas paz a mis brazos\\
que a los tuyos tienen presos:\\
son de guerra mis abrazos\\
y son de incendio mis besos;\\
y sería vano intento\\
el tornar mi mente obscura\\
si me enciende el pensamiento\\
la locura.  

Clara está la mente mía\\
de llamas de amor, señora,\\
como la tienda del día\\
o el palacio de la aurora.\\
Y al perfume de tu ungüento\\
te persigue mi ventura,\\
y me enciende el pensamiento\\
la locura.  

Mi gozo tu paladar\\
rico panal conceptúa,\\
como en el santo Cantar:\\
Mel et lac sub lingua tua.\\
La delicia de tu aliento\\
en tan fino vaso apura,\\
y me enciende el pensamiento\\
la locura.  

\end{verse}

\clearpage
\poemtitle {Las ánforas de Epicuro}
\begin{verse}

{\bfseries\scshape {La fuente}}

Joven, te ofrezco el don de esta copa de plata\\
Para que un día puedas calmar la sed ardiente,\\
La sed que con fuego más que la muerte mata.\\
Mas debes abrevarte tan sólo en una fuente,  

Otro agua que la suya tendrá que serte ingrata,\\
Busca su oculto origen en la gruta viviente\\
Donde la interna música de su cristal desata,\\
Junto al árbol que llora y la roca que siente.  

Guíete el misterioso eco de su murmullo,\\
Asciende por los riscos ásperos del orgullo,\\
Baja por la constancia y desciende al abismo  

Cuya entrada sombría guardan siete panteras:\\
Son los Siete Pecados las siete bestias fieras.\\
Llena la copa y bebe: la fuente está en ti mismo.  

{\bfseries\scshape {Palabras de la Satiresa}}

Un día oí una risa bajo la fronda espesa,\\
Vi brotar de lo verde dos manzanas lozanas;\\
Erectos senos eran las lozanas manzanas\\
Del busto que bruñía de sol la Satiresa:  

Era una Satiresa de mis fiestas paganas,\\
Que hace brotar clavel o rosa cuando besa;\\
Y furiosa y riente y que abrasa y que mesa,\\
Con los labios manchados por las moras tempranas.  

«Tú que fuiste, me dijo, un antiguo argonauta,\\
Alma que el sol sonrosa y que la mar zafira,\\
Sabe que está el secreto de todo ritmo y pauta  

En unir carne y alma a la esfera que gira,\\
Y amando a Pan y Apolo en la lira y la flauta,\\
Ser en la flauta Pan, como Apolo en la lira».  

{\bfseries\scshape {La hoja de oro}}

En el verde laurel que decora la frente\\
Que besaron los sueños y pulieron las horas,\\
Una hoja suscita como la luz naciente\\
En que entreabren sus ojos de fuego las auroras;  

O las solares pompas, o los fastos de Oriente,\\
Preseas bizantinas diademas de Theodoras,\\
O la lejana Cólquida que el soñador presiente\\
Y adonde los Jasones dirigirán las proras.  

Hoja de oro rojo, mayor es tu valía,\\
Pues para tus colores imperiales evocas\\
Con el triunfo de otoño y la sangre del día,  

El marfil de las frentes, la brasa de las bocas,\\
y la autumnal tristeza de las vírgenes locas\\
Por la Lujuria, madre de la Melancolía.  

{\bfseries\scshape {La gitanilla}}

Maravillosamente danzaba. Los diamantes\\
Negros de sus pupilas vertían su destello;\\
Era bello su rostro, era un rostro tan bello\\
Como el de las gitanas de don Miguel Cervantes.  

Ornábase con rojos claveles detonantes\\
La redondez obscura del casco del cabello,\\
Y la cabeza firme sobre el bronce del cuello\\
Tenía la patina de las horas errantes.  

Las guitarras decían en sus cuerdas sonoras\\
Las vagas aventuras y las errantes horas,\\
Volaban los fandangos, daba el clavel fragancia;  

La gitana, embriagada de lujuria y cariño,\\
Sintió cómo caída dentro de su corpiño\\
El bello luis de oro del artista de Francia.  
\end{verse}
\end{document}
