% rubber: module xelatex
\documentclass[12pt]{article}

\usepackage[spanish]{babel}
\usepackage[utf8]{inputenc}
% \usepackage{palatino}
\usepackage{verse}

\usepackage{fontspec}
\setmainfont{EB Garamond}

\date{}
\title{Poesía selecta de Pablo Neruda VII}
\begin{document}
\maketitle
\tableofcontents
\clearpage
\poemtitle {Jardín de invierno (1972)}
\begin{verse}

Llega el invierno. Espléndido dictado\\
me dan las lentas hojas\\
vestidas de silencio y amarillo.  

Soy un libro de nieve,\\
una espaciosa mano, una pradera,\\
un círculo que espera,\\
pertenezco a la tierra y a su invierno.  

Creció el rumor del mundo en el follaje,\\
ardió después el trigo constelado\\
por flores rojas como quemaduras,\\
luego llegó el otoño a establecer\\
la escritura del vino:\\
todo pasó, fue cielo pasajero\\
la copa del estío,\\
y se apagó la nube navegante.  

Yo esperé en el balcón, tan enlutado\\
como ayer con las yedras de mi infancia,\\
que la tierra extendiera\\
sus alas en mi amor deshabitado.  

Yo supe que la rosa caería\\
y el hueso del durazno transitorio\\
volvería a dormir y a germinar:\\
y me embriagué con la copa del aire\\
hasta que todo el mar se hizo nocturno\\
y el arrebol se convirtió en ceniza.  

La tierra vive ahora\\
tranquilizando su interrogatorio,\\
extendida la piel de su silencio.\\
Yo vuelvo a ser ahora\\
el taciturno que llegó de lejos\\
envuelto en lluvia fría y en campanas:\\
debo a la muerte pura de la tierra\\
la voluntad de mis germinaciones.  

\end{verse}

\clearpage
\poemtitle {La piel del abedul (1972)}
\begin{verse}

Como la piel del abedul\\
eres plateada y olorosa:\\
tengo que contar con tus ojos\\
al describir la primavera.  

Y aunque no sé cómo te llamas\\
no hay primer tomo sin mujer:\\
los libros se escriben con besos\\
(y yo les ruego que se callen\\
para que se acerque la lluvia).  

Quiero decir que entre dos mares\\
está colgando mi estatura\\
como una bandera abatida.\\
Y por mi amada sin mirada\\
estoy dispuesto hasta morir\\
aunque mi muerte se atribuya\\
a mi deficiente organismo\\
o a la tristeza innecesaria\\
depositada en los roperos.\\
Lo cierto es que el tiempo se escapa\\
y con voz de viuda me llama\\
desde los bosque olvidados.  

Antes de ver el mundo, entonces,\\
cuando mis ojos no se abrían\\
yo disponía de cuatro ojos:\\
los míos y los de mi amor:\\
no me pregunten si he cambiado\\
(es solo el tiempo el que envejece)\\
(vive cambiando de camisa\\
mientras yo sigo caminando).  

Todos los labios del amor\\
fueron haciendo mi ropaje\\
desde que me sentí desnudo:\\
ella se llamaba María\\
(tal vez Teresa se llamaba),\\
y me acostumbré a caminar\\
consumido por mis pasiones.  

Eres tú la que tú serás\\
mujer innata de mi amor,\\
la que de greda fue formada\\
o la de plumas que voló\\
o la mujer territorial\\
de cabellera en el follaje\\
o la concéntrica caída\\
como una moneda desnuda\\
en el estanque de un topacio\\
o la presente cuidadora\\
de mi incorrecta indisciplina\\
o bien la que nunca nació\\
y que yo espero todavía.  

Porque la luz del abedul\\
es la piel de la primavera.  

\end{verse}

\clearpage
\poemtitle {Los vivos, aún vivientes (1972)}
\begin{verse}

Los vivos, aún vivientes,\\
el amor del poeta de bronce,\\
una mujer más frágil que un huevo de perdiz,\\
delgada como el silbido del canario salvaje,\\
una llamada Lily Brik es mi amiga,\\
mi vieja amiga mía. No conocí su hoguera:\\
y solo su retrato en las cubiertas\\
de Mayakovski me advirtieron\\
que fueron estos ojos apagados\\
los que encendieron púrpura soviética\\
en la dimensión descubierta.  

Aquí Lily, aún fosforescente\\
desde su puñadito de cenizas\\
con una mano en todo lo que nace,\\
con una rosa de recibimiento\\
a todo golpe de ala que aparece,\\
herida por alguna tardía pedrada\\
destinada hoy aún a Mayakovski:\\
dulce y bravía Lily, buenas noches,\\
dame otra vez tu copa transparente\\
para beber de un trago y en tu honor\\
el pasado que canta y que crepita\\
como un ave de fuego.  

\end{verse}

\clearpage
\poemtitle {Aire de Europa y aire de Asia (1972)}
\begin{verse}

Aire de Europa y aire de Asia\\
se encuentran, se rechazan,\\
se casan, se confunden\\
en la ciudad del límite:\\
llega el polvo carbónico de Silesia,\\
la fragancia vinícola de Francia,\\
olor a Italia con cebollas fritas,\\
humo, sangre, claveles españoles,\\
todo lo trae el aire, la ventisca\\
de tundra y taiga bailan en la estepa,\\
el aire siberiano, fuerza pura,\\
viento de astro silvestre,\\
el ancho viento que hasta los Urales\\
con manos verdes como malaquita\\
plancha los caseríos, las praderas,\\
guarda en su centro un corazón de lluvia,\\
se desploma en arcángeles de nieve.  

\end{verse}

\clearpage
\poemtitle {Llama el océano (1972)}
\begin{verse}

No voy al mar en este ancho verano\\
cubierto de calor, no voy más lejos\\
de los muros, las puertas y las grietas\\
que circundan las vidas y mi vida.  

En qué distancia, frente a cuál ventana,\\
en qué estación de trenes\\
dejé olvidado el mar? Y allí quedamos,\\
yo dando las espaldas a lo que amo\\
mientras allá seguía la batalla\\
de blanco y verde y piedra y centelleo.  

Así fue, así parece que así fue:\\
cambian las vidas, y el que va muriendo\\
no sabe que esa parte de la vida,\\
esa nota mayor, esa abundancia\\
de cólera y fulgor quedaron lejos,\\
te fueron ciegamente cercenadas.  

No, yo me niego al mar desconocido,\\
muerto, rodeado de ciudades tristes,\\
mar cuyas olas no saben matar,\\
ni cargarse de sal y de sonido.\\
Yo quiero el mío mar, la artillería\\
del océano golpeando las orillas,\\
aquel derrumbe insigne de turquesas,\\
la espuma donde muere el poderío.  

No salgo al mar este verano: estoy\\
encerrado, enterrado, y a lo largo\\
del túnel que me lleva prisionero\\
oigo remotamente un trueno verde,\\
un cataclismo de botellas rotas,\\
un susurro de sal y de agonía.\\
Es el libertador. Es el océano,\\
lejos, allá, en mi patria, que me espera.  

\end{verse}

\clearpage
\poemtitle {Animal de luz (1973)}
\begin{verse}

Soy en este sin fin sin soledad\\
un animal de luz acorralado\\
por sus errores y por su follaje:\\
ancha es la selva: aquí mis semejantes\\
pululan, retroceden o trafican,\\
mientras yo me retiro acompañado\\
por la escolta que el tiempo determina:\\
olas del mar, estrellas de la noche.  

Es poco, es ancho, es escaso y es todo.\\
De tanto ver mis ojos otros ojos\\
y mi boca de tanto ser besada,\\
de haber tragado el humo\\
de aquellos trenes desaparecidos,\\
las viejas estaciones despiadadas\\
y el polvo de incesantes librerías,\\
el hombre yo, el mortal, se fatigó\\
de ojos, de besos, de humo, de caminos,\\
de libros más espesos que la tierra.  

Y hoy en el fondo del bosque perdido\\
oye el rumor del enemigo y huye\\
no de los otros sino de sí mismo,\\
de la conversación interminable,\\
del coro que cantaba con nosotros\\
y del significado de la vida.  

Porque una vez, porque una voz, porque una\\
sílaba o el transcurso de un silencio\\
o el sonido insepulto de la ola\\
me dejan frente a la verdad,\\
y no hay nada más que descifrar,\\
ni nada más que hablar: eso era todo:\\
se cerraron las puertas de la selva,\\
circula el sol abriendo los follajes,\\
sube la luna como fruta blanca\\
y el hombre se acomoda a su destino.  

\end{verse}

\clearpage
\poemtitle {Orégano (1973)}
\begin{verse}

Cuando aprendí con lentitud\\
a hablar\\
creo que ya aprendí la incoherencia:\\
no me entendía nadie, ni yo mismo,\\
y odié aquellas palabras\\
que me volvían siempre\\
al mismo pozo,\\
al pozo de mi ser aún oscuro,\\
aún traspasado de mi nacimiento,\\
hasta que me encontré sobre un andén\\
o en un campo recién estrenado\\
una palabra: \emph{orégano},\\
palabra que me desenredó\\
como sacándome de un laberinto.  

No quise aprender más palabra alguna.  

Quemé los diccionarios,\\
me encerré en esas sílabas cantoras,\\
retrospectivas, mágicas, silvestres,\\
y a todo grito por la orilla\\
de los ríos,\\
entre las afiladas espadañas\\
o en el cemento de la ciudadela,\\
en minas, oficinas y velorios,\\
yo masticaba mi palabra \emph{orégano}\\
y era como si fuera una paloma\\
la que soltaba entre los ignorantes.  

Qué olor a corazón temible,\\
qué olor a violetario verdadero,\\
y qué forma de párpado\\
para dormir cerrando los ojos:\\
la noche tiene \emph{orégano}\\
y otras veces haciéndose revólver\\
me acompañó a pasear entre las fieras:\\
esa palabra defendió mis versos.\\
Un tarascón, unos colmillos (iban\\
sin duda a destrozarme\\
los jabalíes y los cocodrilos):\\
entonces\\
saqué de mi bolsillo\\
mi estimable palabra:\\
\emph{orégano}, grité con alegría,\\
blandiéndola en mi mano temblorosa.  

Oh milagro, las fieras asustadas\\
me pidieron perdón y me pidieron\\
humildemente \emph{orégano}.  

Oh lepidóptero entre las palabras,\\
oh palabra helicóptero,\\
purísima y preñada\\
como una aparición sacerdotal\\
y cargada de aroma,\\
territorial como un leopardo negro,\\
fosforescente orégano\\
que me sirvió para no hablar con nadie,\\
y para aclarar mi destino\\
renunciando al alarde del discurso\\
con un secreto idioma, el del orégano.  

\end{verse}

\clearpage
\poemtitle {Cuando yo decidí quedarme claro (1973)}
\begin{verse}

Cuando yo decidí quedarme claro\\
y buscar mano a mano la desdicha\\
para jugar a los dados,\\
encontré la mujer que me acompaña\\
a troche y moche y noche,\\
a nube y a silencio.  

Matilde es esta,\\
esta se llama así\\
desde Chillán,\\
y llueva\\
o truene o salga\\
el día con su pelo azul\\
o la noche delgada,\\
ella,\\
dele que dele,\\
lista para mi piel,\\
para mi espacio,\\
abriendo todas las ventanas del mar\\
para que vuele la palabra escrita,\\
para que se llenen los muebles\\
de signos silenciosos,\\
de fuego verde.  

\end{verse}

\clearpage
\poemtitle {Todos (1973)}
\begin{verse}

Yo tal vez yo no seré, tal vez no pude,\\
no fui, no vi, no estoy:\\
qué es esto? Y en qué junio, en qué madera\\
crecí hasta ahora, continué naciendo?  

No crecí, no crecí, seguí muriendo?  

Yo repetí en las puertas\\
el sonido del mar,\\
de las campanas.\\
Yo pregunté por mí, con embeleso\\
(con ansiedad más tarde),\\
con cascabel, con agua,\\
con dulzura:\\
siempre llegaba tarde.\\
Ya estaba lejos mi anterioridad,\\
ya no me respondía yo a mí mismo,\\
me había ido muchas veces yo.  

Y fui a la próxima casa,\\
a la próxima mujer,\\
a todas partes\\
a preguntar por mí, por ti, por todos:\\
y donde yo no estaba ya no estaban,\\
todo estaba vacío\\
porque sencillamente no era hoy,\\
era mañana.  

Por qué buscar en vano\\
en cada puerta en que no existiremos\\
porque no hemos llegado todavía?  

Así fue como supe\\
que yo era exactamente como tú\\
y como todo el mundo.  

\end{verse}

\clearpage
\poemtitle {La chillaneja (1973)}
\begin{verse}

La vida me la pasé\\
buscando una chillaneja.\\
Por los palacios pasé,\\
miré detrás de las rejas,\\
los ranchos examiné\\
desde el patio hasta las tejas\\
hasta que,\\
hasta que me la encontré.  

Por las pampas, por los mares,\\
por los montes, los desiertos,\\
por los ríos, las ciudades,\\
por los campos y los puertos\\
fui buscando mi pareja.\\
Fui por las calles del mundo\\
buscando una chillaneja.  

Por los montes de Bolivia,\\
por las viñas de Rumania,\\
por los caminos de Francia,\\
en las casas de Alemania,\\
en China y en Portugal,\\
en otoño, en primavera,\\
en México, en Indostán,\\
por toda la tierra entera\\
una mujer de Chillán\\
que fuera mi compañera.  

La tierra la recorrí\\
buscando una chillaneja,\\
iba de allí para allí,\\
no encontraba mi pareja,\\
y el alma con que nací\\
se me iba poniendo vieja\\
hasta que,\\
hasta que la encontré.  

\end{verse}

\clearpage
\poemtitle {Final (1973)}
\begin{verse}

\emph{Matilde, años o días\\
dormidos, afiebrados,\\
aquí o allá,\\
clavando,\\
rompiendo el espinazo,\\
sangrando sangre verdadera,\\
despertando tal vez\\
o perdido, dormido:\\
camas clínicas, ventanas extranjeras,\\
vestidos blancos de las sigilosas,\\
la torpeza en los pies.}  

\emph{Luego estos viajes\\
y el mío mar de nuevo:\\
tu cabeza en la cabecera,}  

\emph{tus manos voladoras\\
en la luz, en mi luz,\\
sobre mi tierra.}  

\emph{Fue tan bello vivir\\
cuando vivías!}  

\emph{El mundo es más azul y más terrestre\\
de noche, cuando duermo\\
enorme, adentro de tus breves manos.}
\end{verse}
\end{document}
