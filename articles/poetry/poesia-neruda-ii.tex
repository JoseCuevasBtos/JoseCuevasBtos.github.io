% rubber: module xelatex
\documentclass[12pt]{article}

\usepackage[spanish]{babel}
\usepackage[utf8]{inputenc}
% \usepackage{palatino}
\usepackage{verse}

\usepackage{fontspec}
\setmainfont{EB Garamond}

\date{}
\title{Poesía selecta de Pablo Neruda II}
\begin{document}
\maketitle
\tableofcontents
\clearpage
\poemtitle {Barcarola (1932)}
\begin{verse}

Si solamente me tocaras el corazón,\\
si solamente pusieras tu boca en mi corazón,\\
tu fina boca, tus dientes,\\
si pusieras tu lengua como una flecha roja\\
allí donde mi corazón polvoriento golpea,\\
si soplaras en mi corazón, cerca del mar, llorando,\\
sonaría con un ruido oscuro, con sonido de ruedas de tren con sueño,\\
como aguas vacilantes,\\
como el otoño en hojas,\\
como sangre,\\
con un ruido de llamas húmedas quemando el cielo,\\
sonando como sueños o ramas o lluvias,\\
o bocinas de puerto triste,\\
si tú soplaras en mi corazón cerca del mar,\\
como un fantasma blanco,\\
al borde de la espuma,\\
en mitad del viento,\\
como un fantasma desencadenado, a la orilla del mar, llorando.  

Como ausencia extendida, como campana súbita,\\
el mar reparte el sonido del corazón,\\
lloviendo, atardeciendo, en una costa sola:\\
la noche cae sin duda,\\
y su lúgubre azul de estandarte en naufragio\\
se puebla de planetas de plata enronquecida.  

Y suena el corazón como un caracol agrio,\\
llama, oh mar, oh lamento, oh derretido espanto\\
esparcido en desgracias y olas desvencijadas:\\
de lo sonoro el mar acusa\\
sus sombras recostadas, sus amapolas verdes.  

Si existieras de pronto, en una costa lúgubre,\\
rodeada por el día muerto,\\
frente a una nueva noche,\\
llena de olas,\\
y soplaras en mi corazón de miedo frío,\\
soplaras en la sangre sola de mi corazón,\\
soplaras en su movimiento de paloma con llamas,\\
sonarían sus negras sílabas de sangre,\\
crecerían sus incesantes aguas rojas,\\
y sonaría, sonaría a sombras,\\
sonaría como la muerte,\\
llamaría como un tubo lleno de viento o llanto,\\
o una botella echando espanto a borbotones.  

Así es, y los relámpagos cubrirían tus trenzas\\
y la lluvia entraría por tus ojos abiertos\\
a preparar el llanto que sordamente encierras,\\
y las alas negras del mar girarían en torno\\
de ti, con grandes garras, y graznidos, y vuelos.  

¿Quieres ser el fantasma que sople, solitario,\\
cerca del mar su estéril, triste instrumento?\\
Si solamente llamaras,\\
su prolongado son, su maléfico pito,\\
su orden de olas heridas,\\
alguien vendría acaso,\\
alguien vendría,\\
desde las cimas de las islas, desde el fondo rojo del mar,\\
alguien vendría, alguien vendría.  

Alguien vendría, sopla con furia,\\
que suene como sirena de barco roto,\\
como lamento,\\
como un relincho en medio de la espuma y la sangre,\\
como un agua feroz mordiéndose y sonando.  

En la estación marina\\
su caracol de sombra circula como un grito,\\
los pájaros del mar lo desestiman y huyen,\\
sus listas de sonido, sus lúgubres barrotes\\
se levantan a orillas del océano solo.  

\end{verse}

\clearpage
\poemtitle {El sur del océano (1933)}
\begin{verse}

\emph{De consumida sal y garganta en peligro\\
están hechas las rosas del océano solo,\\
el agua rota sin embargo,\\
y pájaros temibles,\\
y no hay sino la noche acompañada\\
del día, y el día acompañado\\
de un refugio, de una\\
pezuña, del silencio.}  

\emph{En el silencio crece el viento\\
con su hoja única y su flor golpeada,\\
y la arena que tiene sólo tacto y silencio,\\
no es nada, es una sombra,\\
una pisada de caballo vago,\\
no es nada sino una ola que el tiempo ha recibido,\\
porque todas las aguas van a los ojos fríos\\
del tiempo que debajo del océano mira.}  

\emph{Ya sus ojos han muerto de agua muerta y palomas,\\
y son dos agujeros de latitud amarga\\
por donde entran los peces de ensangrentados dientes\\
y las ballenas buscando esmeraldas,\\
y esqueletos de pálidos caballeros deshechos\\
por las lentas medusas, y además\\
varias asociaciones de arrayán venenoso,\\
manos aisladas, flechas,\\
revólveres de escama,\\
interminablemente corren por sus mejillas\\
y devoran sus ojos de sal destituida.\\
Cuando la luna entrega sus naufragios,\\
sus cajones, sus muertos\\
cubiertos de amapolas masculinas,\\
cuando en el saco de la luna caen\\
los trajes sepultados en el mar\\
con sus largos tormentos, sus barbas derribadas,\\
sus cabezas que el agua y el orgullo pidieron para siempre\\
en la extensión se oyen caer rodillas\\
hacia el fondo del mar traídas por la luna\\
en su saco de piedra gastado por las lágrimas\\
y por las mordeduras de pescados siniestros.}  

\emph{Es verdad, es la luna descendiendo\\
con crueles sacudidas de esponja, es, sin embargo,\\
la luna tambaleando entre las madrigueras,\\
la luna carcomida por los gritos del agua,\\
los vientres de la luna, sus escamas\\
de acero despedido: y desde entonces\\
al final del Océano desciende,\\
azul y azul, atravesada por azules,\\
ciegos azules de materia ciega,\\
arrastrando su cargamento corrompido,\\
buzos, maderas, dedos,\\
pescadora de la sangre que en las cimas del mar\\
ha sido derramada por grandes desventuras.}  

\emph{Pero hablo de una orilla, es allí donde azota\\
el mar con furia y las olas golpean\\
los muros de ceniza. ¿Qué es esto? ¿Es una sombra?\\
No es la sombra, es la arena de la triste república,\\
es un sistema de algas, hay alas, hay\\
un picotazo en el pecho del cielo:\\
oh superficie herida por las olas,\\
oh manantial del mar,\\
si la lluvia asegura tus secretos, si el viento interminable\\
mata los pájaros, si solamente el cielo,\\
sólo quiero morder tus costas y morirme,\\
sólo quiero mirar la boca de las piedras\\
por donde los secretos salen llenos de espuma.}  

\emph{Es una región sola, ya he hablado\\
de esta región tan sola,\\
donde la tierra está llena de océano,\\
y no hay nadie sino unas huellas de caballo,\\
no hay nadie sino el viento, no hay nadie\\
sino la lluvia que cae sobre las aguas del mar,\\
nadie sino la lluvia que crece sobre el mar.}  

\end{verse}

\clearpage
\poemtitle {Solo la muerte (1933)}
\begin{verse}

\emph{Hay cementerios solos,\\
tumbas llenas de huesos sin sonido,\\
el corazón pasando un túnel\\
oscuro, oscuro, oscuro,\\
como un naufragio hacia adentro nos morimos,\\
como ahogarnos en el corazón,\\
como irnos cayendo desde la piel al alma.}  

\emph{Hay cadáveres,\\
hay pies de pegajosa losa fría,\\
hay la muerte en los huesos,\\
como un sonido puro,\\
como un ladrido sin perro,\\
saliendo de ciertas campanas, de ciertas tumbas,\\
creciendo en la humedad como el llanto o la lluvia.}  

\emph{Yo veo, solo, a veces,\\
ataúdes a vela\\
zarpar con difuntos pálidos, con mujeres de trenzas muertas,\\
con panaderos blancos como ángeles,\\
con niñas pensativas casadas con notarios,\\
ataúdes subiendo el río vertical de los muertos,\\
el río morado,\\
hacia arriba, con las velas hinchadas por el sonido de la muerte,\\
hinchadas por el sonido silencioso de la muerte.}  

\emph{A lo sonoro llega la muerte\\
como un zapato sin pie, como un traje sin hombre,\\
llega a golpear con un anillo sin piedra y sin dedo,\\
llega a gritar sin boca, sin lengua, sin garganta.}  

\emph{Sin embargo sus pasos suenan\\
y su vestido suena, callado, como un árbol.}  

\emph{Yo no sé, yo conozco poco, yo apenas veo,\\
pero creo que su canto tiene color de violetas húmedas,\\
de violetas acostumbradas a la tierra\\
porque la cara de la muerte es verde,\\
y la mirada de la muerte es verde,\\
con la aguda humedad de una hoja de violeta\\
y su grave color de invierno exasperado.}  

\emph{Pero la muerte va también por el mundo vestida de escoba,\\
lame el suelo buscando difuntos,\\
la muerte está en la escoba,\\
es la lengua de la muerte buscando muertos,\\
es la aguja de la muerte buscando hilo.}  

\emph{La muerte está en los catres:\\
en los colchones lentos, en las frazadas negras\\
vive tendida, y de repente sopla:\\
sopla un sonido oscuro que hincha sábanas,\\
y hay camas navegando a un puerto\\
en donde está esperando, vestida de almirante.}  

\end{verse}

\clearpage
\poemtitle {Walking around (1933)}
\begin{verse}

Sucede que me canso de ser hombre.\\
Sucede que entro en las sastrerías y en los cines\\
marchito, impenetrable, como un cisne de fieltro\\
navegando en un agua de origen y ceniza.  

El olor de las peluquerías me hace llorar a gritos.\\
Sólo quiero un descanso de piedras o de lana,\\
sólo quiero no ver establecimientos ni jardines,\\
ni mercaderías, ni anteojos, ni ascensores.  

Sucede que me canso de mis pies y mis uñas\\
y mi pelo y mi sombra.\\
Sucede que me canso de ser hombre.  

Sin embargo sería delicioso\\
asustar a un notario con un lirio cortado\\
o dar muerte a una monja con un golpe de oreja.\\
Sería bello\\
ir por las calles con un cuchillo verde\\
y dando gritos hasta morir de frío.  

No quiero seguir siendo raíz en las tinieblas,\\
vacilante, extendido, tiritando de sueño,\\
hacia abajo, en las tripas mojadas de la tierra,\\
absorbiendo y pensando, comiendo cada día.  

No quiero para mí tantas desgracias.\\
No quiero continuar de raíz y de tumba,\\
de subterráneo solo, de bodega con muertos\\
ateridos, muriéndome de pena.  

Por eso el día lunes arde como el petróleo\\
cuando me ve llegar con mi cara de cárcel,\\
y aúlla en su transcurso como una rueda herida,\\
y da pasos de sangre caliente hacia la noche.  

Y me empuja a ciertos rincones, a ciertas casas húmedas,\\
a hospitales donde los huesos salen por la ventana,\\
a ciertas zapaterías con olor a vinagre,\\
a calles espantosas como grietas.  

Hay pájaros de color de azufre y horribles intestinos\\
colgando de las puertas de las casas que odio,\\
hay dentaduras olvidadas en una cafetera,\\
hay espejos\\
que debieran haber llorado de vergüenza y espanto,\\
hay paraguas en todas partes, y venenos, y ombligos.  

Yo paseo con calma, con ojos, con zapatos,\\
con furia, con olvido,\\
paso, cruzo oficinas y tiendas de ortopedia,\\
y patios donde hay ropas colgadas de un alambre:\\
calzoncillos, toallas y camisas que lloran\\
lentas lágrimas sucias.  

\end{verse}

\clearpage
\poemtitle {Alberto Rojas Giménez viene volando (1934)}
\begin{verse}

Entre plumas que asustan, entre noches,\\
entre magnolias, entre telegramas,\\
entre el viento del Sur y el Oeste marino,\\
vienes volando.  

Bajo las tumbas, bajo las cenizas,\\
bajo los caracoles congelados,\\
bajo las últimas aguas terrestres,\\
vienes volando.  

Más abajo, entre niñas sumergidas,\\
y plantas ciegas, y pescados rotos,\\
más abajo, entre nubes otra vez,\\
vienes volando.  

Más allá de la sangre y de los huesos,\\
más allá del pan, más allá del vino,\\
más allá del fuego,\\
vienes volando.  

Más allá del vinagre y de la muerte,\\
entre putrefacciones y violetas,\\
con tu celeste voz y tus zapatos húmedos,\\
vienes volando.  

Sobre diputaciones y farmacias,\\
y ruedas, y abogados, y navíos,\\
y dientes rojos recién arrancados,\\
vienes volando.  

Sobre ciudades de tejado hundido\\
en que grandes mujeres se destrenzan\\
con anchas manos y peines perdidos,\\
vienes volando.  

Junto a bodegas donde el vino crece\\
con tibias manos turbias, en silencio,\\
con lentas manos de madera roja,\\
vienes volando.  

Entre aviadores desaparecidos,\\
al lado de canales y de sombras,\\
al lado de azucenas enterradas,\\
vienes volando.  

Entre botellas de color amargo,\\
entre anillos de anís y desventura,\\
levantando las manos y llorando,\\
vienes volando.  

Sobre dentistas y congregaciones,\\
sobre cines, y túneles y orejas,\\
con traje nuevo y ojos extinguidos,\\
vienes volando.  

Sobre tu cementerio sin paredes\\
donde los marineros se extravían,\\
mientras la lluvia de tu muerte cae,\\
vienes volando.  

Mientras la lluvia de tus dedos cae,\\
mientras la lluvia de tus huesos cae,\\
mientras tu médula y tu risa caen,\\
vienes volando.  

Sobre las piedras en que te derrites,\\
corriendo, invierno abajo, tiempo abajo,\\
mientras tu corazón desciende en gotas,\\
vienes volando.  

No estás allí, rodeado de cemento,\\
y negros corazones de notarios,\\
y enfurecidos huesos de jinetes:\\
vienes volando.  

Oh amapola marina, oh deudo mío,\\
oh guitarrero vestido de abejas,\\
no es verdad tanta sombra en tus cabellos:\\
vienes volando.  

No es verdad tanta sombra persiguiéndote,\\
no es verdad tantas golondrinas muertas,\\
tanta región oscura con lamentos:\\
vienes volando.  

El viento negro de Valparaíso\\
abre sus alas de carbón y espuma\\
para barrer el cielo donde pasas:\\
vienes volando.  

Hay vapores, y un frío de mar muerto,\\
y silbatos, y mesas, y un olor\\
de mañana lloviendo y peces sucios:\\
vienes volando.  

Hay ron, tú y yo, y mi alma donde lloro,\\
y nadie, y nada, sino una escalera\\
de peldaños quebrados, y un paraguas:\\
vienes volando.  

Allí está el mar. Bajo de noche y te oigo\\
venir volando bajo el mar sin nadie,\\
bajo el mar que me habita, oscurecido:\\
vienes volando.  

Oigo tus alas y tu lento vuelo,\\
y el agua de los muertos me golpea\\
como palomas ciegas y mojadas:\\
vienes volando.  

Vienes volando, solo solitario,\\
solo entre muertos, para siempre solo,\\
vienes volando sin sombra y sin nombre,\\
sin azúcar, sin boca, sin rosales,  

\end{verse}

\clearpage
\poemtitle {Enfermedades en mi casa (1934)}
\begin{verse}

Cuando el deseo de alegría con sus dientes de rosa\\
escarba los azufres caídos durante muchos meses\\
y su red natural, sus cabellos sonando\\
a mis habitaciones extinguidas con ronco paso llegan,\\
allí la rosa de alambre maldito\\
golpea con arañas las paredes\\
y el vidrio roto hostiliza la sangre,\\
y las uñas del cielo se acumulan,\\
de tal modo que no se puede salir, que no se puede dirigir\\
un asunto estimable,\\
es tanta la niebla, la vaga niebla cagada por los pájaros,\\
es tanto el humo convertido en vinagre\\
y el agrio aire que horada las escalas:\\
en ese instante en que el día se cae con las plumas deshechas,\\
no hay sino llanto, nada más que llanto,\\
porque sólo sufrir, solamente sufrir,\\
y nada más que llanto.  

El mar se ha puesto a golpear por años una pata de pájaro,\\
y la sal golpea y la espuma devora,\\
las raíces de un árbol sujetan una mano de niña,\\
las raíces de un árbol más grande que una mano de niña,\\
más grande que una mano del cielo,\\
y todo el año trabajan, cada día de luna\\
sube sangre de niña hacia las hojas manchadas por la luna,\\
y hay un planeta de terribles dientes\\
envenenando el agua en que caen los niños,\\
cuando es de noche, y no hay sino la muerte,\\
solamente la muerte, y nada más que el llanto.  

Como un grano de trigo en el silencio, pero\\
¿a quién pedir piedad por un grano de trigo?\\
Ved cómo están las cosas: tantos trenes,\\
tantos hospitales con rodillas quebradas,\\
tantas tiendas con gentes moribundas:\\
entonces, ¿cómo?, ¿cuándo?,\\
¿a quién pedir por unos ojos del color de un mes frío,\\
y por un corazón del tamaño del trigo que vacila?\\
No hay sino ruedas y consideraciones,\\
alimentos progresivamente distribuidos,\\
líneas de estrellas, copas\\
en donde nada cae, sino sólo la noche,\\
nada más que la muerte.  

Hay que sostener los pasos rotos.\\
Cruzar entre tejados y tristezas mientras arde\\
una cosa quemada con llamas de humedad,\\
una cosa entre trapos tristes como la lluvia,\\
algo que arde y solloza,\\
un síntoma, un silencio.\\
Entre abandonadas conversaciones y objetos respirados,\\
entre las flores vacías que el destino corona y abandona,\\
hay un río que cae en una herida,\\
hay el océano golpeando una sombra de flecha quebrantada,\\
hay todo el cielo agujereando un beso.  

Ayudadme, hojas que mi corazón ha adorado en silencio,\\
ásperas travesías, inviernos del sur, cabelleras\\
de mujeres mojadas en mi sudor terrestre,\\
luna del sur del cielo deshojado,\\
venid a mí con un día sin dolor,\\
con un minuto en que pueda reconocer mis venas.  

Estoy cansado de una gota,\\
estoy herido en solamente un pétalo,\\
y por un agujero de alfiler sube un río de sangre sin consuelo,\\
y me ahogo en las aguas del rocío que se pudre en la sombra,\\
y por una sonrisa que no crece, por una boca dulce,\\
por unos dedos que el rosal quisiera\\
escribo este poema que sólo es un lamento,\\
solamente un lamento.  

\end{verse}

\clearpage
\poemtitle {Estatuto del vino (1934)}
\begin{verse}

Cuando a regiones, cuando a sacrificios\\
manchas moradas como lluvias caen,\\
el vino abre las puertas con asombro,\\
y en el refugio de los meses vuela\\
su cuerpo de empapadas alas rojas.  

Sus pies tocan los muros y las tejas\\
con humedad de lenguas anegadas,\\
y sobre el filo del día desnudo\\
sus abejas en gotas van cayendo.  

Yo sé que el vino no huye dando gritos\\
a la llegada del invierno,\\
ni se esconde en iglesias tenebrosas\\
a buscar fuego en trapos derrumbados,\\
sino que vuela sobre la estación,\\
sobre el invierno que ha llegado ahora\\
con un puñal entre las cejas duras.  

Yo veo vagos sueños,\\
yo reconozco lejos,\\
y miro frente a mí, detrás de los cristales,\\
reuniones de ropas desdichadas.  

A ellas la bala del vino no llega,\\
su amapola eficaz, su rayo rojo\\
mueren ahogados en tristes tejidos,\\
y se derrama por canales solos,\\
por calles húmedas, por ríos sin nombre,\\
el vino amargamente sumergido,\\
el vino ciego y subterráneo y solo.  

Yo estoy de pie en su espuma y sus raíces,\\
yo lloro en su follaje y en sus muertos,\\
acompañado de sastres caídos\\
en medio del invierno deshonrado,\\
yo subo escalas de humedad y sangre\\
tanteando las paredes,\\
y en la congoja del tiempo que llega\\
sobre una piedra me arrodillo y lloro.  

Y hacia túneles acres me encamino\\
vestido de metales transitorios,\\
hacia bodegas solas, hacia sueños,\\
hacia betunes verdes que palpitan,\\
hacia herrerías desinteresadas,\\
hacia sabores de lodo y garganta,\\
hacia imperecederas mariposas.  

Entonces surgen los hombres del vino\\
vestidos de morados cinturones\\
y sombreros de abejas derrotadas,\\
y traen copas llenas de ojos muertos,\\
y terribles espadas de salmuera,\\
y con roncas bocinas se saludan\\
cantando cantos de intención nupcial.  

Me gusta el canto ronco de los hombres del vino,\\
y el ruido de mojadas monedas en la mesa,\\
y el olor de zapatos y de uvas\\
y de vómitos verdes:\\
me gusta el canto ciego de los hombres,\\
y ese sonido de sal que golpea\\
las paredes del alba moribunda.  

Hablo de cosas que existen, ¡Dios me libre\\
de inventar cosas cuando estoy cantando!\\
Hablo de la saliva derramada en los muros,\\
hablo de lentas medias de ramera,\\
hablo del coro de los hombres del vino\\
golpeando el ataúd con un hueso de pájaro.  

Estoy en medio de ese canto, en medio\\
del invierno que rueda por las calles,\\
estoy en medio de los bebedores,\\
con los ojos abiertos hacia olvidados sitios,\\
o recordando en delirante luto,\\
o durmiendo en cenizas derribado.  

Recordando noches, navíos, sementeras,\\
amigos fallecidos, circunstancias,\\
amargos hospitales y niñas entreabiertas:\\
recordando un golpe de ola en cierta roca,\\
con un adorno de harina y espuma,\\
y la vida que hace uno en ciertos países,\\
en ciertas costas solas,\\
un sonido de estrellas en las palmeras,\\
un golpe del corazón en los vidrios,\\
un tren que cruza oscuro de ruedas malditas\\
y muchas cosas tristes de esta especie.  

A la humedad del vino, en las mañanas,\\
en las paredes a menudo mordidas por los días de invierno\\
que caen en bodegas sin duda solitarias,\\
a esa virtud del vino llegan luchas,\\
y cansados metales y sordas dentaduras,\\
y hay un tumulto de objeciones rotas,\\
hay un furioso llanto de botellas,\\
y un crimen, como un látigo caído.  

El vino clava sus espinas negras,\\
y sus erizos lúgubres pasea,\\
entre puñales, entre mediasnoches,\\
entre roncas gargantas arrastradas,\\
entre cigarros y torcidos pelos,\\
y como ola de mar su voz aumenta\\
aullando llanto y manos de cadáver.  

Y entonces corre el vino perseguido\\
y sus tenaces odres se destrozan\\
contra las herraduras, y va el vino en silencio,\\
y sus toneles, en heridos buques en donde el aire muerde\\
rostros, tripulaciones de silencio,\\
y el vino huye por las carreteras,\\
por las iglesias, entre los carbones,\\
y se caen sus plumas de amaranto,\\
y se disfraza de azufre su boca,\\
y el vino ardiendo entre calles usadas,\\
buscando pozos, túneles, hormigas,\\
bocas de tristes muertos,\\
por donde ir al azul de la tierra\\
en donde se confunden la lluvia y los ausentes.  

\end{verse}

\clearpage
\poemtitle {Oda a Federico García Lorca (1935)}
\begin{verse}

Si pudiera llorar de miedo en una casa sola,\\
si pudiera sacarme los ojos y comérmelos,\\
lo haría por tu voz de naranjo enlutado\\
y por tu poesía que sale dando gritos.  

Porque por ti pintan de azul los hospitales\\
y crecen las escuelas y los barrios marítimos,\\
y se pueblan de plumas los ángeles heridos,\\
y se cubren de escamas los pescados nupciales,\\
y van volando al cielo los erizos:\\
por ti las sastrerías con sus negras membranas\\
se llenan de cucharas y de sangre\\
y tragan cintas rotas, y se matan a besos,\\
y se visten de blanco.  

Cuando vuelas vestido de durazno,\\
cuando ríes con risa de arroz huracanado,\\
cuando para cantar sacudes las arterias y los dientes,\\
la garganta y los dedos,\\
me moriría por lo dulce que eres,\\
me moriría por los lagos rojos\\
en donde en medio del otoño vives\\
con un corcel caído y un dios ensangrentado,\\
me moriría por los cementerios\\
que como cenicientos ríos pasan\\
con agua y tumbas,\\
de noche, entre campanas ahogadas:\\
ríos espesos como dormitorios\\
de soldados enfermos, que de súbito crecen\\
hacia la muerte en ríos con números de mármol\\
y coronas podridas, y aceites funerales:\\
me moriría por verte de noche\\
mirar pasar las cruces anegadas,\\
de pie llorando,\\
porque ante el río de la muerte lloras\\
abandonadamente, heridamente,\\
lloras llorando, con los ojos llenos\\
de lágrimas, de lágrimas, de lágrimas.  

Si pudiera de noche, perdidamente solo,\\
acumular olvido y sombra y humo\\
sobre ferrocarriles y vapores,\\
con un embudo negro,\\
mordiendo las cenizas,\\
lo haría por el árbol en que creces,\\
por los nidos de aguas doradas que reúnes,\\
y por la enredadera que te cubre los huesos\\
comunicándote el secreto de la noche.  

Ciudades con olor a cebolla mojada\\
esperan que tú pases cantando roncamente,\\
y silenciosos barcos de esperma te persiguen,\\
y golondrinas verdes hacen nido en tu pelo,\\
y además caracoles y semanas,\\
mástiles enrollados y cerezas\\
definitivamente circulan cuando asoman\\
tu pálida cabeza de quince ojos\\
y tu boca de sangre sumergida.  

Si pudiera llenar de hollín las alcaldías\\
y, sollozando, derribar relojes,\\
sería para ver cuándo a tu casa\\
llega el verano con los labios rotos,\\
llegan muchas personas de traje agonizante,\\
llegan regiones de triste esplendor,\\
llegan arados muertos y amapolas,\\
llegan enterradores y jinetes,\\
llegan planetas y mapas con sangre,\\
llegan buzos cubiertos de ceniza,\\
llegan enmascarados arrastrando doncellas\\
atravesadas por grandes cuchillos,\\
llegan raíces, venas, hospitales,\\
manantiales, hormigas,\\
llega la noche con la cama en donde\\
muere entre las arañas un húsar solitario,\\
llega una rosa de odio y alfileres,\\
llega una embarcación amarillenta,\\
llega un día de viento con un niño,\\
llego yo con Oliverio, Norah\\
Vicente Aleixandre, Delia,\\
Maruca, Malva Marina, María Luisa y Larco,\\
la Rubia, Rafael Ugarte,\\
Cotapos, Rafael Alberti,\\
Carlos, Bebé, Manolo Altolaguirre,\\
Molinari,\\
Rosales, Concha Méndez,\\
y otros que se me olvidan.  

Ven a que te corone, joven de la salud\\
y de la mariposa, joven puro\\
como un negro relámpago perpetuamente libre,\\
y conversando entre nosotros,\\
ahora, cuando no queda nadie entre las rocas,\\
hablemos sencillamente como eres tú y soy yo:\\
¿para qué sirven los versos si no es para el rocío?  

¿Para qué sirven los versos si no es para esa noche\\
en que un puñal amargo nos averigua, para ese día,\\
para ese crepúsculo, para ese rincón roto\\
donde el golpeado corazón del hombre se dispone a morir?  

Sobre todo de noche,\\
de noche hay muchas estrellas,\\
todas dentro de un río\\
como una cinta junto a las ventanas\\
de las casas llenas de pobres gentes.  

Alguien se les ha muerto, tal vez\\
han perdido sus colocaciones en las oficinas,\\
en los hospitales, en los ascensores,\\
en las minas,\\
sufren los seres tercamente heridos\\
y hay propósito y llanto en todas partes:\\
mientras las estrellas corren dentro de un río interminable\\
hay mucho llanto en las ventanas,\\
los umbrales están gastados por el llanto,\\
las alcobas están mojadas por el llanto\\
que llega en forma de ola a morder las alfombras.  

Federico,\\
tú ves el mundo, las calles,\\
el vinagre,\\
las despedidas en las estaciones\\
cuando el humo levanta sus ruedas decisivas\\
hacia donde no hay nada sino algunas\\
separaciones, piedras, vías férreas.  

Hay tantas gentes haciendo preguntas\\
por todas partes.\\
Hay el ciego sangriento, y el iracundo, y el\\
desanimado,\\
y el miserable, el árbol de las uñas,\\
el bandolero con la envidia a cuestas.  

Así es la vida, Federico, aquí tienes\\
las cosas que te puede ofrecer mi amistad\\
de melancólico varón varonil.\\
Ya sabes por ti mismo muchas cosas.\\
Y otras irás sabiendo lentamente.  

\end{verse}

\clearpage
\poemtitle {Himno y regreso (1939)}
\begin{verse}

Patria, mi patria, vuelvo hacia ti la sangre.\\
Pero te pido, como a la madre el niño\\
lleno de llanto.\\
Acoge\\
esta guitarra ciega\\
y esta frente perdida.  

Salí a encontrarte hijos por la tierra,\\
salía cuidar caídos con tu nombre de nieve,\\
salí a hacer una casa con tu madera pura,\\
salí a llevar tu estrella a los héroes heridos.\\
Ahora quiero dormir en tu substancia.\\
Dame tu clara noche de penetrantes cuerdas,\\
tu noche de navío, tu estatura estrellada.  

Patria mía: quiero mudar de sombra.\\
Patria mía: quiero cambiar de rosa.\\
Quiero poner mi brazo en tu cintura exigua\\
y sentarme en tus piedras por el mar calcinadas,\\
a detener el trigo y mirarlo por dentro.\\
Voy a escoger la flora delgada del nitrato,\\
voy a hilar el estambre glacial de la campana,\\
y mirando tu ilustre y solitaria espuma\\
un ramo litoral tejeré a tu belleza.  

Patria, mi patria\\
toda rodeada de agua combatiente\\
y nieve combatida,\\
en ti se junta el águila al azufre,\\
y en tu antártica mano de armiño y de zafiro\\
una gota de pura luz humana\\
brilla encendiendo el enemigo cielo.  

Guarda tu luz, oh patria! mantén\\
tu dura espiga de esperanza en medio\\
del ciego aire temible.\\
En tu remota tierra ha caído toda esta luz difícil,\\
este destino de los hombres,\\
que te hace defender una flor, misteriosa\\
sola, en la inmensidad de América dormida.  

\end{verse}

\clearpage
\poemtitle {Botánica (1940)}
\begin{verse}

El sanguinario litre y el benéfico boldo\\
diseminan su estilo\\
en irritantes besos de animal esmeralda\\
o antologías de agua oscura entre las piedras.  

El chupón en la cima del árbol establece\\
su dentadura nívea\\
y el salvaje avellano construye su castillo\\
de páginas y gotas.  

La altamisa y la chépica rodean\\
los ojos del orégano\\
y el radiante laurel de la frontera\\
perfuma las lejanas intendencias.  

Quila y quelenquelén de las mañanas.\\
Idioma frío de las fucsias,\\
que se va por las piedras tricolores\\
gritando viva Chile con la espuma!  

El dedal de oro espera\\
los dedos de la nieve\\
y rueda el tiempo sin su matrimonio\\
que uniría a los ángeles del fuego y del azúcar.  

El mágico canelo\\
lava en la lluvia su racial ramaje,\\
y precipita sus lingotes verdes\\
bajo la vegetal agua del Sur.  

La dulce aspa del ulmo\\
con fanegas de llores\\
sube las gotas del copihue rojo\\
a conocer el sol de las guitarras.  

La agreste delgadilla\\
y el celestial poleo\\
bailan en las praderas con el joven rocío\\
recientemente armado por el río Toltén.  

La indescifrable doca\\
decapita su púrpura en la arena\\
y conduce sus triángulos marinos\\
hacia las secas lunas litorales.  

La bruñida amapola,\\
relámpago y herida, dardo y boca,\\
sobre el quemante trigo\\
pone sus puntuaciones escarlata.  

La patagua evidente\\
condecora sus muertos\\
y teje sus familias\\
con manantiales aguas y medallas de río.  

El paico arregla lámparas\\
en el clima del Sur, desamparado,\\
cuando viene la noche\\
del mar nunca dormido.  

El roble duerme solo,\\
muy vertical, muy pobre, muy mordido,\\
muy decisivo en la pradera pura\\
con su traje de roto maltratado\\
y su cabeza llena de solemnes estrellas.  

\end{verse}

\clearpage
\poemtitle {Quiero volver al sur (1941)}
\begin{verse}

Enfermo en Veracruz, recuerdo un día\\
del Sur, mi tierra, un día de plata\\
como un rápido pez en el agua del cielo.\\
Loncoche, Lonquimay, Carahue, desde arriba\\
esparcidos, rodeados por silencio y raíces,\\
sentados en sus tronos de cueros y maderas.\\
El Sur es un caballo echado a pique\\
coronado con lentos árboles y rocío,\\
cuando levanta el verde hocico caen las gotas,\\
la sombra de su cola moja el gran archipiélago\\
y en su intestino crece el carbón venerado.\\
Nunca más, dime, sombra, nunca más, dime, mano,\\
nunca más, dime, pie, puerta, pierna, combate,\\
trastornarás la selva, el camino, la espiga,\\
la niebla, el frío, lo que, azul, determinaba\\
cada uno de tus pasos sin cesar consumidos?\\
Cielo, déjame un día de estrella a estrella irme\\
pisando luz y pólvora, destrozando mi sangre\\
hasta llegar al nido de la lluvia!\\
Quiero ir\\
detrás de la madera por el río\\
Toltén fragante, quiero salir de los aserraderos,\\
entrar en las cantinas con los pies empapados,\\
guiarme por la luz del avellano eléctrico,\\
tenderme junto al excremento de las vacas,\\
morir y revivir mordiendo trigo.\\
Océano, tráeme\\
un día del Sur, un día agarrado a tus olas,\\
un día de árbol mojado, trae un viento\\
azul polar a mi bandera fría!  

\end{verse}

\clearpage
\poemtitle {Nuevo canto de amor a Stalingrado (1943)}
\begin{verse}

Yo escribí sobre el tiempo y sobre el agua,\\
describí el luto y su metal morado,\\
yo escribí sobre el cielo y la manzana,\\
ahora escribo sobre Stalingrado.  

Ya la novia guardó con su pañuelo\\
el rayo de mi amor enamorado,\\
ahora mi corazón está en el suelo,\\
en el humo y la luz de Stalingrado.  

Yo toqué con mis manos la camisa\\
del crepúsculo azul y derrotado:\\
ahora toco el alba de la vida\\
naciendo con el sol de Stalingrado.  

Yo sé que el viejo joven transitorio\\
de pluma, como un cisne encuadernado,\\
desencuaderna su dolor notorio\\
por mi grito de amor a Stalingrado.  

Yo pongo el alma mía donde quiero.\\
Y no me nutro de papel cansado,\\
adobado de tinta y de tintero.\\
Nací para cantar a Stalingrado.  

Mi voz estuvo con tus grandes muertos\\
contra tus propios muros machacados,\\
mi voz sonó como campana y viento\\
mirándote morir, Stalingrado.  

Ahora americanos combatientes\\
blancos y oscuros como los granados\\
matan en el desierto a la serpiente.\\
Ya no estás sola, Stalingrado.  

Francia vuelve a las viejas barricadas\\
con pabellón de furia enarbolado\\
sobre las lágrimas recién secadas.\\
Ya no estás sola, Stalingrado.  

Y los grandes leones de Inglaterra\\
volando sobre el mar huracanado\\
clavan las garras en la parda tierra.\\
Ya no estás sola, Stalingrado.  

Hoy bajo tus montañas de escarmiento\\
no sólo están los tuyos enterrados:\\
temblando está la carne de los muertos\\
que tocaron tu frente, Stalingrado.  

Deshechas van las invasoras manos,\\
triturados los ojos del soldado,\\
están llenos de sangre los zapatos\\
que pisaron tu puerta, Stalingrado.  

Tu acero azul de orgullo construido,\\
tu pelo de planetas coronados,\\
tu baluarte de panes divididos,\\
tu frontera sombría, Stalingrado.  

Tu Patria de martillos y laureles,\\
la sangre sobre tu esplendor nevado,\\
la mirada de Stalin a la nieve\\
tejida con tu sangre, Stalingrado.  

Las condecoraciones que tus muertos\\
han puesto sobre el pecho traspasado\\
de la tierra, y el estremecimiento\\
de la muerte y la vida, Stalingrado.  

La sal profunda que de nuevo traes\\
al corazón del hombre acongojado\\
con la rama de rojos capitanes\\
salidos de tu sangre, Stalingrado.  

La esperanza que rompe en los jardines\\
como la flor del árbol esperado,\\
la página grabada de fusiles,\\
las letras de la luz, Stalingrado.  

La torre que concibes en la altura,\\
los altares de piedra ensangrentados,\\
los defensores de tu edad madura,\\
los hijos de tu piel, Stalingrado.  

Las águilas ardientes de tus piedras,\\
los metales por tu alma amamantados,\\
los adioses de lágrimas inmensas\\
y las olas de amor, Stalingrado.  

Los huesos de asesinos malheridos,\\
los invasores párpados cerrados,\\
y los conquistadores fugitivos\\
detrás de tu centella, Stalingrado.  

Los que humillaron la curva del Arco\\
y las aguas del Sena han taladrado\\
con el consentimiento del esclavo,\\
se detuvieron en Stalingrado.  

Los que sobre Praga la Bella en lágrimas,\\
sobre lo enmudecido y traicionado,\\
pasaron pisoteando sus heridas,\\
murieron en Stalingrado.  

Los que en la gruta griega han escupido,\\
la estalactita de cristal truncado\\
y su clásico azul enrarecido,\\
ahora dónde están, Stalingrado?  

Los que España quemaron y rompieron\\
dejando el corazón encadenado\\
de esa madre de encinos y guerreros,\\
se pudren a tus pies, Stalingrado.  

Los que en Holanda, tulipanes y agua\\
salpicaron de lodo ensangrentado\\
y esparcieron el látigo y la espada\\
ahora duermen en Stalingrado.  

Los que en la noche blanca de Noruega\\
con un aullido de chacal soltado\\
quemaron esa helada primavera\\
enmudecieron en Stalingrado.  

Honor a ti por lo que el aire trae,\\
lo que se ha de cantar y lo cantado,\\
honor para tus madres y tus hijos\\
y tus nietos, Stalingrado.  

Honor al combatiente de la bruma,\\
honor al Comisario y al soldado,\\
honor al cielo detrás de tu luna,\\
honor al sol de Stalingrado.  

Guárdame un trozo de violenta espuma,\\
guárdame un rifle, guárdame un arado,\\
y que los pongan en mi sepultura\\
con una espiga roja de tu estado,\\
para que sepan, si hay alguna duda,\\
que he muerto amándote y que me has amado,\\
y si no he combatido en tu cintura\\
dejo en tu honor esta granada oscura,\\
este canto de amor a Stalingrado.  

\end{verse}

\clearpage
\poemtitle {Juventud (1943)}
\begin{verse}

Un perfume como una ácida espada\\
de ciruelas en un camino,\\
los besos del azúcar en los dientes,\\
las gotas vitales resbalando en los dedos,\\
la dulce pulpa erótica,\\
las eras, los pajares, los incitantes\\
sitios secretos de las casas anchas,\\
los colchones dormidos en el pasado, el agrio valle verde\\
mirado desde arriba, desde el vidrio escondido:\\
toda la adolescencia mojándose y ardiendo\\
como una lámpara derribada en la lluvia.  

\end{verse}

\clearpage
\poemtitle {En la soberbia la espina (Tres sonetos para Laureano Gómez) (1943)}
\begin{verse}

{\bfseries\scshape {I}}

Adiós, Laureano nunca laureado.\\
Sátrapa triste, rey advenedizo.\\
Adiós, emperador de cuarto piso\\
antes de tiempo y sin cesar pagado.  

Administras las tumbas del pasado,\\
y, hechizado, aprovechas el hechizo\\
en el agusanado paraíso\\
donde llega el soberbio derrotado.  

Allí eres dios sin luz ni primavera.\\
Allí eres capitán de gusanera,\\
y en la terrible noche del arcano  

el cetro de violencia que te espera\\
caerá podrido como polvo y cera\\
bajo la jerarquía del gusano.  

{\bfseries\scshape {II}}

Caballero del látigo mezquino,\\
excomulgado por el ser humano,\\
iracunda piltrafa del camino,\\
oh pequeño anticristo anticristiano.  

Como tú, con el látigo en la mano,\\
tiembla en España Franco, el asesino,\\
y en la Alemania tu sangriento hermano\\
lee sobre la nieve su destino.  

Es tarde para ti, triste Laureano.\\
Quedarás como cola de tirano\\
en el museo de lo que no existe,\\
en tu pequeño parque de veneno\\
con tu pistola que dispara cieno.\\
Tes vas antes de ser. Tarde viniste!  

{\bfseries\scshape {III}}

Donde estén la canción y el pensamiento,\\
donde bailen o canten los poetas,\\
donde la lira diga su lamento\\
no te metas, Laureano, no te metas.  

Las críticas que aúllas en el viento,\\
la estricnina que llena tus maletas,\\
te las devolverán con escarmiento\\
no te metas, Laureano, no te metas.  

No toques con tus pies la geografía\\
de la verdad o de la poesía,\\
no está en lo verdaderamente tu terreno.  

Vuelve al látigo, vuelve a la amargura,\\
vuelve a tu recorosa sepultura.\\
Que no nos abandone tu veneno!  
\end{verse}
\end{document}
