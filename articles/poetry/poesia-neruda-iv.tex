% rubber: module xelatex
\documentclass[12pt]{article}

\usepackage[spanish]{babel}
\usepackage[utf8]{inputenc}
% \usepackage{palatino}
\usepackage{verse, gmverse}

\usepackage{fontspec}
\setmainfont{EB Garamond}

\date{}
\title{Poesía selecta de Pablo Neruda IV}
\begin{document}
\maketitle
\tableofcontents

\clearpage
\poemtitle{ Sólo el hombre (1951)}
\begin{verse}
Yo atravesé las hostiles  
cordilleras,  
entre los árboles pasé a caballo.  
El humus ha dejado  
en el suelo  
su alfombra de mil años.  

Los árboles se tocan en la altura,  
en la unidad temblorosa.  
Abajo, oscura es la selva.  
Un vuelo corto, un grito  
la atraviesan,  
los pájaros del frío,  
los zorros de eléctrica cola,  
una gran hoja que cae,  
y mi caballo pisa el blando  
lecho del árbol dormido,  
pero bajo la tierra  
los árboles de nuevo  
se entienden y se tocan.  
La selva es una sola,  
un solo gran puñado de perfume,  
una sola raíz bajo la tierra.  

Las púas me mordían,  
las duras piedras herían mi caballo,  
el hielo iba buscando bajo mi ropa rota  
mi corazón para cantarle y dormirlo.  
Los ríos que nacían  
ante mi vista bajaban veloces  
y querían matarme.  
De pronto un árbol ocupaba el camino  
como si hubiera  
echado a andar y entonces  
lo hubiera derribado  
la selva, y allí estaba  
grande como mil hombres,  
lleno de cabelleras,  
pululado de insectos,  
podrido por la lluvia,  
pero desde la muerte  
quería detenerme.  

Yo salté el árbol,  
lo rompí con el hacha,  
acaricié sus hojas hermosas como manos,  
toqué las poderosas  
raíces que mucho más que yo  
conocían la tierra.  
Yo pasé sobre el árbol,  
crucé todos los ríos,  
la espuma me llevaba,  
las piedras me mentían,  
el aire verde que creaba  
alhajas a cada minuto  
atacaba mi frente,  
quemaba mis pestañas.  
Yo atravesé las altas cordilleras  
porque conmigo un hombre,  
otro hombre, un hombre  
iba conmigo.  

No venían los árboles,  
no iba conmigo el agua  
vertiginosa que quiso matarme,  
ni la tierra espinosa.  
Solo el hombre,  
solo el hombre estaba conmigo.  
No las manos del árbol,  
hermosas como rostros, ni las graves  
raíces que conocen la tierra  
me ayudaron.  
Solo el hombre.  
No sé cómo se llama.  
Era tan pobre como yo, tenía  
ojos como los míos, y con ellos  
descubría el camino  
para que otro hombre pasara.  
Y aquí estoy.  
Por eso existo.  

Creo  
que no nos juntaremos en la altura.  
Creo  
que bajo la tierra nada nos espera,  
pero sobre la tierra  
vamos juntos.  
Nuestra unidad está sobre la tierra.  

\end{verse}

\clearpage
\poemtitle{ El río (1951)}
\begin{verse}
Yo entré en Florencia. Era  
de noche. Temblé escuchando  
casi dormido lo que el dulce río  
me contaba. Yo no sé  
lo que dicen los cuadros ni los libros  
(no todos los cuadros ni todos los libros,  
solo algunos),  
pero sé lo que dicen  
todos los ríos.  
Tienen el mismo idioma que yo tengo.  
En las tierras salvajes  
el Orinoco me habla  
y entiendo, entiendo  
historias que no puedo repetir.  
Hay secretos míos  
que el río se ha llevado,  
y lo que me pidió lo voy cumpliendo  
poco a poco en la tierra.  
Reconocí en la voz del Arno entonces  
viejas palabras que buscaban mi boca,  
como el que nunca conoció la miel  
y halla que reconoce su delicia.  
Así escuché las voces  
del río de Florencia,  
como si antes de ser me hubieran dicho  
lo que ahora escuchaba:  
sueños y pasos que me unían  
a la voz del río,  
seres en movimiento,  
golpes de luz en la historia,  
tercetos encendidos como lámparas.  
El pan y la sangre cantaban  
con la voz nocturna del agua.  

\end{verse}

\clearpage
\poemtitle{ La reína (1951)}
\begin{verse}
Yo te he nombrado reina.  
Hay más altas que tú, más altas.  
Hay más puras que tú, más puras.  
Hay más bellas que tú, hay más bellas.  
Pero tú eres la reina.  

Cuando vas por las calles  
nadie te reconoce.  
Nadie ve tu corona de cristal, nadie mira  
la alfombra de oro rojo  
que pisas donde pasas,  
la alfombra que no existe.  

Y cuando asomas  
suenan todos los ríos  
en mi cuerpo, sacuden  
el cielo las campanas,  
y un himno llena el mundo.  

Solo tú y yo,  
solo tú y yo, amor mío,  
lo escuchamos.  

\end{verse}

\clearpage
\poemtitle{ El hijo (1951)}
\begin{verse}
Ay hijo, sabes, sabes  
de dónde vienes?  

De un lago con gaviotas  
blancas y hambrientas.  
Junto al agua de invierno  
ella y yo levantamos  
una fogata roja  
gastándonos los labios  
de besarnos el alma,  
echando al fuego todo,  
quemándonos la vida.  

Así llegaste al mundo.  

Pero ella para verme  
y para verte un día  
atravesó los mares  
y yo para abrazar  
su pequeña cintura  
toda la tierra anduve,  
con guerras y montañas,  
con arenas y espinas.  

Así llegaste al mundo.  

De tantos sitios vienes,  
del agua y de la tierra,  
del fuego y de la nieve,  
de tan lejos caminas  
hacia nosotros dos,  
desde el amor terrible  
que nos ha encadenado,  
que queremos saber  
cómo eres, qué nos dices,  
porque tú sabes más  
del mundo que te dimos.  

Como una gran tormenta  
sacudimos nosotros  
el árbol de la vida  
hasta las más ocultas  
fibras de las raíces  
y apareces ahora  
cantando en el follaje,  
en la más alta rama  
que contigo alcanzamos.  

\end{verse}

\clearpage
\poemtitle{ El hombre invisible (1952)}
\begin{verse}
\itshape
Yo me río,
me sonrío
de los viejos poetas,
yo adoro toda
la poesía escrita,
todo el rocío,
luna, diamante, gota
de plata sumergida
que fue mi antiguo hermano
agregando a la rosa,
pero
me sonrío,
siempre dicen «yo»,
a cada paso
les sucede algo,
es siempre «yo»,
por las calles
solo ellos andan
o la dulce que aman,
nadie más,
no pasan pescadores,
ni libreros,
no pasan albañiles,
nadie se cae
de un andamio,
nadie sufre,
nadie ama,
solo mi pobre hermano,
el poeta,
a él le pasan
todas las cosas
y a su dulce querida,
nadie vive
sino él solo,
nadie llora de hambre
o de ira,
nadie sufre en sus versos
porque no puede
pagar el alquiler,
a nadie en poesía
echan a la calle
con camas y con sillas
y en las fábricas
tampoco pasa nada,
no pasa nada,
se hacen paraguas, copas,
armas, locomotoras,
se extraen minerales
rascando el infierno,
hay huelga,
vienen soldados,
disparan,
disparan contra el pueblo,
es decir,
contra la poesía,
y mi hermano
el poeta
estaba enamorado,
o sufría
porque sus sentimientos
son marinos,
ama los puertos
remotos, por sus nombres,
y escribe sobre océanos
que no conoce,
junto a la vida, repleta
como el maíz de granos,
él pasa sin saber
desgranarla,
él sube y baja
sin tocar la tierra,
o a veces
se siente profundísimo
y tenebroso,
él es tan grande
que no cabe en sí mismo,
se enreda y desenreda,
se declara maldito,
lleva con gran dificultad la cruz
de las tinieblas,
piensa que es diferente
a todo el mundo,
todos los días come pan
pero no ha visto nunca
un panadero
ni ha entrado a un sindicato
de panificadores,
y así mi pobre hermano
se hace oscuro,
se tuerce y se retuerce
y se halla
interesante,
interesante,
esta es la palabra,
yo no soy superior
a mi hermano
pero sonrío,
porque voy por las calles
y sólo yo no existo,
la vida corre
como todos los ríos,
yo soy el único
invisible,
no hay misteriosas sombras,
no hay tinieblas,
todo el mundo me habla,
me quieren contar cosas,
me hablan de sus parientes,
de sus miserias
y de sus alegrías,
todos pasan y todos
me dicen algo,
y cuántas cosas hacen!
cortan maderas,
suben hilos eléctricos,
amasan hasta tarde en la noche
el pan de cada día,
con una lanza de hierro
perforan las entrañas
de la tierra
y convierten el hierro
en cerraduras,
suben al cielo y llevan
cartas, sollozos, besos,
en cada puerta
hay alguien,
nace alguno,
o me espera la que amo,
y yo paso y las cosas
me piden que las cante,
yo no tengo tiempo,
debo pensar en todo,
debo volver a casa,
pasar al Partido,
qué puedo hacer,
todo me pide
que hable,
todo me pide
que cante y cante siempre,
todo está lleno
de sueños y sonidos,
la vida es una caja
llena de cantos, se abre
y vuela y viene
una bandada
de pájaros
que quieren contarme algo
descansando en mis hombros,
la vida es una lucha
como un río que avanza
y los hombres
quieren decirme,
decirte,
por qué luchan,
si mueren,
por qué mueren,
y yo paso y no tengo
tiempo para tantas vidas,
yo quiero
que todos vivan
en mi vida
y canten en mi canto,
yo no tengo importancia,
no tengo tiempo
para mis asuntos,
de noche y de día
debo anotar lo que pasa,
y no olvidar a nadie.
Es verdad que de pronto
me fatigo
y miro las estrellas,
me tiendo en el pasto, pasa
un insecto color de violín,
pongo el brazo
sobre un pequeño seno
o bajo la cintura
de la dulce que amo,
y miro el terciopelo
duro
de la noche que tiembla
con sus constelaciones congeladas,
entonces
siento subir a mi alma
la ola de los misterios,
la infancia,
el llanto en los rincones,
la adolescencia triste,
y me da sueño,
y duermo
como un manzano,
me quedo dormido
de inmediato
con las estrellas o sin las estrellas,
con mi amor o sin ella,
y cuando me levanto
se fue la noche,
la calle ha despertado antes que yo,
a su trabajo
van las muchachas pobres,
los pescadores vuelven
del océano,
los mineros
van con zapatos nuevos
entrando en la mina,
todo vive,
todos pasan,
andan apresurados,
y yo tengo apenas tiempo
para vestirme,
yo tengo que correr:
ninguno puede
pasar sin que yo sepa
adónde va, qué cosa
le ha sucedido.
No puedo
sin la vida vivir,
sin el hombre ser hombre
y corro y veo y oigo
y canto,
las estrellas no tienen
nada que ver conmigo,
la soledad no tiene
flor ni fruto.
Dadme para mi vida
todas las vidas,
dadme todo el dolor
de todo el mundo,
yo voy a transformarlo
en esperanza.
Dadme
todas las alegrías,
aun las más secretas,
porque si así no fuera,
cómo van a saberse?
Yo tengo que contarlas,
dadme
las luchas
de cada día
porque ellas son mi canto,
y así andaremos juntos,
codo a codo,
todos los hombres,
mi canto los reúne:
el canto del hombre invisible
que canta con todos los hombres.

\end{verse}

\clearpage
\poemtitle{ Oda al mar (1953)}
\begin{verse}
Aquí en la isla  
el mar  
y cuánto mar  
se sale de sí mismo  
a cada rato,  
dice que sí, que no,  
que no, que no, que no,  
dice que sí, en azul,  
en espuma, en galope,  
dice que no, que no.  
No puede estarse quieto,  
me llamo mar, repite  
pegando en una piedra  
sin lograr convencerla,  
entonces  
con siete lenguas verdes  
de siete perros verdes,  
de siete tigres verdes,  
de siete mares verdes,  
la recorre, la besa,  
la humedece  
y se golpea el pecho  
repitiendo su nombre.  
Oh mar, así te llamas,  
oh camarada océano,  
no pierdas tiempo y agua,  
no te sacudas tanto,  
ayúdanos,  
somos los pequeñitos  
pescadores,  
los hombres de la orilla,  
tenemos frío y hambre,  
eres nuestro enemigo,  
no golpees tan fuerte,  
no grites de ese modo,  
abre tu caja verde  
y déjanos a todos  
en las manos  
tu regalo de plata:  
el pez de cada día.  

Aquí en cada casa  
lo queremos  
y aunque sea de plata,  
de cristal o de luna,  
nació para las pobres  
cocinas de la tierra.  
No lo guardes,  
avaro,  
corriendo frío como  
relámpago mojado  
debajo de tus olas.  
Ven, ahora,  
ábrete  
y déjalo  
cerca de nuestras manos,  
ayúdanos, océano,  
padre verde y profundo,  
a terminar un día  
la pobreza terrestre.  
Déjanos  
cosechar la infinita  
plantación de tus vidas,  
tus trigos y tus uvas,  
tus bueyes, tus metales,  
el esplendor mojado  
y el fruto sumergido.  

Padre mar, ya sabemos  
cómo te llamas, todas  
las gaviotas reparten  
tu nombre en las arenas:  
ahora, pórtate bien,  
no sacudas tus crines,  
no amenaces a nadie,  
no rompas contra el cielo  
tu bella dentadura,  
déjate por un rato  
de gloriosas historias,  
danos a cada hombre,  
a cada  
mujer y a cada niño,  
un pez grande o pequeño  
cada día.  
Sal por todas las calles  
del mundo  
a repartir pescado  
y entonces  
grita,  
grita  
para que te oigan todos  
los pobres que trabajan  
y digan,  
asomando a la boca  
de la mina:  
«Ahí viene el viejo mar  
repartiendo pescado».  
Y volverán abajo,  
a las tinieblas,  
sonriendo, y por las calles  
y los bosques  
sonreirán los hombres  
y la tierra  
con sonrisa marina.  

Pero  
si no lo quieres,  
si no te da la gana,  
espérate,  
espéranos,  
lo vamos a pensar,  
vamos en primer término  
a arreglar los asuntos  
humanos,  
los más grandes primero,  
todos los otros después,  
y entonces  
entraremos en ti,  
cortaremos las olas  
con cuchillo de fuego,  
en un caballo eléctrico  
saltaremos la espuma,  
cantando  
nos hundiremos  
hasta tocar el fondo  
de tus entrañas,  
un hilo atómico  
guardará tu cintura,  
plantaremos  
en tu jardín profundo  
plantas  
de cemento y acero,  
te amarraremos  
pies y manos,  
los hombres por tu piel  
pasearán escupiendo,  
sacándote racimos,  
construyéndote arneses,  
montándote y domándote,  
dominándote el alma.  
Pero eso será cuando  
los hombres  
hayamos arreglado  
nuestro problema,  
el grande,  
el gran problema.  
Todo lo arreglaremos  
poco a poco:  
te obligaremos, mar,  
te obligaremos, tierra,  
a hacer milagros,  
porque en nosotros mismos,  
en la lucha,  
está el pez, está el pan,  
está el milagro.  

\end{verse}

\clearpage
\poemtitle{ Oda a la cebolla (1953)}
\begin{verse}
Cebolla,  
luminosa redoma,  
pétalo a pétalo  
se formó tu hermosura,  
escamas de cristal te acrecentaron  
y en el secreto de la tierra oscura  
se redondeó tu vientre de rocío.  
Bajo la tierra  
fue el milagro  
y cuando apareció  
tu torpe tallo verde,  
y nacieron  
tus hojas como espadas en el huerto,  
la tierra acumuló su poderío  
mostrando tu desnuda transparencia,  
y como en Afrodita el mar remoto  
duplicó la magnolia  
levantando sus senos,  
la tierra  
así te hizo,  
cebolla,  
clara como un planeta,  
y destinada  
a relucir,  
constelación constante,  
redonda rosa de agua,  
sobre  
la mesa  
de las pobres gentes.  

Generosa  
deshaces  
tu globo de frescura  
en la consumación  
ferviente de la olla,  
y el jirón de cristal  
al calor encendido del aceite  
se transforma en rizada pluma de oro.  

También recordaré cómo fecunda  
tu influencia el amor de la ensalada  
y parece que el cielo contribuye  
dándote fina forma de granizo  
a celebrar tu claridad picada  
sobre los hemisferios de un tomate.  
Pero al alcance  
de las manos del pueblo,  
regada con aceite,  
espolvoreada  
con un poco de sal,  
matas el hambre  
del jornalero en el duro camino.  
Estrella de los pobres,  
hada madrina  
envuelta  
en delicado  
papel, sales del suelo,  
eterna, intacta, pura  
como semilla de astro,  
y al cortarte  
el cuchillo en la cocina  
sube la única lágrima  
sin pena.  
Nos hiciste llorar sin afligirnos.  

Yo cuanto existe celebré, cebolla,  
pero para mí eres  
más hermosa que un ave  
de plumas cegadoras,  
eres para mis ojos  
globo celeste, copa de platino,  
baile inmóvil  
de anémona nevada.  

y vive la fragancia de la tierra  
en tu naturaleza cristalina.  

\end{verse}

\clearpage
\poemtitle{ Oda a la luna del mar (1955)}
\begin{verse}
Luna  
de la ciudad,  
me pareces  
cansada,  
oscura  
me pareces  
o amarilla,  
con algo  
de uña gastada  
o gancho de candado,  
cadavérica,  
vieja,  
borrascosa,  
tambaleante  
como una  
religiosa oxidada  
en el transcurso  
de las metálicas  
revoluciones:  
luna  
transmigratoria,  
respetable,  
impasible:  
tu  
palidez  
ha visto  
barricadas  
sangrientas,  
motines  
del pueblo que sacude  
sus cadenas,  
amapolas  
abiertas  
sobre  
la guerra  
y sus  
exterminados  
y allí, cansada, arriba,  
con tus párpados viejos  
cada vez  
más cansada,  
más  
triste,  
más rellena con humo,  
con sangre, con tabaco,  
con infinitas interrogaciones,  
con el sudor nocturno  
de las panaderías,  
luna  
gastada  
como  
la única muela  
del cielo  
de la noche  
desdentada.  

De pronto  
llego  
al mar  
y otra luna  
me pareces,  
blanca,  
mojada  
y fresca  
como  
yegua  
reciente  
que corre  
en el rocío,  
joven  
como una perla,  
diáfana  
como frente  
de sirena.  
Luna  
del mar,  
te lavas  
cada noche  
y amaneces  
mojada  
por una aurora eterna,  
desposándote  
sin cesar con el cielo, con el aire,  
con el viento marino,  
desarrollado cada  
nueva hora  
por el interno impulso  
vital de la marea,  
limpia como las uñas  
en la sal  
del océano.  

Oh, luna de los mares,  
luna  
mía,  
cuando  
de las calles  
regreso,  
de mi número  
vuelvo,  
tú me lavas  
el polvo,  
el sudor  
y las manchas  
del camino,  
lavandera  
marina,  
lavas  
mi corazón cansado,  
mi camisa.  
En la noche  
te miro,  
pura,  
encendida  
lámpara  
del cielo,  
fresca, recién nacida  
entre las olas,  
y me duermo  
bajo tu esfera limpia,  
reluciente,  
de universal reloj,  
de rosa blanca.  
Amanezco  
nuevo, recién vestido,  
lavado por tus manos,  
lavandera,  
buena para el trabajo  
y la batalla.  
Tal vez tu paz, tu nimbo  
nacarado,  
tu nave  
entre las olas,  
eterna, renaciendo  
con la sombra,  
tienen que ver conmigo  
y a tu fresca  
eternidad de plata  
y de marea  
debe mi corazón  
su levadura.  

\end{verse}

\clearpage
\poemtitle{ Oda a unas flores amarillas (1957)}
\begin{verse}
Contra el azul moviendo sus azules,  
el mar, y contra el cielo,  
unas flores amarillas.  

Octubre llega.  

Y aunque sea  
tan importante el mar desarrollando  
su mito, su misión, su levadura,  
estalla  
sobre la arena el oro  
de una sola  
planta amarilla  
y se amarran  
tus ojos  
a la tierra,  
huyen del magno mar y sus latidos.  

Polvo somos, seremos.  

Ni aire, ni fuego, ni agua  
sino  
tierra  
solo tierra  
seremos  
y tal vez  
unas flores amarillas.  

\end{verse}

\clearpage
\poemtitle{ Pido silencio (1957)}
\begin{verse}
\itshape
Ahora me dejen tranquilo.  
Ahora se acostumbren sin mí.  

Yo voy a cerrar los ojos.  

Y solo quiero cinco cosas,  
cinco raíces preferidas.  

Una es el amor sin fin.  
Lo segundo es ver el otoño.  
No puedo ser sin que las hojas  
vuelen y vuelvan a la tierra.  

Lo tercero es el grave invierno,  
la lluvia que amé, la caricia  
del fuego en el frío silvestre.  

En cuarto lugar el verano  
redondo como una sandía.  

La quinta cosa son tus ojos,  
Matilde mía, bienamada,  
no quiero dormir sin tus ojos,  
no quiero ser sin que me mires:  
yo cambio la primavera  
por que tú me sigas mirando.  

Amigos, eso es cuanto quiero.  
Es casi nada y casi todo.  

Ahora si quieren se vayan.  

He vivido tanto que un día  
tendrán que olvidarme por fuerza,  
borrándome de la pizarra:  
mi corazón fue interminable.  

Pero porque pido silencio  
no crean que voy a morirme:  
me pasa todo lo contrario:  
sucede que voy a vivirme.  

Sucede que soy y que sigo.  

No será, pues, sino que adentro  
de mí crecerán cereales,  
primero los granos que rompen  
la tierra para ver la luz,  
pero la madre tierra es oscura:  
y dentro de mí soy oscuro:  
soy como un pozo en cuyas aguas  
la noche deja sus estrellas  
y sigue sola por el campo.  

Se trata de que tanto he vivido  
que quiero vivir otro tanto.  

Nunca me sentí tan sonoro,  
nunca he tenido tantos besos.  

Ahora, como siempre, es temprano.  

Vuela la luz con sus abejas.  

Déjenme solo con el día.  
Pido permiso para nacer.  

\end{verse}

\clearpage
\poemtitle{ Fábula de la sirena y los borrachos (1957)}
\begin{verse}
\itshape
Todos estos señores estaban dentro  
cuando ella entró completamente desnuda  
ellos habían bebido y comenzaron a escupirla  
ella no entendía nada recién salía del río  
era una sirena que se había extraviado  
los insultos corrían sobre su carne lisa  
la inmundicia cubrió sus pechos de oro  
ella no sabía llorar por eso no lloraba  
no sabía vestirse por eso no se vestía  
la tatuaron con cigarrillos y con corchos quemados  
y reían hasta caer al suelo de la taberna  
ella no hablaba porque no sabía hablar  
sus ojos eran color de amor distante  
sus brazos construidos de topacios gemelos  
sus labios se cortaron en la luz del coral  
y de pronto salió por esa puerta  
apenas entró al río quedó limpia  
relució como una piedra blanca en la lluvia  
y sin mirar atrás nadó de nuevo  
nadó hacia nunca más hacia morir.  

\end{verse}

\clearpage
\poemtitle{ El miedo (1957)}
\begin{verse}
Todos me piden que dé saltos,  
que tonifique y que futbole,  
que corra, que nade y que vuele.  
Muy bien.  

Todos me aconsejan reposo,  
todos me destinan doctores,  
mirándome de cierta manera.  
Qué pasa?  

Todos me aconsejan que viaje,  
que entre y que salga, que no viaje,  
que me muera y que no me muera.  
No importa.  

Todos ven las dificultades  
de mis vísceras sorprendidas  
por radioterribles retratos.  
No estoy de acuerdo.  

Todos pican mi poesía  
con invencibles tenedores  
buscando, sin duda, una mosca.  
Tengo miedo.  

Tengo miedo de todo el mundo,  
del agua fría, de la muerte.  
Soy como todos los mortales,  
inaplazable.  

Por eso en estos cortos días  
no voy a tomarlos en cuenta,  
voy a abrirme y voy a encerrarme  
con mi más pérfido enemigo,  
Pablo Neruda.  

\end{verse}

\clearpage
\poemtitle{ Muchos somos (1958)}
\begin{verse}
De tantos hombres que soy, que somos,  
no puedo encontrar a ninguno:  
se me pierden bajo la ropa,  
se fueron a otra ciudad.  

Cuando todo está preparado  
para mostrarme inteligente  
el tonto que llevo escondido  
se toma la palabra en mi boca.  

Otras veces me duermo en medio  
de la sociedad distinguida  
y cuando busco en mí al valiente,  
un cobarde que no conozco  
corre a tomar con mi esqueleto  
mil deliciosas precauciones.  

Cuando arde una casa estimada  
en vez del bombero que llamo  
se precipita el incendiario  
y ese soy yo. No tengo arreglo.  
Qué debo hacer para escogerme?  

Cómo puedo rehabilitarme?  
Todos los libros que leo  
celebran héroes refulgentes  
siempre seguros de sí mismos:  
me muero de envidia por ellos,  
y en los filmes de vientos y balas  
me quedo envidiando al jinete,  
me quedo admirando al caballo.  

Pero cuando pido al intrépido  
me sale el viejo perezoso,  
y así yo no sé quién soy,  
no sé cuántos soy o seremos.  
Me gustaría tocar un timbre  
y sacar el mí verdadero  
porque si yo me necesito  
no debo desaparecerme.  

Mientras escribo estoy ausente  
y cuando vuelvo ya he partido:  
voy a ver si a las otras gentes  
les pasa lo que a mí me pasa,  
si son tantos como soy yo,  
si se parecen a sí mismos  
y cuando lo haya averiguado  
voy a aprender tan bien las cosas  
que para explicar mis problemas  
les hablaré de geografía.
\end{verse}
\end{document}
